\section{A Empresa}
\label{sec:chap02_company}

A {\companyname} é uma empresa fundada em 2009, com sede e centro de engenharia na Maia (Porto, Portugal), subsidiárias em Dresden (Alemanha), Suzhou (China), Austin (Estados Unidos da América) e um escritório comercial em Taiwan. O objetivo é proporcionar à indústria uma solução de gestão e controlo de produção, procurando reduzir os custos de produção, flexibilizar para satisfazer a procura e capacitar a organização de uma maior agilidade, visibilidade e fiabilidade~\parencite{cmf_overview}. O compromisso da empresa~\parencite{cmf_overview} foca-se no desenvolvimento de~\inquotes{soluções de vanguarda, indo de encontro aos desafios mais importantes da indústria e disponibilizar à lista crescente de clientes satisfeitos, soluções de elevado valor acrescentado, no prazo e orçamento requerido}\footnote{Tradução livre do autor. No original~\inquotes{[...] solutions that address the most urgent industry challenges and provide our growing list of satisfied customers with the highest value solution, on-time and on-budget.}.}.

A estratégia da empresa está sintetizada na sua missão, visão e valores. Se a missão descreve a razão da empresa existir, ou seja, o seu propósito, já a visão retrata o que se aspira alcançar~\parencite[pp.~65-66]{mission_vision_values_what_do_they_say}. Isto posto, a missão e visão são divulgados a seguir~\parencite{cmf_strategy}: 

\begin{itemize}
    \item 
    {
        \textit{Missão} -- \inquotes{Trazer valor através da convergência de inteligência, operações e tecnologias de automação para a Indústria 4.0.}\footnote{Tradução livre do autor. No original \inquotes{We drive business value through the convergence of intelligence, operations, and automation technologies for Industry 4.0.}.}.
    }
    \item 
    {
        \textit{Visão} -- \inquotes{Tornar a Indústria 4.0 uma realidade para todos fabricantes.}\footnote{Tradução livre do autor. No original \inquotes{We will make Industry 4.0 a reality for all manufacturers.}.}.
    }
\end{itemize}

Relativamente aos valores, são estes que suportam a visão, moldam a cultura empresarial e são a essência da sua identidade. Como tal, de seguida apresentam-se os valores da {\companyname}~\parencite{cmf_strategy}:

\begin{itemize}
    \item 
    {
        \textit{Inovação} -- \inquotes{Exceder as expectativas dos clientes através das soluções mais eficientes e de mais alto valor para indústria.}\footnote{Tradução livre do autor. No original \inquotes{We constantly exceed our customers’ expectations through the most efficient and high value-added manufacturing solutions.}.}.
    }
    \item
    {
        \textit{Agilidade} -- \inquotes{Adaptar as pessoas, processos e soluções de forma a responder à evolução do mundo da manufatura de alta tecnologia.}\footnote{Tradução livre do autor. No original \inquotes{We continuously adapt our people, processes and solutions to respond to the evolving world of high-tech manufacturing.}.}.
    }
    \item
    {
        \textit{Compromisso} -- \inquotes{Defender o sucesso contínuo dos clientes e da empresa.}\footnote{Tradução livre do autor. No original \inquotes{We champion the continued success of our customers and our company.}.}.
    }
\end{itemize}

Por conseguinte, com base na sua estratégia, o presente trabalho pretende demonstrar viabilidade e o valor da utilização da \gls{ia}, nomeadamente na área do \gls{pln}, para a interação com o {\productname} e obter informação pertinente para o utilizador final. 
