\chapter{Contexto}
\label{chap:Chapter2}

A descrição do contexto é necessária na medida em que contribui para a compreensão e resolução do problema. Por isso, no presente capítulo apresenta-se a empresa que visa ter uma solução para o problema exposto, fazendo um breve descrição do seu negócio (Secção~\ref{sec:chap2_company}), o sistema pré-existente no qual o presente trabalho se apoia, indicando as tecnologias usadas (Secção~\ref{sec:chap2_product}) e, no final, a análise do valor da solução, de forma a concluir sobre o seu propósito e relevância para o produto (Secção~\ref{sec:chap2_valueanalysis}).  

\section{A Empresa}
\label{sec:chap2_company}

A {\companyname} é uma empresa fundada em 2009, com sede e centro de engenharia na Maia (Porto, Portugal), subsidiárias em Dresden (Alemanha), Suzhou (China), Austin (Estados Unidos da América) e um escritório comercial em Taiwan. O objetivo é proporcionar à indústria uma solução de gestão e controlo de produção, procurando reduzir os custos de produção, flexibilizar para satisfazer a procura e capacitar a organização de uma maior agilidade, visibilidade e fiabilidade~\parencite{cmf_overview}. O compromisso da empresa~\parencite{cmf_overview} foca-se no desenvolvimento de~\inquotes{soluções de vanguarda, indo de encontro aos desafios mais importantes da indústria e disponibilizar à lista crescente de clientes satisfeitos, soluções de elevado valor acrescentado, no prazo e orçamento requerido}\footnote{Tradução livre do autor. No original~\inquotes{[...] solutions that address the most urgent industry challenges and provide our growing list of satisfied customers with the highest value solution, on-time and on-budget.}.}.

A estratégia da empresa está sintetizada na sua missão, visão e valores. Se a missão descreve a razão da empresa existir, ou seja, o seu propósito, já a visão retrata o que se aspira alcançar~\parencite[pp.~65-66]{mission_vision_values_what_do_they_say}. Isto posto, a missão e visão são divulgados a seguir~\parencite{cmf_strategy}: 

\begin{itemize}
    \item 
    {
        \textit{Missão} -- \inquotes{Trazer valor através da convergência de inteligência, operações e tecnologias de automação para a Indústria 4.0.}\footnote{Tradução livre do autor. No original \inquotes{We drive business value through the convergence of intelligence, operations, and automation technologies for Industry 4.0.}.}.
    }
    \item 
    {
        \textit{Visão} -- \inquotes{Tornar a Indústria 4.0 uma realidade para todos fabricantes.}\footnote{Tradução livre do autor. No original \inquotes{We will make Industry 4.0 a reality for all manufacturers.}.}.
    }
\end{itemize}

Relativamente aos valores, são estes que suportam a visão, moldam a cultura empresarial e são a essência da sua identidade. Como tal, de seguida apresentam-se os valores da {\companyname}~\parencite{cmf_strategy}:

\begin{itemize}
    \item 
    {
        \textit{Inovação} -- \inquotes{Exceder as expectativas dos clientes através das soluções mais eficientes e de mais alto valor para indústria.}\footnote{Tradução livre do autor. No original \inquotes{We constantly exceed our customers’ expectations through the most efficient and high value-added manufacturing solutions.}.}.
    }
    \item
    {
        \textit{Agilidade} -- \inquotes{Adaptar as pessoas, processos e soluções de forma a responder à evolução do mundo da manufatura de alta tecnologia.}\footnote{Tradução livre do autor. No original \inquotes{We continuously adapt our people, processes and solutions to respond to the evolving world of high-tech manufacturing.}.}.
    }
    \item
    {
        \textit{Compromisso} -- \inquotes{Defender o sucesso contínuo dos clientes e da empresa.}\footnote{Tradução livre do autor. No original \inquotes{We champion the continued success of our customers and our company.}.}.
    }
\end{itemize}

Por conseguinte, com base na sua estratégia, o presente trabalho pretende demonstrar viabilidade e o valor da utilização da \gls{IA}, nomeadamente na área do \gls{PLN}, para a interação com o {\productname} e obter informação pertinente para o utilizador final. 

\section{O Produto}
\label{sec:chap2_product}

Nos últimos anos, o mercado dos sistemas de informação empresariais tem vindo a crescer, sobretudo pela necessidade das empresas aumentarem a sua produtividade e consequentemente, melhorarem a sua competitividade. Embora sistemas \gls{ERP} sejam cada vez mais usuais nas empresas, no sentido de gerir as suas operações, estes falham quando aplicados num contexto fabril, ou seja, no \inquotes{chão de fábrica}. Os departamentos produtivos beneficiam de \textit{software} personalizado, que responda às necessidades específicas do foro produtivo/industrial~\parencite{mes_literature_review}. 

Nestas circunstâncias surge o conceito de \gls{MES}, fruto da necessidade das empresas de manufatura progredirem no mercado, num ponto de vista de melhoria da reatividade, da qualidade, dos custo de produção e dos prazos de entrega. Desse modo, as funções de um \gls{MES} estão sobretudo ligadas a atividades de manufatura, que representa uma parte substancial do valor acrescentado em empresas deste setor~\parencite{mes_literature_review}. 

Com o objetivo de apresentar o produto, nesta secção faz-se um enquadramento genérico do conceito \gls{MES} e posteriormente, foca-se o caso específico do {\productname}.

\subsection{\textit{Manufacturing Execution Systems}}

A organização MESA\footnote{\textit{Manufacturing Enterprise Solutions Association}. \url{http://www.mesa.org}}, uma comunidade mundial, sem fins lucrativos, que junta empresas de manufatura, de prestação de serviços, analistas, académicos e estudantes, com o propósito de melhorar os resultados do negócio e as operações de produção, através da implementação e implantação de tecnologias de informação e das melhores práticas de gestão, deu o primeiro passo na definição formal de \gls{MES}~\parencite{mes_explained_high_level_vision}:

\begin{quote}
    \inquotes{\textit{Os Manufacturing Execution Systems (MES) fornecem informações que possibilitam a otimização de atividades de produção, desde o lançamento do pedido até aos produtos acabados. Usando dados atualizados e precisos, o MES orienta, inicia, responde e relata as atividades da fábrica à medida que elas ocorrem. A resposta rápida, resultante das mudanças nas condições, associada ao foco na redução de atividades sem valor acrescentado, impulsiona a eficácia das operações e processos fabris. O MES melhora o retorno dos ativos operacionais, bem como o prazo de entrega, gestão de stock, margem bruta e desempenho do fluxo de caixa. O MES fornece informações críticas acerca das atividades de produção em toda a empresa e cadeia logística através de comunicações bidirecionais.}}\footnote{Tradução livre do autor. No original \inquotes{Manufacturing Execution Systems (MES) deliver information that enables the optimization of production activities from order launch to finished goods. Using current and accurate data, MES guides, initiates, responds to, and reports on plant activities as they occur. The resulting rapid response to changing conditions, coupled with a focus on reducing non value-added activities, drives effective plant operations and processes. MES improves the return on operational assets as well as on-time delivery, inventory turns, gross margin, and cash flow performance. MES provides mission-critical information about production activities across the enterprise and supply chain via bi-directional communications.}.}.
\end{quote}

Portanto, o \gls{MES} age como um intermediário entre os diversos processos existentes no \inquotes{chão de fábrica} e os sistemas de \inquotes{alto nível}, existindo comunicação bidirecional entre as camadas, como se demonstra na Figura~\ref{fig:mes_layers}. O \gls{MES} tanto pode fornecer informação acerca dos custos de produção, de indicadores de \textit{performance}, do estado das ordens de fabrico ou rendimento produtivo, como pode também obter dados sobre o planeamento das atividades fabris, parâmetros operacionais, receitas ou instruções de fabrico, por forma a inferir de forma inteligente sobre a fábrica e os seus processos~\parencite{mes_explained_high_level_vision}. Esta bidirecionalidade na comunicação e abrangência no processo produtivo faz com que o \gls{MES} tenha um papel crucial na Indústria $4.0$, já que pode acomodar a integração, descentralização e novas tecnologias, ainda que nem todos os sistemas deste tipo tenham sido desenhados dessa forma~\parencite{cmf_mes_definition}.

\begin{figure}[!ht]
    \centering
    \includegraphics[width=.75\textwidth]{ch2/assets/mes_layers.jpg}
    \caption{Ambiente \glsfirst{MES} e as suas camadas, baseado em~\textcite[p.~526]{mes_literature_review}.}
    \label{fig:mes_layers}
\end{figure}

Com o intuito de dar resposta às necessidades de diversos ambientes produtivos, as funções apresentadas na Figura~\ref{fig:mes_functions} e descriminadas a seguir são essenciais para um \gls{MES}, nomeadamente no suporte, no controlo e na rastreabilidade de cada atividade produtiva~\parencite{mes_literature_review, mes_explained_high_level_vision, introduction_mes}:

\begin{enumerate}
    \item 
    {
        \textit{Operações/Agendamento de detalhes} -- sequenciamento e distribuição temporal das atividades fabris, por forma a otimizar a \textit{performance}, com base nos recursos disponíveis;
    }
    \item
    {
        \textit{Gestão do processo} -- controlo do fluxo de trabalho, baseado nas atividades produtivas reais e planeadas;
    }
    \item
    {
        \textit{Controlo documental} -- gestão e distribuição de informação relativa a produtos, processos, ordens de fabrico, assim como recolher os certificados e condições de trabalho;
    }
    \item
    {
        \textit{Aquisição de dados} -- monitorização, recolha e tratamento de dados sobre os processos, os materiais e operações, por pessoas, máquinas ou controlos;
    }
    \item
    {
        \textit{Gestão laboral} -- supervisão no uso de pessoal de operações num determinado turno, com base nas qualificações, padrões de trabalho e na necessidade de negócio;
    }
    \item
    {
        \textit{Gestão da qualidade} -- registo e análise das características do produto e do processo face aos requisitos ideais;
    }
    \item
    {
        \textit{Expedição de unidades de produção} -- dar a ordem para envio de materiais ou ordens para certos setores da fábrica, com o intuito de iniciar um processo ou sub-processo;
    }
    \item
    {
        \textit{Gestão de manutenção} -- planeamento e execução de tarefas que visam manter o equipamento e outros ativos capazes de executar a sua tarefa, de forma eficaz;
    }
    \item
    {
        \textit{Genealogia e rastreabilidade do produto} -- monitorização do progresso das unidades, amostras ou lotes de saída, para a criação de histórico completo do produto;
    }
    \item
    {
        \textit{Análise de desempenho} -- comparação dos resultados medidos com os objetivos e métricas definidas pela corporação, pelos clientes ou órgãos reguladores;
    }
    \item
    {
        \textit{Estado e alocação de recursos} -- orientação sobre o que as pessoas, máquinas ou ferramentas devem fazer, acompanhando que já fizeram e o que estão a fazer no momento.
    }
\end{enumerate}

\begin{figure}[!ht]
    \centering
    \includegraphics[width=.9\textwidth]{ch2/assets/mes_functions.jpg}
    \caption{Funções do \glsfirst{MES} e o seu enquadramento, extraído de~\textcite{mes_literature_review}}
    \label{fig:mes_functions}
\end{figure}

As funções do \gls{MES} enunciadas servem como base para praticamente qualquer fábrica, fornecendo ferramentas a gestores de fábrica, departamentos de qualidade e manutenção, estando interrelacionadas. Por isso, tratam-se de funções críticas para a maioria dos fabricantes, na medida em que a exigência de processos novos e mais rigorosos no negócio é viável, possibilitando o sucesso no mercado~\parencite{,mes_explained_high_level_vision}. Logo, torna-se evidente que o \gls{MES} traz benefícios para as corporações, alguns alcançáveis num período curto de tempo -- aumento de eficiência e redução de custos; redução no tempo de execução de ordens de fabrico; redução dos custos associados ao trabalho; diminuição ou eliminação de papelada; redução da quantidade de material em processamento; utilização de máquinas mais eficaz --, enquanto que outros, possíveis a longo prazo -- melhoria geral dos processos; maior satisfação do cliente; melhoria na conformidade regulamentar; maior agilidade; melhoria nos prazos de entrega; maior visibilidade da cadeia logística~\parencite{cmf_mes_definition}.

\subsection{{\productname}}

O {\productname} afirma-se como o futuro do \gls{MES}. Trata-se de uma plataforma de \textit{software} inovadora, com um vasto conjunto modular de aplicações e ferramentas, que dotam os utilizadores de indústrias complexas de agilidade, visibilidade e fiabilidade. O produto adapta-se a diversos processos fabris e às suas operações, sendo fácil a sua implantação, independentemente da infraestrutura existente, permitindo o controlo de produção e de custos na empresa e cadeia logística, resultando nas seguintes vantagens~\parencite{cmf_product_overview}:

\begin{enumerate}
    \item 
    {
        Apresenta um \textbf{baixo custo total de posse}\footnote{\textit{Total Cost of Ownership} (TCO). É uma estimativa financeira usada para avaliar os custos diretos e indiretos associados a uma compra.}, visto que a empresa reduz as despesas associadas à implantação, operação e manutenção do sistema;
    }
    \item
    {
        Fornece um largo conjunto de capacidades que dão resposta aos mais variados requisitos, demonstrando a sua \textbf{cobertura funcional};
    }
    \item
    {
        \textbf{Capacita o utilizador na sua função}, sendo este capaz de desenhar e colocar em produtivo o plano da fábrica, rastreando os materiais e os detalhes do processo;
    }
    \item
    {
        É sistema modular, o que o torna \textbf{extensível, flexível e escalável}, dando aos seus utilizadores acesso a inteligência operacional, de forma fácil e rápida; 
    }
    \item
    {
        A arquitetura logicamente descentralizada, ligada à conectividade a diferentes protocolos para equipamentos e dispositivos e ao suporte de produtos preparados para \gls{IoT} e \gls{CPS}, tornam o produto \textbf{preparado para a Indústria 4.0}.
    }
\end{enumerate}

Relativamente à arquitetura do produto, demonstrada na Figura~\ref{fig:mes_framework}, a {\companyname} baseou-se nas tecnologias mais
recentes para dotar a sua plataforma da capacidade de adaptação aos diversos ambientes produtivos. Posto isto, a infraestrutura consiste em três camadas, que além de fornecerem particionamento, modularidade e escalabilidade das aplicações, foram projetadas para funcionar em conjunto. Além disso, esta foi desenhada de forma a ser customizável e extensível, dado que cada cliente pode ter os seus próprios requisitos~\parencite{cmf_mes_framework}.

\begin{figure}[!ht]
    \centering
    \includegraphics[width=.9\textwidth]{ch2/assets/mes_framework.jpg}
    \caption{Arquitetura do {\productname} e tecnologias usadas, baseado em~\textcite{cmf_mes_framework}}
    \label{fig:mes_framework}
\end{figure}

Quanto às especificidades de cada camada, são descritas de seguida, numa perspetiva de entender a responsabilidade de cada uma delas e o seu contributo para a plataforma~\parencite{cmf_mes_framework}. Já as tecnologias usadas estão especificadas na Figura~\ref{fig:mes_framework}.

\begin{itemize}
    \item 
    {
        \textit{Camada de Apresentação (Presentation Tier)} -- projetada para trazer ao utilizador uma experiência rica e interativa. Dispõe de várias capacidades (\exempligratia{permitir aos utilizadores criar a sua própria interface gráfica ou desenvolver ecrãs para um propósito em particular, numa fábrica ou setor}) e é desenvolvida com suporte multi-plataforma, executando em qualquer sistema operativo \textit{desktop} ou móvel;
    }
    \item
    {
        \textit{Camada de Negócio (Business Tier)} -- implementa e expõe todas as funcionalidades como serviços, estando disponíveis vários protocolos de comunicação. Contém uma sub-camada de orquestração usada para definir os diversos fluxos de negócio, fornecendo a capacidade de coordenação usando os objetos de negócio. Por fim, a sub-camada de objetos de negócio segue um modelo hierárquico, o qual facilita o desenvolvimento de entidades com um comportamento comum;
    }
    \item
    {
        \textit{Camada de Dados (Data Tier)} -- desenhado para suportar as capacidades de armazenamento de dados, possibilitando a integração com fontes de dados externas, geração e modificação de relatórios e mineração de dados.
    }
\end{itemize}

O {\productname} é usado em diversas indústrias, particularmente a indústria de semicondutores~\parencite{cmf_industries_semiconductor}, de equipamentos médicos~\parencite{cmf_industries_medical_devices}, de montagem eletrónica~\parencite{cmf_industries_electronics}, procurando dar resposta aos desafios inerentes a cada uma delas.

\section{Análise de Valor}
\label{sec:chap2_valueanalysis}

Até ao momento, deu-se o contexto do trabalho a ser desenvolvido, de forma a perceber a realidade atual dos sistemas de controlo de produção. Contudo, é preciso perceber qual o impacto que a solução a ser desenvolvida terá no produto e no mercado no qual se insere. Visto que o módulo a desenvolver será integrado num produto já existente, analisa-se a oportunidade de negócio que surge com a nova funcionalidade.

\subsection{O Processo de Inovação}

De acordo com~\textcite{ffe_effectivemethods_tools_techniques}, o processo de inovação, representado na Figura~\ref{fig:inovation_process}, está dividido em três áreas -- o \gls{FFE}, \gls{NPD} e a comercialização -- que correspondem às fases inerentes ao \gls{NCD}, um modelo desenvolvido por um conjunto de empresas, com o objetivo de \inquotes{[...]~fornecer uma linguagem e compreensão comum para as atividades \textit{front end}}\footnote{Tradução livre de autor. No original \inquotes{[...]~to provide a common language and insights on the front end activities.}.}~\parencite{providing_clarity_common_language_ffe}.

O \gls{FFE} representa uma oportunidade para melhoria de todo o processo de inovação, focando todas as atividades que antecedem o desenvolvimento do produto, com o propósito de potenciar o valor, a importância e a probabilidade de sucesso das fases que se seguem. Ou seja, consiste no investimento do tempo em atividades de discussão da ideia, por forma a identificar e estruturar o problema ou oportunidade~\parencite{ffe_effectivemethods_tools_techniques, ffe_theoretical_model}. Porém, as atividades inerentes ao \gls{FFE} são fundamentalmente diferentes da fase \gls{NPD}, pelo que se torna necessária a definição de vocabulário específico, permitindo a geração de conhecimento e clara distinção entre as diferentes fases do processo~\parencite{ffe_effectivemethods_tools_techniques}.

\begin{figure}[!ht]
    \centering
    \includegraphics[width=.95\textwidth]{ch2/assets/inovation_process.jpg}
    \caption{O processo de inovação, extraído de~\textcite{ffe_effectivemethods_tools_techniques}}
    \label{fig:inovation_process}
\end{figure}

O modelo \gls{NCD}, demonstrado na Figura~\ref{fig:ncd_model}, baseado num modelo relacional ao invés de um processo linear, visa providenciar uma terminologia para o \gls{FFE}~\parencite{ffe_effectivemethods_tools_techniques}. A área interna define os cinco elementos chave do \textit{Front End of Inovation}: a identificação de oportunidade (\textit{Opportunity Identification}), a análise de oportunidade (\textit{Opportunity Analysis}), a geração e enriquecimento de ideias (\textit{Idea Generation and Enrichment}), a seleção de ideias (\textit{Idea Selection}) e a definição do conceito (\textit{Concept Definition}). O motor central (\textit{Engine}) corresponde à liderança, cultura e estratégia organizacional, que suporta os elementos que compõem o \gls{FFE}, são controláveis pela organização e possibilita a interação entre eles. Já na periferia, encontram-se os fatores de influência (\textit{Influencing Factors}), geralmente incontroláveis pela organização, consistem nas capacidades organizacionais, na estratégia de negócio, no mundo exterior, nomeadamente os canais de distribuição, clientes, fornecedores, concorrentes, política governamental ou legislação, ou quaisquer fatores que possam influenciar todo o processo de inovação~\parencite{ffe_effectivemethods_tools_techniques, providing_clarity_common_language_ffe}.

\begin{figure}[!ht]
    \centering
    \includegraphics[width=.55\textwidth]{ch2/assets/ncd_model.jpg}
    \caption{A representação do modelo \glsfirst{NCD}, extraído de~\textcite{ffe_effectivemethods_tools_techniques}}
    \label{fig:ncd_model}
\end{figure}

Quanto à representação do modelo, as partes internas são designadas de elementos por oposição a processos, pois estes implicam estrutura, que pode não ser possível ser aplicada. O formato circular indica que é esperado que as ideias fluam, circulem e iterem ao longo dos elementos, por qualquer ordem ou combinação, permitindo o uso dos elementos, repetidamente. Este comportamento é intrínseco às atividades do \gls{FFE}, permitindo uma definição clara do mercado, dos requisitos, dos riscos associados e do plano de negócio, tornando mais eficazes as fases de desenvolvimento e comercialização, devido à redução do tempo total de projeto, fruto da diminuição da repetição de algumas atividades~\parencite{ffe_effectivemethods_tools_techniques}. 

\subsection{O 'Fuzzy Front End' de Inovação}

Como mencionado anteriormente, o \gls{FFE} corresponde a um conjunto de atividades geralmente caóticas, imprevisíveis e não estruturadas que antecedem o desenvolvimento de um produto~\parencite{ffe_incremental_platform_breakthrough_products}. Todavia, é preciso perceber a natureza do produto a desenvolver, de forma a melhor enquadrar o processo de inovação.

Segundo~\textcite{ffe_incremental_platform_breakthrough_products}, pode-se caracterizar os produtos de acordo com a extensão da mudança ou do processo: incremental, requer pouca mudança a nível do produto ou do processo, uma vez que geralmente consiste na redução de custos, melhoria, extensão ou reposicionamento no mercado de produtos já existentes; plataforma, estabelecem uma arquitetura básica para uma nova geração de produtos ou processos; pioneiro, envolve uma mudança significativa no processo ou produto.

O presente trabalho visa o desenvolvimento dum módulo de linguagem natural para o {\productname}, uma plataforma já estabelecida, ou seja, trata-se de uma extensão ao produto já existente, enquadrando-se no tipo incremental. A ideia surge do processo de planeamento estratégico da empresa com a finalidade de trazer novas funcionalidades aos seus clientes, melhorando a qualidade do produto. Portanto, nas secções seguintes, aplica-se a metodologia explicitada, no sentido de enriquecer a proposta de projeto apresentada pela {\companyname}.

\subsubsection*{Identificação da Oportunidade}

O \gls{PLN} é uma área de investigação que explora a forma como os computadores podem manipular a linguagem natural (texto ou voz) para executar determinadas tarefas. Aplica-se em diversos campos de estudo: tradução, processamento de texto, interfaces com o utilizador, reconhecimento de voz, sistemas periciais~\parencite{nlp}.

\textcite{end_to_end_neural_nli_databases} menciona que, apesar da expressividade da \gls{SQL}, os utilizadores necessitam de algum conhecimento técnico para perceber como extrair informação de um sistema, o que conduziu à investigação para o desenvolvimento de interfaces alternativas que permitam aos utilizadores, sem conhecimento técnico, explorar e interagir com os dados, de forma conveniente. Também \textcite{towards_theory_nli_databases} menciona que a necessidade de interfaces de linguagem natural se torna mais evidente, devido ao número de pessoas sem conhecimentos técnicos que acedem a informação através de \textit{browsers} ou telemóveis, tornando paradigmas como o reconhecimento de voz mais atrativos.

Nesse sentido, a {\companyname} tenciona o desenvolvimento do módulo de linguagem natural para que os utilizadores do produto, sem conhecimento orientado às tecnologias de informação, possam fácil, rápida e intuitivamente consultar o sistema. Desta forma, a funcionalidade destaca o produto pelo uso de novas tecnologias, facilita-se a interação com o sistema, reduzindo-se o tempo de formação técnica associado ao mesmo. 

\subsubsection*{Análise da Oportunidade}

A pesquisa realizada por~\textcite{roadmap_nlp_research_is}, apresentada na Figura~\ref{fig:number_articles_per_year_nlp}, cuja metodologia consistiu na pesquisa de termos como \inquotes{Natural Language Processing} e \inquotes{NLP} em bases de dados académicas, determina que tem havido uma tendência crescente de interesse por esta área. Nos últimos anos, a quantidade de dados textuais disponíveis nas redes sociais ou em sistemas de comunicação, juntamente com a necessidade de acesso a informação, contribuíram para o avanço e adoção comercial do \gls{PLN}~\parencite{roadmap_nlp_research_is}.

\begin{figure}[!ht]
    \centering
    \includegraphics[width=.9\textwidth]{ch2/assets/number_articles_nlp.jpg}
    \caption{Número de artigos de \glsfirst{PLN} pesquisados por ano, extraído de~\textcite{roadmap_nlp_research_is}}
    \label{fig:number_articles_per_year_nlp}
\end{figure}

Quanto ao segmento de mercado no qual se integra, cresce a visão de fábricas inteligentes, associadas à quarta revolução industrial, prezando a integração do operador humano num ambiente complexo e rico em dados~\parencite{social_factory}. \textcite{industry40_revolution_future_mes} afirma que a revolução supracitada é já conhecida pelas empresas, o que lhe permite tomar ações no sentido de definir o seu modelo de fabrico e o seu plano de transformação, particularmente na adaptação do \gls{MES} de forma a manter o desempenho, qualidade e agilidade nas desafios espoletados pelas empresas de manufatura. Portanto, a interação entre o ser humano e o sistema pode melhorar o processo de fabrico e potenciar o negócio, na medida em que o operador, em vez do trabalho manual repetitivo que pode facilmente ser automatizado, passa a tomar decisões no processo para resolução de problemas, as quais requerem acesso à informação correta e de forma atempada~\parencite{social_factory}. É nesse sentido que o {\productname} ganha vantagem com o desenvolvimento desta nova funcionalidade.

\subsubsection*{Geração, Enriquecimento e Seleção de Ideias}

No seguimento deste assunto, foram realizadas duas reuniões com o supervisor do projeto na {\companyname}, em que foram discutidos alguns requisitos operacionais e de usabilidade, restrições ao desenvolvimento da solução, como a preferência por uso de ferramentas de \gls{PLN} que possam ser mantidas internamente e a sua facilidade de utilização, e ideias para futuras implementações, as quais podem ter um impacto na especificação arquitetural do protótipo.

\begin{figure}[!ht]
    \centering
    \resizebox{\textwidth}{!}{\begin{tikzpicture}
  \path[mindmap, concept color=black!50,text=white,
    every node/.append style={concept, minimum size=0.5cm, inner sep=0.2mm},
    level 1 concept/.append style={text width=1.5cm,font=\scriptsize},
    level 2 concept/.append style={text width=1.27cm,font=\tiny\bfseries,level distance=50}
  ]
    % Root
    node[text width=2.3cm,font=\small] {Projeto}
    %
    [clockwise from=0]
    child[concept color=black!25,text=black,level distance=80] { node {Restrições}
      [clockwise from=120]
      child { node {Uso Interno} }
      child { node {Custo} }
      child { node {Eficiência} }
      child { node {Aprendizagem da Ferramenta} }
      child { node {Usabilidade da Solução} }
    }
    %
    child[concept color=black!25,text=black,level distance=115] { node {Requisitos}
      [clockwise from=0]
      child { node {Semântica Temporal} }
      child { node {Semântica de Domínio} }
      child { node {Integração com Produto} }
      child { node {Auto-aprendizagem} }
    }
    %
    child[concept color=black!25,text=black,level distance=90] { node {Estado da arte} 
      [clockwise from=-115]
      child { node {Soluções Análogas} }
      child { node {Ferramentas} }
    }
    %
    child[concept color=black!25,text=black,level distance=75] { node {Tecnologia}
      [counterclockwise from=75]
      child { node {PLN} }
      child { node {Interfaces de Linguagem Natural} }
      child { node {\textit{Data Warehouses}} }
    }
    %
    child[concept color=black!25,text=black,level distance=125] { node {Clientes}
      [counterclockwise from=-30]
      child { node {Semi\\condutores} }
      child { node {Equipamentos Médicos} }
      child { node {Montagem Eletrónica} }
      child { node {Energia Solar} }
      child { node {Outros} }
    }
    child[concept color=black!25,text=black,level distance=135] { node {Utilizadores}
      [counterclockwise from=-35]
      child { node {Engenheiros de Produção} }
      child { node {Engenheiros de Qualidade} }
      child { node {Responsáveis de Linha} }
      child { node {Operadores} }
    };
\end{tikzpicture}}
    \caption{\textit{Mindmap} das ideias geradas}
    \label{fig:mindmap}
\end{figure}

Em relação às ideias e conceitos contempladas no \textit{mindmap} da Figura~\ref{fig:mindmap}, o presente projeto pretende dar resposta a praticamente todos, tendo em consideração que, numa fase inicial, o cumprimento de todos é praticamente inatingível. A descrição de cada conceito é feito de seguida:

\begin{itemize}
    \item
    {
        \textit{Tecnologia} -- a ideia inerente ao trabalho assenta sobre as temáticas de \gls{PLN}, especificamente Interfaces de Linguagem Natural, e \textit{Data Warehouses}. Esta consiste no estudo aprofundado deste tipo de interfaces orientado à consulta em armazéns de dados e disseminação do conhecimento internamente, para que no futuro, o projeto possa ter continuidade;
    }
    \item
    {
        \textit{Estado da Arte} -- abordagem de ferramentas e soluções análogas, com o objetivo de especificar uma arquitetura para o sistema. Este processo dá origem aos documentos de especificação que devem ser usados para consulta por parte dos desenvolvedores, quer numa perspetiva de conhecimento arquitetural, quer das ferramentas que são usadas;
    }
    \item
    {
        \textit{Clientes} -- uma vez que a {\companyname} possui clientes com diferentes realidades, a ideia é que o módulo final esteja preparado para elaborar consultas em qualquer domínio, de forma configurada ou cerne da solução. Contudo, como já abordado anteriormente, no contexto deste trabalho, apenas um domínio será considerado;
    }
    \item
    {
        \textit{Utilizadores} -- a solução deverá responder às necessidades de qualquer utilizador, desde os mais técnicos (Engenheiros de Produção) aos menos técnicos (Operadores). Porém, o protótipo terá em consideração os utilizadores mais comuns do {\productname};
    }
    \item
    {
        \textit{Restrições} -- nesta temática, foram discutidas alternativas de como avaliar o custo de uma ferramenta proprietária, uso de ferramentas \textit{Open Source} ou o desenvolvimento interno da própria biblioteca de \gls{PLN}, para garantir que não existem dependências externas à plataforma, a eficiência da solução em contexto produtivo, a facilidade de aprendizagem e usabilidade da mesma. Assim, a usabilidade da solução será estudada a partir do mecanismo de \textit{feedback} provido no módulo e através de inquéritos aos utilizadores, o que também se aplica para a aprendizagem da ferramenta. Relativamente aos restantes tópicos, não houve conclusão acerca das ideias a serem selecionadas;
    }
    \item
    {
        \textit{Requisitos} -- pressupõe-se o uso de auto-aprendizagem para adaptação automática do módulo ao \textit{feedback} do utilizador, ainda que para o protótipo, a resposta a uma simples pergunta como \inquotes{A resposta obtida foi-lhe útil?} é suficiente. Também a integração com o produto, quer a nível aplicacional, quer a nível de processo deve ser considerada, o que resultará na organização de \textit{meetups} com as equipas responsáveis pelo processo de manutenção da plataforma. Quanto aos restantes conceitos, não houve conclusão acerca das ideias selecionadas.
    }
\end{itemize}

\subsubsection*{Definição do Conceito}

Todo o processo presente no modelo \gls{NCD} de \gls{FFE} culmina com a definição do conceito, a fase que encaminha o projeto para a implementação~\parencite{ffe_effectivemethods_tools_techniques}.

O presente trabalho, denominado de \inquotes{Natural Language Querying} consiste no desenvolvimento de um módulo de linguagem natural para interface com o {\productname}. Este módulo permitirá a consulta e pesquisa de estados do processo de fabrico por utilizadores com pouco ou nenhum conhecimento associado a tecnologias de informação, garantindo a interação com o sistema de uma forma simples, fácil e intuitiva, melhorando o processo numa perspetiva de apoio à decisão (ver Secção~\ref{sec:pre1_problem}). Os objetivos deste projeto estão descritos na Secção~\ref{sec:pre1_objectives}, a metodologia e critérios de sucesso na Secção~\ref{sec:pre1_solutionevaluation}, e o respetivo plano de trabalho no Capítulo~\ref{chap:pre7}.

O projeto traz benefícios para a empresa e o seu produto, pela adoção de tecnologia de \gls{PLN} num contexto industrial, pela melhoria de usabilidade do sistema e pela evolução no processo de apoio à decisão dos seus clientes. Uma vez que o {\productname} é um produto bem posicionado no mercado, não se esperam riscos a nível comercial. Contudo, a uso de tecnologia recente, cujos conceitos não estão totalmente estudados, pode provocar atrasos no desenvolvimento do projeto ou incumprimento do orçamento definido. Não obstante, o projeto avança com o desenvolvimento de um protótipo, fase que decidirá a inclusão do módulo na plataforma da {\companyname}.