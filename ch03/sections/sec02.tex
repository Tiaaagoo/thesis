\section{Casos de Estudo}
\label{sec:chap03_marketstudy}

A investigação neste campo de estudo tem vindo a desenvolver-se desde o século XX~\parencite{survey_nlidb}. Assim sendo, é importante apresentar e examinar os casos mais pertinentes para o protótipo em desenvolvimento neste trabalho, na perspetiva de perceber quais as inovações que cada um deles trouxe para a área das \glspl{ilnbd} e em que medida se enquadram com o problema em resolução.

\subsection{LUNAR}

O LUNAR é um sistema que dá resposta ao domínio de amostras de rochas trazidas da lua e foi o primeiro sistema \gls{ilnbd}~\parencite{nlidb_brief_review, survey_nlidb}. O desenvolvimento deste sistema surgiu da necessidade de possibilitar aos cientistas envolvidos no estudo das rochas lunares poderem obter informação para formular e testar as suas hipóteses, de uma forma simples e intuitiva. O LUNAR permitia ao cientista executar diversas ações como fazer questões, computar médias e taxas, criar listas baseadas em critérios de seleção ou comparar medidas de diferentes investigadores, usando informação de duas bases de dados, uma contendo dados de análises químicas e a outra com dados de referências bibliográficas. Apesar de ter sido desenvolvido como protótipo, este sistema apresentou um desempenho satisfatório, sendo que cerca de 78\% dos pedidos foram respondidos com sucesso~\parencite{lunar_sciences_nlis}.

\subsection{LADDER}

O LADDER foi um sistema desenhado para consultar informação sobre navios da Marinha Americana, por forma a auxiliar os gestores da Marinha no processo de tomada de decisão~\parencite{nlidb_brief_review, developing_nli_complex_data}. O sistema, que usa gramática semântica para tratar \textit{queries} a uma base de dados distribuída, apresenta uma arquitetura de três camadas, cada uma correspondente a um componente do sistema: o INLAND -- \textit{Infomal Natural Language Access to Navy Data} --, é responsável por aceitar a \textit{query} de linguagem natural, produzir a respetiva \textit{query} de base de dados a partir da decomposição da mesma em fragmentos, sendo posteriormente combinados para unidades sintáticas a alto nível, para que sejam reconhecidas, dando origem a um comando enviado para o próximo componente; o IDA -- \textit{Intelligent Data Access} --, compõe uma resposta com base no comando recebido e organiza a sequência correta de \textit{queries} a realizar; o FAM -- \textit{File Access Manager} --, o último componente, tem a responsabilidade de gerir o acesso à base de dados distribuída~\parencite{developing_nli_complex_data}.

\subsection{CHAT-80}

Segundo \textcite{nlidb_brief_review}, o CHAT-80 é um dos sistemas \gls{pln} mais referenciados nos anos 80. O CHAT-80 foi desenvolvido pensando na adaptabilidade a diversos domínios, de forma fácil e eficiente. Foi implementado em \textit{Prolog} e incluía uma base de conhecimento com factos geográficos de mais de 150 países (domínio de geografia mundial) e vocabulário inglês suficiente para interação com uma base de dados, que neste caso específico seria implementada totalmente em \textit{Prolog}. Os autores concordaram que a aplicação devia lidar com um conjunto restrito de linguagem natural relevante para o domínio, uma vez que dessa forma se torna uma linguagem de \textit{query} formal mas acessível para o utilizador~\parencite{efficient_easily_adaptable_system_interpreting_nlq}.

\subsection{JANUS}

\subsection{NALIX}

\subsection{PRECISE}

\subsection{WASP}

\subsection{Sumário do Casos de Estudo}