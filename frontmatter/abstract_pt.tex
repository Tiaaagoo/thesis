\begin{abstract}
O paradigma de interação entre Homem e Máquina tem vindo a mudar nos últimos anos. Se ao longo das últimas décadas, o ser humano tem vindo a interagir com o computador através da escrita (linha de comandos) ou das interfaces gráficas, mais recentemente, surge a interação por linguagem natural. Como potenciar a comunicação entre o Homem e os sistemas usados diariamente, usando linguagem natural? Recorrendo ao Processamento de Linguagem Natural, um campo de estudo ligado à Inteligência Artificial, que pode envolver técnicas de \textit{Machine Learning} ou \textit{Deep Learning}, torna-se possível transformar a linguagem do ser humano numa representação adaptada aos sistemas computacionais. 

A presente tese debruça-se na conceção de uma abordagem que permita a consulta e apresentação de informação contida em armazéns de dados, recorrendo a linguagem natural. Neste âmbito, foi desenvolvido um protótipo que assenta sobre a abordagem conceptualizada. Assim, o intuito final é adaptar e usar a abordagem proposta no desenvolvimento dum módulo de linguagem natural para interface com o {\productname}, procurando aprimorar a usabilidade do sistema.
\end{abstract}
