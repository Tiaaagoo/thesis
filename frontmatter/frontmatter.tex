% we include the glossary here (frontmatter is included with \input, so this command is as if it was in main.tex)
% Acrónimos
\newacronym{iot}{IoT}{\textit{Internet of Things}}
\newacronym{ios}{IoS}{\textit{Internet of Services}}
\newacronym{ihc}{IHC}{Interação Humano-Computador}
\newacronym{ia}{IA}{Inteligência Artificial}
\newacronym{pln}{PLN}{Processamento de Linguagem Natural}
\newacronym{mes}{MES}{\textit{Manufacturing Execution System}}
\newacronym{sql}{SQL}{\textit{Structured Query Language}}
\newacronym{uml}{UML}{\textit{Unified Modeling Language}}
\newacronym{ceo}{CEO}{\textit{Chief Executive Officer}}
\newacronym{cto}{CTO}{\textit{Chief Technical Officer}}
\newacronym{erp}{ERP}{\textit{Enterprise Resource Planning}}
\newacronym{cps}{CPS}{\textit{Cyber-Physical System}}
\newacronym{ffe}{FFE}{\textit{Fuzzy Front End}}
\newacronym{npd}{NPD}{\textit{New Product Development}}
\newacronym{ncd}{NCD}{\textit{New Concept Development}}
\newacronym{slp}{SLP}{\textit{Single Layer Perceptron}}
\newacronym{ilnbd}{ILNBD}{Interfaces de Linguagem Natural para Bases de Dados}
\newacronym{atn}{ATN}{\textit{Augmented Transition Network}}
\newacronym{xml}{XML}{\textit{Extensible Markup Language}}
% for defining plural form
% \newacronym[shortplural=aa,longplural=letters a]{a}{A}{the a}

\frontmatter % Use roman page numbering style (i, ii, iii, iv...) for the pre-content pages

\pagestyle{plain} % Default to the plain heading style until the thesis style is called for the body content

%----------------------------------------------------------------------------------------
%	TITLE PAGE
%----------------------------------------------------------------------------------------

\maketitlepage

%----------------------------------------------------------------------------------------
%	DEDICATION  (optional)
%----------------------------------------------------------------------------------------
%\dedicatory{For/Dedicated to/To my\ldots}
\begin{dedicatory}
\tbd
\end{dedicatory}

%----------------------------------------------------------------------------------------
%	ABSTRACT PAGE
%----------------------------------------------------------------------------------------
\begin{abstract}

O paradigma de interação entre Homem e Máquina tem vindo a mudar nos últimos anos. Se ao longo das últimas décadas, o ser humano tem vindo a interagir com o computador através da escrita (linha de comandos) ou das interfaces gráficas, mais recentemente, surge a interação por linguagem natural. Como potenciar a comunicação entre o Homem e os sistemas usados diariamente, usando linguagem natural? Recorrendo ao Processamento de Linguagem Natural, um campo de estudo ligado à Inteligência Artificial, que pode envolver técnicas de \textit{Machine Learning} ou \textit{Deep Learning}, torna-se possível transformar a linguagem do ser humano numa representação adaptada aos sistemas computacionais. 

A presente tese debruça-se na conceção de uma abordagem que permita a consulta e apresentação de informação contida em armazéns de dados, recorrendo a linguagem natural. Neste âmbito, foi desenvolvido um protótipo que assenta sobre a abordagem conceptualizada. Assim, o intuito final é adaptar e usar a abordagem proposta no desenvolvimento dum módulo de linguagem natural para interface com o {\productname}, procurando aprimorar a usabilidade do sistema.

\end{abstract}

\begin{abstractotherlanguage}
The interaction paradigm between man and machine has been changing in the last years. Over the last decades, humans have been interacting with the computer through writing (command line) or graphical interfaces. Recently, emerges the interaction through natural language. How to enhance the communication between man and the system used on daily basis, by using natural language? The usage of Natural Language Processing, a field of study of Artificial Intelligence, which may involve Machine Learning or Deep Learning techniques, allows the transformation of human language into a representation adapted to computation systems.

This thesis focus on the design of an approach that allows to consult and present information stored in data warehouses, through usage of natural language. As result, a prototype has been developed by putting into practise the conceptualized approach. Thus, the main goal is to adapt and use the suggested approach in the development of a natural language module to interact with the {\productname}, thereby improving the system's usability.

\end{abstractotherlanguage}

%----------------------------------------------------------------------------------------
%	ACKNOWLEDGEMENTS (optional)
%----------------------------------------------------------------------------------------
\begin{acknowledgements}

O trabalho desempenhado ao longo de quase um ano não seria possível sem a presença de várias pessoas, as quais marcam a minha vida todos os dias, e das quais tenho apoio incondicional.

Quero agradecer aos meus pais, pela educação, pelo apoio, pelo incentivo e principalmente, pelos valores que me foram incutidos, que fizeram de mim a pessoa que sou hoje. Alguns anos conturbados passaram e hoje tudo faz sentido -- \textit{I'm in hell without you, cannot cope without you two, shocked at the world that I see}. Obrigado.

Agradeço aos meus amigos, por aturarem as minhas longas \inquotes{palestras}, por me ouvirem, na alegria e na tristeza, e por incentivarem o meu sucesso (por ordem alfabética): Diogo, Egídio, Francisco, Joel, Sérgio, Wilson. Grande abraço.

Para o Diogo, obrigado por seres o \textit{brother from another mother}. Ligeiramente picuinhas, eu sei, mas não precisamos de palavras.

Deixo o meu agradecimento ao Instituto Superior de Engenharia do Porto, em particular aos professores que acompanharam o meu percurso e que foram uma inspiração. 

Agradeço à Critical Manufacturing, em especial ao Engenheiro Ricardo Magalhães, pela incentivo dado neste projeto, pelas ideias e pelos desafios que se demonstraram difíceis, mas que com diálogo foram possíveis alcançar. 

Ainda no contexto Critical Manufacturing, quero também deixar o meu agradecimento a todas as pessoas com quem tive o prazer de trabalhar. Quero deixar um agradecimento especial ao Rui Santos, por ter sido o meu mentor na jornada CMF e por me ter \inquotes{dado asas}. Obrigado pelas longas conversas musicais, técnicas e dos momentos de riso. Com certeza que nos vamos cruzar, especialmente em muitos concertos! \textit{Rock On}!

Ao meu orientador, Doutor Paulo Gandra de Sousa, que confiou em mim e no desempenho deste trabalho, que se demonstrou preocupado quando não dava notícias (peço desculpa, professor) e me ajudou a traçar o caminho seguido neste trabalho.

E, como sou uma pessoa que deixa o melhor para o fim, agradeço à Patrícia, minha amiga, confidente, apoiante número um, psicóloga e ouvinte nas horas vagas. Sempre acreditaste em mim, e continuas a fazê-lo, e não há palavras suficientes para descrever a minha gratidão -- \textit{Love of my Life}.

A todos, o meu mais profundo obrigado!
\end{acknowledgements}

%----------------------------------------------------------------------------------------
%	LIST OF CONTENTS/FIGURES/TABLES PAGES
%----------------------------------------------------------------------------------------

\tableofcontents % Prints the main table of contents

\listoffigures % Prints the list of figures

\listoftables % Prints the list of tables

\iflanguage{portuguese}{
\renewcommand{\listalgorithmname}{Lista de Algor\'itmos}
}
\listofalgorithms % Prints the list of algorithms
\addchaptertocentry{\listalgorithmname}


\renewcommand{\lstlistlistingname}{List of Source Code}
\iflanguage{portuguese}{
\renewcommand{\lstlistlistingname}{Lista de C\'odigo}
}
\lstlistoflistings % Prints the list of listings (programming language source code)
\addchaptertocentry{\lstlistlistingname}


%----------------------------------------------------------------------------------------
%	ABBREVIATIONS
%----------------------------------------------------------------------------------------
%\begin{abbreviations}{ll} % Include a list of abbreviations (a table of two columns)
%%\textbf{LAH} & \textbf{L}ist \textbf{A}bbreviations \textbf{H}ere\\
%%\textbf{WSF} & \textbf{W}hat (it) \textbf{S}tands \textbf{F}or\\
%\end{abbreviations}

%----------------------------------------------------------------------------------------
%	SYMBOLS
%----------------------------------------------------------------------------------------

\begin{symbols}{lll} % Include a list of Symbols (a three column table)

$a$ & distance & \si{\meter} \\
$P$ & power & \si{\watt} (\si{\joule\per\second}) \\
%Symbol & Name & Unit \\

\addlinespace % Gap to separate the Roman symbols from the Greek

$\omega$ & angular frequency & \si{\radian} \\

\end{symbols}



%----------------------------------------------------------------------------------------
%	ACRONYMS
%----------------------------------------------------------------------------------------

\newcommand{\listacronymname}{List of Acronyms}
\iflanguage{portuguese}{
\renewcommand{\listacronymname}{Lista de Acr\'onimos}
}

%Use GLS
\glsresetall
\printglossary[title=\listacronymname,type=\acronymtype,style=long]

%----------------------------------------------------------------------------------------
%	DONE
%----------------------------------------------------------------------------------------

\mainmatter % Begin numeric (1,2,3...) page numbering
\pagestyle{thesis} % Return the page headers back to the "thesis" style
