% we include the glossary here (frontmatter is included with \input, so this command is as if it was in main.tex)
% Acrónimos
\newacronym{iot}{IoT}{\textit{Internet of Things}}
\newacronym{ios}{IoS}{\textit{Internet of Services}}
\newacronym{ihc}{IHC}{Interação Humano-Computador}
\newacronym{ia}{IA}{Inteligência Artificial}
\newacronym{pln}{PLN}{Processamento de Linguagem Natural}
\newacronym{mes}{MES}{\textit{Manufacturing Execution System}}
\newacronym{sql}{SQL}{\textit{Structured Query Language}}
\newacronym{uml}{UML}{\textit{Unified Modeling Language}}
\newacronym{ceo}{CEO}{\textit{Chief Executive Officer}}
\newacronym{cto}{CTO}{\textit{Chief Technical Officer}}
\newacronym{erp}{ERP}{\textit{Enterprise Resource Planning}}
\newacronym{cps}{CPS}{\textit{Cyber-Physical System}}
\newacronym{ffe}{FFE}{\textit{Fuzzy Front End}}
\newacronym{npd}{NPD}{\textit{New Product Development}}
\newacronym{ncd}{NCD}{\textit{New Concept Development}}
\newacronym{slp}{SLP}{\textit{Single Layer Perceptron}}
\newacronym{ilnbd}{ILNBD}{Interfaces de Linguagem Natural para Bases de Dados}
\newacronym{atn}{ATN}{\textit{Augmented Transition Network}}
% for defining plural form
% \newacronym[shortplural=aa,longplural=letters a]{a}{A}{the a}

\frontmatter % Use roman page numbering style (i, ii, iii, iv...) for the pre-content pages

\pagestyle{plain} % Default to the plain heading style until the thesis style is called for the body content

%----------------------------------------------------------------------------------------
%	TITLE PAGE
%----------------------------------------------------------------------------------------

\maketitlepage

%----------------------------------------------------------------------------------------
%	DEDICATION  (optional)
%----------------------------------------------------------------------------------------
%\dedicatory{For/Dedicated to/To my\ldots}
\begin{dedicatory}
\tbd
\end{dedicatory}

%----------------------------------------------------------------------------------------
%	ABSTRACT PAGE
%----------------------------------------------------------------------------------------
\begin{abstract}

O paradigma de interação entre Homem e Máquina tem vindo a mudar nos últimos anos. Se, ao longo das últimas décadas, o ser humano tem vindo a interagir com o computador através da escrita (linha de comandos) ou das interfaces gráficas, como se desenvolverá esta interação quando a máquina for capaz de \inquotes{entender} a linguagem natural humana?

\end{abstract}

\begin{abstractotherlanguage}
The interaction paradigm between man and machine has been changing in the last years. Over the last decades, humans have been interacting with the computer through writing (command line) or graphical interfaces. Recently, emerges the interaction through natural language. How to enhance the communication between man and the system used on daily basis, by using natural language? The usage of Natural Language Processing, a field of study of Artificial Intelligence, which may involve Machine Learning or Deep Learning techniques, allows the transformation of human language into a representation adapted to computation systems.

This thesis focus on the design of an approach that allows to consult and present information stored in data warehouses, through usage of natural language. As result, a prototype has been developed by putting into practise the conceptualized approach. Thus, the main goal is to adapt and use the suggested approach in the development of a natural language module to interact with the {\productname}, thereby improving the system's usability.

\end{abstractotherlanguage}

%----------------------------------------------------------------------------------------
%	ACKNOWLEDGEMENTS (optional)
%----------------------------------------------------------------------------------------
\begin{acknowledgements}

\tbd

\end{acknowledgements}

%----------------------------------------------------------------------------------------
%	LIST OF CONTENTS/FIGURES/TABLES PAGES
%----------------------------------------------------------------------------------------

\tableofcontents % Prints the main table of contents

\listoffigures % Prints the list of figures

\listoftables % Prints the list of tables

\iflanguage{portuguese}{
\renewcommand{\listalgorithmname}{Lista de Algor\'itmos}
}
\listofalgorithms % Prints the list of algorithms
\addchaptertocentry{\listalgorithmname}


\renewcommand{\lstlistlistingname}{List of Source Code}
\iflanguage{portuguese}{
\renewcommand{\lstlistlistingname}{Lista de C\'odigo}
}
\lstlistoflistings % Prints the list of listings (programming language source code)
\addchaptertocentry{\lstlistlistingname}


%----------------------------------------------------------------------------------------
%	ABBREVIATIONS
%----------------------------------------------------------------------------------------
%\begin{abbreviations}{ll} % Include a list of abbreviations (a table of two columns)
%%\textbf{LAH} & \textbf{L}ist \textbf{A}bbreviations \textbf{H}ere\\
%%\textbf{WSF} & \textbf{W}hat (it) \textbf{S}tands \textbf{F}or\\
%\end{abbreviations}

%----------------------------------------------------------------------------------------
%	SYMBOLS
%----------------------------------------------------------------------------------------

\begin{symbols}{lll} % Include a list of Symbols (a three column table)

$a$ & distance & \si{\meter} \\
$P$ & power & \si{\watt} (\si{\joule\per\second}) \\
%Symbol & Name & Unit \\

\addlinespace % Gap to separate the Roman symbols from the Greek

$\omega$ & angular frequency & \si{\radian} \\

\end{symbols}



%----------------------------------------------------------------------------------------
%	ACRONYMS
%----------------------------------------------------------------------------------------

\newcommand{\listacronymname}{List of Acronyms}
\iflanguage{portuguese}{
\renewcommand{\listacronymname}{Lista de Acr\'onimos}
}

%Use GLS
\glsresetall
\printglossary[title=\listacronymname,type=\acronymtype,style=long]

%----------------------------------------------------------------------------------------
%	DONE
%----------------------------------------------------------------------------------------

\mainmatter % Begin numeric (1,2,3...) page numbering
\pagestyle{thesis} % Return the page headers back to the "thesis" style
