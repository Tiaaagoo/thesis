\chapter{Problema}
\label{chap:pre1}

Neste capítulo são apresentados o enunciado do problema (secção~\ref{sec:pre1_problem}), os objetivos do trabalho, o âmbito e pressupostos associados (secção~\ref{sec:pre1_objectives}) e, por fim, os critérios de sucesso a considerar para a avaliação final do trabalho (seccção~\ref{sec:pre1_success_criteria}).

\section{Enunciado do problema}
\label{sec:pre1_problem}

O conceito de \gls{MES}, um sistema que, além de gerir as operações dum determinado processo fabril, mantém dados relativos às diversas etapas inerentes ao processo em questão, está intrinsecamente relacionado com a Indústria 4.0. O Critical Manufacturing \gls{MES} é um destes sistemas. Contudo, a sua incapacidade parcial de adaptar-se às características dos utilizadores, torna-o difícil de usar, numa perspetiva de acesso a informação relevante para o processo e de apoio à decisão. Por outras palavras, se o utilizador pretende efetuar uma determinada pesquisa, necessita de conhecer os detalhes da ferramenta a usar, ao invés de simplesmente \inquotes{pedir} (através de texto ou voz) ao sistema que lhe devolva o resultado.

A \textbf{conceção de um módulo de linguagem natural para interface com o Critical Manufacturing \gls{MES}, permitindo a consulta e pesquisa de estados do processo de fabrico}, torna-se importante para o sistema, uma vez que possibilita o utilizador interagir com o sistema de uma forma simples, intuitiva, eficiente e natural, através de escrita.

\section{Objetivos, Âmbito e Pressupostos}
\label{sec:pre1_objectives}
De uma maneira geral, com este trabalho pretende-se desenvolver uma solução baseada em linguagem natural que promova a interação do utilizador com o sistema \gls{MES}. Com o intuito de solucionar o problema enunciado na secção~\ref{sec:pre1_problem}, definem-se os seguintes objetivos:

\begin{enumerate}
    \item
    \label{enum:pre1_objectives_1}
    {
        \textit{Contextualizar o problema numa perspetiva de negócio} -- análise detalhada do problema, as implicações que tem para negócio e para o produto \gls{MES}, descrevendo o valor intrínseco à solução (Capítulo~\ref{chap:Chapter2});
    }
    \item
    \label{enum:pre1_objectives_2}
    {
        \textit{Estudar soluções disponíveis no mercado e/ou bibliotecas de processamento de linguagem natural} -- obtenção de informação da área de conhecimento envolvida, de ferramentas semelhantes e de bibliotecas tipicamente usadas na implementação de tais módulos (Capítulo~\ref{chap:Chapter3});
    }
    \item
    \label{enum:pre1_objectives_3}
    {
        \textit{Definir a solução mais adequada, considerando as diversas opções apresentadas} -- comparação e avaliação das diversas opções identificadas, selecionando a(s) mais adequada(s), justificando essa(s) escolha(s) (Capítulo~\ref{chap:Chapter3});
    }
    \item
    \label{enum:pre1_objectives_4}
    {
        \textit{Especificação da arquitetura do módulo} -- que permita responder aos requisitos definidos e antecipar soluções para possíveis problemas (\textit{a definir});
    }
    \item
    \label{enum:pre1_objectives_5}
    {
        \textit{Descrever a semântica de domínio} -- identificação dos domínios a explorar e construção de uma base de conhecimento semântico para o módulo (\textit{a definir});
    }
    \item
    \label{enum:pre1_objectives_6}
    {
        \textit{Desenvolvimento de prova de conceito} -- implementação da solução de acordo com a arquitetura conceptualizada (\textit{a definir});
    }
    \item
    \label{enum:pre1_objectives_7}
    {
        \textit{Avaliar a qualidade da solução desenvolvida} -- com base nas metodologias de avaliação definidas, concluir acerca da qualidade da solução e do contributo do trabalho para a resolução do problema (\textit{a definir});
    }
    \item
    \label{enum:pre1_objectives_8}
    {
        \textit{Elaboração da tese escrita} -- como forma de transmitir o conhecimento alcançado durante a elaboração do trabalho.
    }
\end{enumerate}

Embora os objetivos estejam definidos, surge a necessidade de explicitar sucintamente o âmbito do trabalho e os pressupostos a ter em consideração. Por conseguinte, os seguintes assuntos não serão abordados:

\begin{itemize}
    \item
    {
        O enquadramento do problema com outros sistemas \gls{MES}. Apenas é contemplada a realidade do problema no contexto do Critical Manufacturing \gls{MES} (objetivo~\ref{enum:pre1_objectives_1});
    }
    \item
    {
        As soluções e bibliotecas de linguagem natural que não mostrem evidências de relevância para o problema, tendo em conta os critérios de preço, adesão da comunidade de desenvolvimento e complexidade (objetivo~\ref{enum:pre1_objectives_2});
    }
    \item 
    {
        A inclusão de diferentes domínios na solução desenvolvida (objetivos~\ref{enum:pre1_objectives_3}, \ref{enum:pre1_objectives_4} e \ref{enum:pre1_objectives_5});
    }
    \item 
    {
        
    }
\end{itemize}



\section{Critérios de Sucesso}
\label{sec:pre1_success_criteria}
\textit{A definir}

% ----------------------------------------
% CONTRIBUIÇÕES
% ----------------------------------------
\chapter{Contribuições Esperadas}
\label{chap:pre2}
\textit{A definir}

% ----------------------------------------
% METODOLOGIA E PLANO DE TRABALHO
% ----------------------------------------
\chapter{Método de trabalho}
\label{chap:pre3}

% ----------------------------------------
% CONTEXTO
% ---------------------------------------- 
\chapter{Contexto}
\label{chap:Chapter2}

\textit{A definir}

\section{A Empresa}
\label{sec:chap2_company}

A {\companyname} é uma empresa fundada em 2009, com sede e centro de engenharia na Maia (Porto, Portugal), subsidiárias em Dresden (Alemanha), Suzhou (China), Austin (Estados Unidos da América) e um escritório comercial em Taiwan. O objetivo é proporcionar à indústria uma solução de gestão e controlo de produção, procurando reduzir os custos de produção, flexibilizar para satisfazer a procura e capacitar a organização de uma maior agilidade, visibilidade e fiabilidade~\parencite{cmf_overview}. O compromisso da empresa~\parencite{cmf_overview} foca-se no desenvolvimento de~\inquotes{soluções de vanguarda, indo de encontro aos desafios mais importantes da indústria e disponibilizar à lista crescente de clientes satisfeitos, soluções de elevado valor acrescentado, no prazo e orçamento requerido}\footnote{Tradução livre do autor. No original~\inquotes{[...] solutions that address the most urgent industry challenges and provide our growing list of satisfied customers with the highest value solution, on-time and on-budget.}.}.

A estratégia da empresa está sintetizada na sua missão, visão e valores. Se a missão descreve a razão da empresa existir, ou seja, o seu propósito, já a visão retrata o que se aspira alcançar. Isto posto, a missão e visão são divulgados a seguir~\parencite{cmf_strategy}: 

\begin{itemize}
    \item 
    {
        \textit{Missão} -- \inquotes{trazer valor através da convergência de inteligência, operações e tecnologias de automação para a Indústria 4.0.}\footnote{Tradução livre do autor. No original \inquotes{We drive business value through the convergence of intelligence, operations, and automation technologies for Industry 4.0.}.}.
    }
    \item 
    {
        \textit{Visão} -- \inquotes{Tornar a Indústria 4.0 uma realidade para todos fabricantes.}\footnote{Tradução livre do autor. No original \inquotes{We will make Industry 4.0 a reality for all manufacturers.}.}.
    }
\end{itemize}

Relativamente aos valores, são estes que suportam a visão, moldam a cultura empresarial e são a essência da sua identidade. Como tal, de seguida apresentam-se os valores da Critical Manufacturing~\parencite{cmf_strategy}:

\begin{itemize}
    \item 
    {
        \textit{Inovação} -- \inquotes{Exceder as expectativas dos clientes através das soluções mais eficientes e de mais alto valor para indústria.}\footnote{Tradução livre do autor. No original \inquotes{We constantly exceed our customers’ expectations through the most efficient and high value-added manufacturing solutions.}.}.
    }
    \item
    {
        \textit{Agilidade} -- \inquotes{Adaptar as pessoas, processos e soluções de forma a responder à evolução do mundo da manufatura de alta tecnologia.}\footnote{Tradução livre do autor. No original \inquotes{We continuously adapt our people, processes and solutions to respond to the evolving world of high-tech manufacturing.}.}.
    }
    \item
    {
        \textit{Compromisso} -- \inquotes{Defender o sucesso contínuo dos clientes e da empresa.}\footnote{Tradução livre do autor. No original \inquotes{We champion the continued success of our customers and our company.}.}.
    }
\end{itemize}

Por conseguinte, com base na sua estratégia, o presente trabalho pretende demonstrar viabilidade e o valor da utilização da \gls{IA}, nomeadamente na área do \gls{PLN}, para a interação com o Critical Manufacturing MES e obter informação pertinente para o utilizador final. 

\section{O Produto}
\label{sec:chap2_product}

\textit{A definir}

\section{Análise de Valor}
\label{sec:chap2_valueanalysis}

\textit{A definir}

% ----------------------------------------
% ESTADO DA ARTE
% ---------------------------------------- 
% Chapter 3

\chapter{Estado da Arte}
\label{chap:Chapter3}

%----------------------------------------------------------------------------------------
%	
%----------------------------------------------------------------------------------------


% ----------------------------------------
% PROPOSTA DE SOLUÇÃO
% ---------------------------------------- 
% Se necessário....

% ----------------------------------------
% PLANO DE TRABALHO
% ---------------------------------------- 
\chapter{Plano de Trabalho}
\label{chap:pre7}
\textit{A definir}