\subsection{Contribuições}
\label{sec:chap01_contributions}
O trabalho a ser desenvolvido pretende providenciar uma solução para o problema descrito anteriormente (Secção \ref{sec:chap01_problem}). Não se aspira fornecer uma solução definitiva, espera-se sim, contribuir com conhecimento de carácter teórico e prático (\idest{um protótipo}), que possibilite a integração futura de uma nova funcionalidade num produto já existente, trazendo-lhe mais-valia funcional, destacando-o dos seus concorrentes. No decorrer deste trabalho serão abordados temas relativos a \gls{mes}, a \gls{ia}, especificamente \gls{pln}, e \textit{Business Intelligence}.

Resumidamente, as contribuições esperadas são a seguir enunciadas:

\begin{itemize}
    \item
    {
        Estado da arte no domínio de Processamento de Linguagem Natural, aplicadas à conversão de texto para \gls{sql} e sistemas análogos à solução a desenvolver;
    }
    \item 
    {
        Documentação dos requisitos de sistema, incluindo análise e desenho, constando os respetivos artefactos de \gls{uml};
    }
    \item 
    {
        Dicionário semântico do domínio a ser definido para o módulo;
    }
    \item
    {
        Especificação e desenvolvimento do protótipo, considerando a futura integração com o sistema {\productname};
    }
    \item
    {
        Conceção do mecanismo de \textit{feedback} permitindo a adaptação ao utilizador e consequentemente, a auto-aprendizagem do sistema.
    }
\end{itemize}

De uma forma geral, é realçada a contribuição para o avanço do conhecimento no domínio da \gls{ia}, mais especificamente na área da \gls{pln}, aplicada ao contexto dos sistemas \gls{mes} e que servirá como base para a integração duma solução deste tipo no {\productname}.