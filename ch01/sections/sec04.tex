\section{Avaliação e Experimentação}
\label{sec:chap01_solutionevaluation}
A avaliação do resultado final é imprescindível para a concluir acerca do sucesso do presente trabalho, permitindo perceber se a conjetura fundada a respeito da prova de conceito é aceite. Desse modo, formula-se a hipótese colocada para a abordagem escolhida, a respetiva metodologia de avaliação e experimentação e os critérios de sucesso a serem considerados.

\subsection{Formulação das Hipóteses}
\label{sec:chap01_hypothesis}
Para a resolução do problema da empresa explicitado em~\ref{sec:chap01_problem}, o qual foca a melhoria da interação do {\productname} com o utilizador, surgem as seguintes questões:

\begin{enumerate}
    \item
    {
        A integração de um módulo de linguagem natural pode, de facto, melhorar a usabilidade do produto e consequentemente, simplificar processo de apoio à decisão?
    }
    \item
    {
        De que forma se pode avaliar a adequabilidade das respostas da solução às necessidades básicas dos utilizadores?
    }
\end{enumerate}

Embora as perguntas anteriores sejam relevantes para a formulação de hipóteses para a solução final, e devem ser tidas em consideração na abordagem escolhida, não terão um peso significativo na avaliação do resultado deste trabalho. O foco desta tese, tal como descrito na Secção~\ref{sec:chap01_objectives}, é o desenvolvimento de um protótipo, em que a abordagem pode ser seguida para a implementação de uma solução definitiva no {\productname}. Assim, surge outra pergunta mais pertinente para esta fase e respetiva hipótese:

\begin{itemize}
    \item
    {  
        \textit{Questão} -- Qual a abordagem adequada para a tradução de linguagem natural numa linguagem capaz de extrair conhecimento de armazéns de dados?
    }
    \item
    {
        \textit{Hipótese} -- A abordagem escolhida permite a extração de conhecimento de armazéns de dados a partir de linguagem natural.
    }
\end{itemize}

A hipótese apresentada auxilia na definição da metodologia de avaliação a adotar na presente tese. A refutação ou aceitação da hipótese formulada permite concluir acerca do trabalho realizado, e da necessidade de reformulação ou adoção de novas hipóteses.

\subsection{Metodologia de Avaliação}
\label{sec:chap01_evaluationmethodologies}
Com o propósito de perceber se a abordagem tomada no protótipo desenvolvido é adequado para o {\productname} e para o utilizador final, e levando em consideração a hipótese formulada anteriormente, definem-se as seguintes estratégias para a metodologia de avaliação deste trabalho:

\begin{enumerate}
\label{enum:chap01_qualitystrategies}
    \item 
    {
        \textit{Garantir que a solução analisa e responde corretamente a um conjunto de perguntas pré-definidas} -- a solução deverá responder adequadamente a um conjunto limitado de perguntas:
        \begin{itemize}
            \item 
            {
                Quantas operações $O$ foram executadas por semana, durante o mês $M$?
            }
            \item
            {
                Qual o número de operações $O$ por produto e turno, durante o mês $M$?
            }
            \item
            {
                Qual a média de $X$ de operações $O$, no passo $P$ do processo, por turno, no mês $M$? 
            }
            \item
            {
                Qual o número de materiais cujo valor de $X$ é inferior a $Y$, para o passo $P$ do processo, agrupando por $G$?
            }
        \end{itemize}
        
        Nas questões apresentadas, as letras representam as variáveis inerentes ao domínio, que o utilizador conhece e que o sistema deve ser capaz de reconhecer.
    }
    \item
    {
        \textit{Usar as respostas devolvidas pelo protótipo para concluir acerca da sua exatidão} -- as respostas fornecidas pelo protótipo, face à resposta expectável, permitirão perceber se a abordagem seguida é adequada.
    }
\end{enumerate}

Ambas estratégias possibilitam perceber a adequabilidade da abordagem para solução a ser integrada no produto e para o utilizador, quer numa perspetiva de facilidade de utilização, quer na exatidão da resposta dada.

\subsection{Critérios de Sucesso}
De seguida, enumeram-se os critérios de sucesso para o trabalho:

\begin{enumerate}
    \item 
    {
        \textit{A hipótese apresentada anteriormente é aceite} -- a abordagem escolhida apresenta resultados satisfatórios face à metodologia de avaliação definida para este trabalho;
    }
    \item
    {
        \textit{A abordagem apresentada é extensível e de fácil integração no {\productname}} -- garante-se que a arquitetura especificada considerou a existência de diversos domínios, facilidade e capacidade de integração com o produto;
    }
    \item
    {
        \textit{Prova de conceito dá resposta correta às questões que lhe são colocadas} -- que implica responder corretamente às questões listadas em~\ref{enum:chap01_qualitystrategies}, garantindo que a resposta fornecida é semelhante ou igual à resposta que seria esperada;
    }
    \item
    {
        \textit{A abordagem é adotada ou refinada de forma a que possa ser usada na solução final} -- a abordagem revela-se efetiva na resolução do problema, e com o levantamento de possíveis melhorias, pode ser implementada no {\productname}.
    }
    % \item 
    % {
    %     \textit{Tese escrita} -- na qual se abordam o problema, o contexto no qual se insere e o valor que traz ao produto final. Deve conter o estado da arte, apresentando a revisão da literatura existente, focando nas soluções semelhantes e/ou ferramentas relevantes que perspetivam estratégias de solução para o problema. Por fim, descreve-se a solução proposta, contemplando cada uma das fases inerentes ao seu desenvolvimento (visão, análise, desenho e implementação) e faz-se a conclusão acerca do trabalho (todos os objetivos descritos em~\ref{sec:chap01_objectives});
    % }
\end{enumerate}