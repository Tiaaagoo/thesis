\section{Avaliação da Solução}
\label{sec:chap01_solutionevaluation}

A avaliação do resultado final é imprescindível para concluir acerca do sucesso do trabalho, permitindo perceber se a conjetura fundada a respeito da solução é aceite. Desse modo, especificam-se a metodologia de avaliação para solução desenvolvida e os critérios de sucesso a serem considerados.

\subsection{Metodologia de Avaliação}
Com o propósito de perceber se a solução desenvolvida é adequada para o {\productname} e para o utilizador final, definem-se as seguintes estratégias para a metodologia de avaliação deste trabalho:

\begin{enumerate}
\label{enum:chap01_qualitystrategies}
    \item 
    {
        \textit{Garantir que a solução analisa e responde corretamente a um conjunto de perguntas pré-definidas} -- a solução deverá responder adequadamente a um conjunto limitado de perguntas:
        \begin{itemize}
            \item 
            {
                Quantas operações foram executadas por semana, durante o mês $M$?
            }
            \item
            {
                Qual o número de operações $O$ por produto e turno, durante o mês $M$?
            }
            \item
            {
                Qual a média de $X$ de operações $O$, no passo $P$ do processo, por turno, no mês $M$? 
            }
            \item
            {
                Qual o número de materiais cujo valor de $X$ é inferior a $Y$, para o passo $P$ do processo, agrupando por $G$?
            }
        \end{itemize}
        
        Nas questões apresentadas, as letras representam as variáveis próprias do domínio, que o utilizador conhece e que o sistema deve ser capaz de reconhecer.
    }
    \item
    {
        \textit{Usar as respostas recolhidas pelo mecanismo de feedback para conclusão acerca da usabilidade da solução} -- as respostas fornecidas pelo utilizador permitirão perceber se a solução contempla as suas necessidades e quais as medidas ou requisitos podem ser futuramente implementados, com a perspetiva de melhorar a qualidade do sistema.
    }
\end{enumerate}

Ambas estratégias possibilitam perceber a adequabilidade da solução para o produto e para o utilizador, quer numa perspetiva de facilidade de utilização, quer na exatidão da resposta dada.

\subsection{Critérios de Sucesso}

De seguida, enumeram-se os critérios de sucesso para o trabalho, associados aos respetivos objetivos:

\begin{enumerate}
    \item
    {
        \textit{A solução apresentada é extensível a outros domínios e facilmente integrada no {\productname}} -- garante-se assim que a arquitetura especificada considerou diversos domínios, facilidade e capacidade de integração com o produto, ainda mesmo sendo um protótipo (objetivos~\ref{enum:chap01_objectives_3}, \ref{enum:chap01_objectives_4} e \ref{enum:chap01_objectives_6});
    }
    \item
    {
        \textit{Prova de conceito dá resposta correta ao domínio definido} -- que implica responder corretamente às questões listadas em \ref{enum:chap01_qualitystrategies}, garantindo que semântica foi bem definida e que o requisito de qualidade está cumprido (objetivos~\ref{enum:chap01_objectives_5}, \ref{enum:chap01_objectives_6}, \ref{enum:chap01_objectives_7} e \ref{enum:chap01_objectives_8}).
    }
    \item 
    {
        \textit{Tese escrita} -- na qual se abordam o problema, o contexto no qual se insere e o valor que traz ao produto final. Deve conter o estado da arte, apresentando a revisão da literatura existente, focando nas soluções semelhantes e/ou ferramentas relevantes que perspetivam estratégias de solução para o problema. Por fim, descreve-se a solução proposta, contemplando cada uma das fases inerentes ao seu desenvolvimento (visão, análise, desenho e implementação) e faz-se a conclusão acerca do trabalho (todos os objetivos descritos em~\ref{sec:chap01_objectives});
    }
\end{enumerate}