\section{Problema}
\label{sec:chap01_problem}
O conceito de \gls{mes}, um sistema que, além de gerir as operações dum determinado processo fabril, mantém dados relativos às diversas etapas inerentes ao processo em questão, está intrinsecamente relacionado com a Indústria 4.0, uma iniciativa que se destina a criar fábricas inteligentes, usando tecnologias como os \glspl{cps}, a \gls{iot} e \textit{Cloud Computing}~\parencite{intelligent_manufacturing_context_industry40_review}. O {\productname} é um destes sistemas. Contudo, a capacidade de adaptação às características dos utilizador é um requisito complexo, que nem sempre é passível de ser cumprido. Isso pode tornar o produto difícil de usar, numa perspetiva de acesso a informação relevante para o processo e de apoio à decisão. Por outras palavras, se o utilizador pretende efetuar uma determinada pesquisa, necessita de conhecer os detalhes da ferramenta a usar, ao invés de simplesmente \inquotes{pedir} (através de texto ou voz) ao sistema que lhe devolva o resultado.

\textbf{A conceção de um módulo de linguagem natural para interface com o {\productname}, permitindo a consulta e pesquisa de estados do processo de fabrico}, torna-se importante para o sistema, uma vez que possibilita o utilizador interagir com o sistema de uma forma simples, intuitiva, eficiente e natural, através de escrita.