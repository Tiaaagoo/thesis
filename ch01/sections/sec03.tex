\section{Âmbito}
\label{sec:chap01_scope}
Embora os objetivos estejam definidos, surge a necessidade de explicitar sucintamente o âmbito deste trabalho, bem como os pressupostos a ter em consideração. Por conseguinte, os seguintes assuntos serão abordados:

\begin{itemize}
    \item
    {
        A contextualização do problema da empresa com a prova de conceito a ser desenvolvida, o seu enquadramento com a Indústria 4.0 e utilidade para o cliente final; 
    }
    \item 
    {
        Os conceitos teóricos e adversidades inerentes ao problema, ainda que explorados de uma forma genérica, evitando abordar pormenores ou especificidades do tema;
    }
    \item
    {
        A apresentação e explicação dos exemplos de resolução de problemas semelhantes por parte de terceiros, fazendo um levantamento das características relevantes para este projeto;
    }
    \item
    {
        As ferramentas disponíveis e relevantes para este contexto, passíveis de ser aplicadas nesta prova de conceito;
    }
    \item
    {
        O método científico e processo de engenharia adotado na busca duma abordagem para resolução do problema em questão.
    }
\end{itemize}

Por outro lado, alguns tópicos são demasiado amplos para serem explorados, ou simplesmente não se enquadram nos objetivos desta tese, pelo que não serão abordados:

\begin{itemize}
    \item
    {
        O enquadramento do problema com outros \glspl{mes}. Apenas é contemplada a realidade do problema no contexto do {\productname};
    }
    \item
    {
        As soluções e ferramentas de linguagem natural que não mostrem evidências de relevância para o problema, tendo em conta os critérios de preço, adesão da comunidade de desenvolvimento e respetiva complexidade;
    }
    \item 
    {
        A inclusão de diferentes domínios na solução desenvolvida.
    }
\end{itemize}

O termo \inquotes{Domínio} é empregue ao longo do texto para denotar um conjunto de características que descrevem uma família de conceitos comuns a um determinado processo. Por exemplo, duas empresas que produzem equipamentos médicos, apesar de poderem ter processos de fabrico diferentes, abordam o mesmo domínio.

Neste trabalho assume-se que a solução a desenvolver, embora desejada para integrar em diferentes processos de manufatura, é uma prova de conceito, pelo que deverá considerar um domínio específico e consequentemente, lidar com a semântica específica desse domínio.