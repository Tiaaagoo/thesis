\section{Âmbito}
\label{sec:chap01_scope}
Embora os objetivos estejam definidos, surge a necessidade de explicitar sucintamente o âmbito do trabalho, bem como os pressupostos a ter em consideração. Por conseguinte, os seguintes assuntos não serão abordados:

\begin{itemize}
    \item
    {
        O enquadramento do problema com outros \glspl{mes}. Apenas é contemplada a realidade do problema no contexto do {\productname} (objetivo~\ref{enum:chap01_objectives_1});
    }
    \item
    {
        As soluções e bibliotecas de linguagem natural que não mostrem evidências de relevância para o problema, tendo em conta os critérios de preço, adesão da comunidade de desenvolvimento e respetiva complexidade (objetivo~\ref{enum:chap01_objectives_2});
    }
    \item 
    {
        A inclusão de diferentes domínios na solução desenvolvida (objetivos~\ref{enum:chap01_objectives_3}, \ref{enum:chap01_objectives_4} e \ref{enum:chap01_objectives_5});
    }
\end{itemize}

O termo \inquotes{Domínio} é empregue ao longo do texto para denotar um conjunto de características que descrevem uma família de conceitos comuns a um determinado processo. Por exemplo, duas empresas que produzem equipamentos médicos, apesar de poderem ter processos de fabrico diferentes, abordam o mesmo domínio.

Neste trabalho assume-se que a solução a desenvolver, embora pensada para integrar em diferentes processos de manufatura, é uma prova de conceito, pelo que deverá considerar um processo fabril (a ser definido) e consequentemente, lidar com a semântica específica desse domínio.