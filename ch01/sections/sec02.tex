\section{Objetivos}
\label{sec:chap01_objectives}
De uma maneira geral, com este trabalho pretende-se desenvolver um protótipo, cuja abordagem pode ser usada para solução final baseada em linguagem natural, que promova a interação do utilizador com o sistema \gls{mes}. Com o intuito de solucionar o problema enunciado na Secção~\ref{sec:chap01_problem}, definem-se os seguintes objetivos:

\begin{enumerate}
    \item
    {
        \textit{Contextualizar o problema numa perspetiva de negócio} -- análise detalhada do problema, as implicações que tem para negócio e para o produto \gls{mes}, descrevendo o valor intrínseco à solução (Capítulo~\ref{chap:Chapter2});
    }
    \item
    {
        \textit{Estudar soluções disponíveis no mercado e/ou ferramentas de processamento de linguagem natural} -- obtenção de informação da área de conhecimento envolvida, de soluções semelhantes e de ferramentas tipicamente usadas na implementação de tais módulos (Capítulo~\ref{chap:Chapter3});
    }
    \item
    {
        \textit{Definir a abordagem mais adequada, considerando as diversas opções apresentadas} -- comparação e avaliação das diversas opções identificadas (Capítulo~\ref{chap:Chapter3});
    }
    \item
    {
        \textit{Especificação da arquitetura do módulo} -- que permita responder aos requisitos definidos e antecipar soluções para possíveis problemas;
    }
    \item
    {
        \textit{Descrever a semântica de domínio} -- identificação dos domínios a explorar e construção de uma base de conhecimento semântico para o módulo;
    }
    \item
    {
        \textit{Desenvolvimento de prova de conceito} -- implementação do protótipo de acordo com a arquitetura conceptualizada;
    }
    \item
    {
        \textit{Prover o protótipo de um mecanismo de feedback para auto-aprendizagem} -- o que permitirá ao módulo adaptar-se às necessidades do utilizador, melhorando a qualidade das suas respostas. Numa fase inicial, este mecanismo pode não constar no protótipo, ou pode consistir em simplesmente questionar o utilizador sobre a exatidão da resposta apresentada;
    }
    \item
    {
        \textit{Avaliar a qualidade da solução desenvolvida} -- com base na hipótese formulada em~\ref{sec:chap01_hypothesis} e na estratégia de avaliação definida em~\ref{enum:chap01_qualitystrategies}, concluir acerca da qualidade da abordagem seguida e do contributo do trabalho para a resolução do problema.
    }
\end{enumerate}