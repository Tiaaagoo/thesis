\section{Objetivos}
\label{sec:chap01_objectives}
De uma maneira geral, com este trabalho pretende-se desenvolver uma solução baseada em linguagem natural que promova a interação do utilizador com o sistema \gls{mes}. Com o intuito de solucionar o problema enunciado na Secção~\ref{sec:chap01_problem}, definem-se os seguintes objetivos:

\begin{enumerate}
    \item
    \label{enum:chap01_objectives_1}
    {
        \textit{Contextualizar o problema numa perspetiva de negócio} -- análise detalhada do problema, as implicações que tem para negócio e para o produto \gls{mes}, descrevendo o valor intrínseco à solução (Capítulo~\ref{chap:Chapter2});
    }
    \item
    \label{enum:chap01_objectives_2}
    {
        \textit{Estudar soluções disponíveis no mercado e/ou bibliotecas de processamento de linguagem natural} -- obtenção de informação da área de conhecimento envolvida, de ferramentas semelhantes e de bibliotecas tipicamente usadas na implementação de tais módulos (Capítulo~\ref{chap:Chapter3});
    }
    \item
    \label{enum:chap01_objectives_3}
    {
        \textit{Definir a solução mais adequada, considerando as diversas opções apresentadas} -- comparação e avaliação das diversas opções identificadas (Capítulo~\ref{chap:Chapter3});
    }
    \item
    \label{enum:chap01_objectives_4}
    {
        \textit{Especificação da arquitetura do módulo} -- que permita responder aos requisitos definidos e antecipar soluções para possíveis problemas;
    }
    \item
    \label{enum:chap01_objectives_5}
    {
        \textit{Descrever a semântica de domínio} -- identificação dos domínios a explorar e construção de uma base de conhecimento semântico (\idest{um dicionário}) para o módulo;
    }
    \item
    \label{enum:chap01_objectives_6}
    {
        \textit{Desenvolvimento de prova de conceito} -- implementação da solução de acordo com a arquitetura conceptualizada;
    }
    \item
    \label{enum:chap01_objectives_7}
    {
        \textit{Prover a solução de um mecanismo de feedback para auto-aprendizagem} -- o que permitirá ao módulo adaptar-se às necessidades do utilizador, melhorando a qualidade das suas respostas. Numa fase inicial, este mecanismo consiste simplesmente em questionar o utilizador sobre a exatidão da resposta apresentada;
    }
    
    \item
    \label{enum:chap01_objectives_8}
    {
        \textit{Avaliar a qualidade da solução desenvolvida} -- com base nas estratégias de avaliação definidas em~\ref{enum:chap01_qualitystrategies}, concluir acerca da qualidade da solução e do contributo do trabalho para a resolução do problema;
    }
    \item
    \label{enum:chap01_objectives_9}
    {
        \textit{Elaboração da tese escrita} -- como forma de transmitir o conhecimento alcançado durante a elaboração do trabalho.
    }
\end{enumerate}