\chapter{Introdução}
\label{chap:Chapter1}
Num mercado crescentemente competitivo e exigente, a necessidade de inovar, de obter vantagem competitiva e simultaneamente, tornar os processos industriais simples e altamente eficazes, recorrendo às tecnologias mais atuais, abrem caminho a uma nova mudança. O fenómeno da Indústria 4.0 surge como a nova (quarta) revolução industrial, baseando-se nas mais recentes tecnologias, que incluem os sistemas ciber-físicos, a \gls{iot} e a \gls{ios}, as quais se baseiam na comunicação através da Internet, permitindo uma interação contínua e partilha de informação entre humanos, entre máquinas e entre o ser humano e máquina~\parencite{complex_view_industry40}. 

A Indústria 4.0 assenta numa variedade de conceitos fundamentais, de diferentes áreas de conhecimento, nomeadamente a noção de \textit{Smart Factory\footnote{Fábrica Inteligente, equipada com sensores, atuadores e sistemas autónomos, permitindo assim um controlo autónomo de processo.}}, a capacidade de auto-organização através da descentralização dos sistemas produtivos, e a interação entre o mundo físico e o digital~\parencite[Fundamental Concepts, p.240]{industry40}. No entanto, é a capacidade de adaptação à necessidade humana, principalmente a \gls{ihc}, que se explora no presente trabalho.

Segundo~\textcite[p.1]{natural_language_translation_intersaction_ai_hci}~\inquotes{as áreas de \gls{ia} e \glsfirst{ihc} estão, cada vez mais, a influenciar-se mutuamente. Alguns sistemas amplamente usados como o Google Translate, Facebook Graph Search e RelateIQ, escondem a complexidade de sistemas de larga escala de \gls{ia} através de interfaces intuitivas.}\footnote{Tradução livre do autor. No original~\inquotes{The fields of artificial intelligence (AI) and human-computer interaction (HCI) are influencing each other like never before. Widely used systems such as Google Translate, Facebook Graph Search, and RelateIQ hide the complexity of large-scale AI systems behind intuitive interfaces.}.}. Apesar de terem propósitos diferentes, ambas as áreas se complementam, na medida em que se focam na relação entre ser humano e máquina. Se a \gls{ia} tem como objetivo emular o intelecto humano, já a \gls{ihc} foca-se em abordagens empíricas de usabilidade e fatores humanos, que influenciam a forma como os utilizadores interagem com o computador~\parencite{natural_language_translation_intersaction_ai_hci}. 

A capacidade dum sistema interpretar a linguagem dos seres humanos e apresentar a informação de uma forma adequada, principalmente no contexto da Indústria 4.0, destaca-se como um fator impulsionador da adaptabilidade do mundo digital à necessidade humana. Nesse sentido, a área de \gls{pln}, a qual se debruça na capacidade dos computadores \inquotes{entenderem} a linguagem humana~\parencite[p.1]{applied_natural_language_processing_with_python}, permite construir ferramentas capazes de definir ações, extrair conhecimento dum sistema e apresentá-lo num formato adequado, a partir de conteúdo textual especificado pelo utilizador, de acordo com a sua própria linguagem. Portanto, este trabalho foca-se no estudo do \gls{pln} aplicado às \glspl{iln}, e na definição de uma abordagem e desenvolvimento dum protótipo que permita interação com um sistema, através de linguagem natural, possibilitando consulta de informação acerca de estados de processos de fabrico.

O capítulo atual faz um pequeno enquadramento do trabalho, apresenta o problema, os objetivos e âmbito da tese, os critérios de sucesso definidos, as contribuições do trabalho para a área e o plano de trabalho e respetivo método a ser seguido. 

Posteriormente, no Capítulo~\ref{chap:Chapter2}, contextualiza-se o problema, fazendo uma descrição da empresa e negócio, e aborda-se o propósito e relevância da resolução do mesmo. 

O Capítulo~\ref{chap:Chapter3} descreve o estado da arte, introduzindo os conceitos importantes de explorar, os trabalhos de referência e outros trabalhos relevantes na área, bem com ferramentas tipicamente usadas no enquadramento do \gls{pln}, perspetivando estratégias e possíveis abordagens.

No Capítulo~\ref{chap:Chapter4} expõe-se o processo de experimentação prática e o modelo proposto como solução, revelando a visão, os requisitos identificados e arquitetura prevista. 

Seguidamente, os Capítulos~\ref{chap:Chapter5} e \ref{chap:Chapter6} abordam, respetivamente, o processo de desenvolvimento do protótipo com base no modelo proposto e a validação deste, tendo em conta os critérios de sucesso estipulados. 

Por fim, o Capítulo~\ref{chap:Chapter7} anuncia as conclusões do trabalho realizado, inferindo sobre os objetivos atingidos e a resposta ao problema, evidenciando-se limitações da solução e problemas por explorar.

%%%%%%%%%%%%%%%%%%%%%%%%%%%%%%%%%
%           SECTION
%%%%%%%%%%%%%%%%%%%%%%%%%%%%%%%%%
\section{Problema}
\label{sec:chap01_problem}
O conceito de \gls{mes}, um sistema que, além de gerir as operações dum determinado processo fabril, mantém dados relativos às diversas etapas inerentes ao processo em questão, está intrinsecamente relacionado com a Indústria 4.0, uma iniciativa que se destina a criar fábricas inteligentes, usando tecnologias como os \glspl{cps}, a \gls{iot} e \textit{Cloud Computing}~\parencite{intelligent_manufacturing_context_industry40_review}. O {\productname} é um destes sistemas. Contudo, a capacidade de adaptação às características dos utilizador é um requisito complexo, que nem sempre é passível de ser cumprido. Isso pode dificultar o utilização do produto, numa perspetiva de acesso a informação relevante para o processo e de apoio à decisão~\parencite{intelligent_manufacturing_context_industry40_review}. Por outras palavras, se o utilizador pretende efetuar uma determinada pesquisa, necessita de conhecer os detalhes da ferramenta a usar, ao invés de simplesmente \inquotes{pedir} ao sistema que lhe devolva o resultado. Posto isto, levanta-se a seguinte questão: \textit{Será que o uso de linguagem natural na interação com o produto pode melhorar a sua usabilidade e simplificar processo de apoio à decisão?}. Partindo desta questão, 
procura-se a conceção de um módulo de linguagem natural para interface com o {\productname}.

%pressupõe-se a interação com o {\productname} através de texto. 

% criação de um módulo de linguagem natural para interface com o produto.

% o uso de linguagem natural para interface com {\productname}, possibilitando a consulta de estados do processo de fabrico, através de texto.

% Dada a questão, depreende-se que o uso de linguagem natural poderá possibilitar o utilizador de interagir com o sistema de um forma simples, eficiente e natural.

%\textbf{A conceção de um módulo de linguagem natural para interface com o {\productname}, permitindo a consulta e pesquisa de estados do processo de fabrico}, torna-se importante para o sistema, uma vez que possibilita o utilizador interagir com o sistema de uma forma simples, intuitiva, eficiente e natural, através de escrita.

%%%%%%%%%%%%%%%%%%%%%%%%%%%%%%%%%
%           SECTION
%%%%%%%%%%%%%%%%%%%%%%%%%%%%%%%%%
\section{Objetivos}
\label{sec:chap01_objectives}
Com este trabalho pretende-se a conceção de um módulo de linguagem natural. Para tal, também se objetiva o desenvolvimento de um protótipo, baseado no modelo concebido, capaz de de compreender três níveis de linguagem natural (aplicada ao contexto do produto): adimensional (sem entidades de domínio associadas), unidimensional (lida com uma entidade de domínio) e multidimensional (lida com múltiplas entidades de domínio).

Para os objetivos mencionados esperam-se as seguintes tarefas:
\begin{enumerate}
    \item
    {
        \textit{Contextualizar o problema numa perspetiva de negócio} -- análise detalhada do problema, as implicações que tem para negócio, para o produto \gls{mes} e para os seus utilizadores, descrevendo a relevância do problema e valor intrínseco à sua resolução (Capítulo~\ref{chap:Chapter1} e \ref{chap:Chapter2});
    }
    \item
    {
        \textit{Estudar soluções disponíveis no mercado e/ou ferramentas de processamento de linguagem natural} -- obtenção de informação da área de conhecimento envolvida, de soluções semelhantes, trabalhos de referência e de ferramentas tipicamente usadas na implementação de soluções análogas (Capítulo~\ref{chap:Chapter3});
    }
    \item
    {
        \textit{Definir abordagem considerando as diversas opções apresentadas} -- comparação e avaliação das diversas opções identificadas, concluindo acerca do caminho a seguir (Capítulo~\ref{chap:Chapter3} e \ref{chap:Chapter4});
    }
    \item
    {
        \textit{Desenvolvimento de prova de conceito} -- implementação do protótipo de acordo com o modelo concetualizado (Capítulo~\ref{chap:Chapter4} e \ref{chap:Chapter5});
    }
    \item
    {
        \textit{Validação da prova de conceito} -- concluir acerca do protótipo desenvolvido, face aos critérios de sucesso estipulados (Capítulo~\ref{chap:Chapter6}).
    }
\end{enumerate}

% Com o intuito de dar resposta a este objetivo definem-se dois objetivos mais específicos:
% \begin{enumerate}
%     \item
%     {
%         \textit{Desenvolvimento de prova de conceito capaz de compreender a linguagem natural} -- implementação de um protótipo, que além de ter a capacidade de compreensão da linguagem natural, possa ter o seu modelo arquitetural a ser usado no contexto dum futuro módulo de linguagem natural para o {\productname};
%     }
%     \item
%     {
%         \textit{Garantir que a prova de conceito desenvolvida analisa e responde corretamente a um conjunto de perguntas pré-definidas, próprias do problema em questão} -- a solução deverá responder adequadamente a um conjunto limitado de perguntas associadas à consulta de estados do processo de fabrico.
%     }
% \end{enumerate}

%%%%%%%%%%%%%%%%%%%%%%%%%%%%%%%%%
%           SECTION
%%%%%%%%%%%%%%%%%%%%%%%%%%%%%%%%%
\section{Âmbito e Pressupostos}
\label{sec:chap01_scope}
Embora os objetivos estejam definidos, surge a necessidade de explicitar sucintamente o âmbito deste trabalho, bem como os pressupostos a ter em consideração. Por conseguinte, os seguintes assuntos serão abordados:

\begin{itemize}
    \item
    {
        A contextualização do problema da empresa com a prova de conceito a ser desenvolvida, o seu enquadramento com a Indústria 4.0 e utilidade para o cliente final; 
    }
    \item 
    {
        Os conceitos teóricos e adversidades inerentes ao problema, ainda que explorados de uma forma genérica, evitando abordar pormenores ou especificidades do tema;
    }
    \item
    {
        A apresentação e explicação dos exemplos de resolução de problemas semelhantes por parte de terceiros -- trabalhos de referência --, fazendo um levantamento das características relevantes para este projeto;
    }
    \item
    {
        As ferramentas disponíveis e relevantes para este contexto, passíveis de ser aplicadas na solução final;
    }
    \item
    {
        O método científico e processo de engenharia adotado na busca duma abordagem para resolução do problema em questão.
    }
\end{itemize}

Por outro lado, alguns tópicos são demasiado amplos para serem explorados, ou simplesmente não se enquadram nos objetivos desta tese, pelo que não serão abordados:

\begin{itemize}
    \item
    {
        O enquadramento do problema com outros \glspl{mes}. Apenas é contemplada a realidade do problema no contexto do {\productname};
    }
    \item
    {
        As ferramentas para linguagem natural que não mostrem evidências de relevância para o problema;
    }
    \item 
    {
        A inclusão de diferentes domínios no protótipo desenvolvido.
    }
\end{itemize}

Quanto aos pressupostos deste trabalhos, enunciam-se a seguir.

\begin{itemize}
    \item 
    {
        O termo \inquotes{domínio} é empregue ao longo do texto para denotar um conjunto de características que descrevem uma família de conceitos comuns a um determinado processo. Por exemplo, duas empresas que produzem equipamentos médicos, apesar de poderem ter processos de fabrico diferentes, abordam o mesmo domínio;
    }
    \item
    {
        Assume-se o inglês como a língua a aplicar no protótipo, pois é a língua principal do {\productname}. Nesse sentido, o inglês abrange todo o tipo de interação com o protótipo, incluindo as questões a serem avaliadas;
    }
    \item
    {
        Também se assume o inglês para os artefactos produzidos, pelo motivo evidenciado no ponto anterior e pelo facto de ser a linguagem mais usada pela comunidade de desenvolvimento.
    }
\end{itemize}

%por isso não se espera que todos os casos de uso ou detalhes esperados para a solução final (\exempligratia{uso de \textit{feedback} de utilizador}) sejam contemplados no protótipo.

%%%%%%%%%%%%%%%%%%%%%%%%%%%%%%%%%
%           SECTION
%%%%%%%%%%%%%%%%%%%%%%%%%%%%%%%%%
\section{Critérios de Sucesso}
\label{sec:chap01_solutionevaluation}
A avaliação do resultado final é imprescindível para a concluir acerca do sucesso do trabalho, permitindo perceber se a conjetura fundada a respeito da prova de conceito é aceite. Desse modo, apresentam-se os critérios de sucesso a serem considerados:
%
\begin{enumerate}
    % \item 
    % {
    %     \textit{A hipótese apresentada anteriormente é aceite} -- a abordagem (modelo) escolhida apresenta resultados satisfatórios face à metodologia de avaliação definida para este trabalho;
    % }
    % \item
    % {
    %     \textit{O modelo apresentado é extensível e de fácil integração no {\productname}} -- garante-se que a arquitetura especificada considerou a existência de diversos domínios, facilidade e capacidade de integração com o produto;
    % }
    \item
    {
        \textit{O protótipo desenvolvido é capaz de compreender a linguagem natural} -- tem a capacidade de identificar os metadados associados à pergunta colocada, ou seja, de identificar linguagem natural adimensional, unidimensional e multidimensional;
    }
    \item
    {
        \textit{O protótipo dá resposta às questões que lhe são colocadas} -- que implica responder às questões-chave, esperando que resposta fornecida seja semelhante à resposta que seria esperada.
    }
\end{enumerate}

Relativamente às questões-chave abordadas no ponto 2, prevêem-se que estejam relacionadas com o domínio do produto. Alguns exemplos são:

\begin{itemize}
\item 
    {
        \textit{What's the total of dispatch operations per week during October?} -- \idest{Quantas operações \textit{dispatch} foram executadas por semana, durante o mês de outubro?};
    }
    \item
    {
        \textit{How many trackout operations were executed by product, on September, per shift?} -- \idest{Quantas operações \textit{track-out} foram executadas, por produto e turno, durante o mês de setembro?};
    }
    \item
    {
        \textit{What's the average primary quantity of trackin operations on burnin step, on september?} -- \idest{Qual a média de quantidade primária, no passo \textit{burnin} do processo, por turno, no mês de Setembro?};
    }
    \item
    {
        \textit{How many materials, by operation and on wire inspection step, have primary quantity inferior to 1000?} -- \idest{Quantos materiais, por operação e no passo \textit{wire inspection}, têm a quantidade primária inferior a 1000?}.
    }
\end{itemize}

As questões apresentadas são exemplos fornecidos pelo supervisor do projeto e correspondem aos tipos de pesquisa mais relevantes no {\productname}.

Relativamente à metodologia de validação dos critérios os critérios mencionados, optam-se por testes manuais de interação com a plataforma subjacente e também com o protótipo desenvolvido.
    
% Nas questões apresentadas, as letras representam as variáveis inerentes ao domínio, que o utilizador conhece.
    
    % \item
    % {
    %     \textit{O modelo é adotado ou refinado de forma a que possa ser usado na solução final} -- o modelo arquitetado revela-se efetivo na resolução do problema, e com o levantamento de possíveis melhorias, pode ser implementado no {\productname}.
    % }
    % \item 
    % {
    %     \textit{Tese escrita} -- na qual se abordam o problema, o contexto no qual se insere e o valor que traz ao produto final. Deve conter o estado da arte, apresentando a revisão da literatura existente, focando nas soluções semelhantes e/ou ferramentas relevantes que perspetivam estratégias de solução para o problema. Por fim, descreve-se a solução proposta, contemplando cada uma das fases inerentes ao seu desenvolvimento (visão, análise, desenho e implementação) e faz-se a conclusão acerca do trabalho (todos os objetivos descritos em~\ref{sec:chap01_objectives});
    % }



% \subsection{Formulação das Hipóteses}
% \label{sec:chap01_hypothesis}
% Para a resolução do problema da empresa, o qual foca a melhoria da interação do {\productname} com o utilizador, surgem as seguintes questões:

% \begin{enumerate}
%     \item
%     {
%         A integração de um módulo de linguagem natural pode, de facto, melhorar a usabilidade do produto e consequentemente, simplificar processo de apoio à decisão?
%     }
%     \item
%     {
%         De que forma se pode avaliar a adequabilidade das respostas da solução às necessidades básicas dos utilizadores?
%     }
% \end{enumerate}

% Embora as perguntas anteriores sejam relevantes para a formulação de hipóteses para a solução final, e devem ser tidas em consideração na abordagem escolhida, não terão um peso significativo na avaliação do resultado deste trabalho. O foco desta tese é o desenvolvimento de um protótipo, cuja abordagem pode ser seguida na implementação de uma solução definitiva no {\productname}. Assim, surge outra pergunta mais pertinente para esta fase, e respetiva hipótese:

% \begin{itemize}
%     \item
%     {  
%         \textit{Questão} -- Qual o modelo adequado para a extração de informação de um sistema, usando linguagem natural?
%     }
%     \item
%     {
%         \textit{Hipótese} -- O modelo escolhido permite a extração de informação a partir de linguagem natural.
%     }
% \end{itemize}

% A hipótese apresentada auxilia na definição da metodologia de avaliação a adotar nesta tese. A aceitação ou refutação da hipótese formulada permite concluir acerca do trabalho realizado, e da necessidade de reformulação ou adoção de novas hipóteses.

% \subsection{Metodologia de Avaliação}
% Com o propósito de perceber se o modelo idealizado, aplicado no protótipo desenvolvido, é adequado para o desenvolvimento de uma solução definitiva, e levando em consideração a hipótese formulada anteriormente, definem-se as seguintes estratégias para a metodologia de avaliação deste trabalho:

% \begin{enumerate}
%     \item 
%     {
%         \textit{Garantir que o protótipo analisa e responde corretamente a um conjunto de perguntas pré-definidas} -- a solução deverá responder adequadamente a um conjunto limitado de perguntas:
%         \begin{itemize}
%             \item 
%             {
%                 Quantas operações $O$ foram executadas por semana, durante o mês $M$?
%             }
%             \item
%             {
%                 Qual o número de operações $O$ por produto e turno, durante o mês $M$?
%             }
%             \item
%             {
%                 Qual a média de operações $O$, no passo $P$ do processo, por turno, no mês $M$? 
%             }
%             \item
%             {
%                 Qual o número de materiais cujo valor de $X$ é inferior a $Y$, para o passo $P$ do processo, agrupando por $G$?
%             }
%         \end{itemize}
        
%         Nas questões apresentadas, as letras representam as variáveis inerentes ao domínio, que o utilizador conhece e que o sistema deve ser capaz de reconhecer. É importante referir que, na reposta às perguntas referidas anteriormente, é considerado o conjunto de dados de exemplo, entregue pelo supervisor desta tese.
%     }
%     \item
%     {
%         \textit{Usar as respostas devolvidas pelo protótipo para concluir acerca da sua exatidão} -- as respostas fornecidas pelo protótipo, face à resposta expectável, permitirão perceber se a abordagem seguida é adequada.
%     }
% \end{enumerate}

% Ambas estratégias possibilitam perceber a adequabilidade do modelo proposto, para a solução a ser integrada no produto e para o utilizador final, quer numa perspetiva de facilidade de utilização, quer na exatidão da resposta dada.

%%%%%%%%%%%%%%%%%%%%%%%%%%%%%%%%%
%           SECTION
%%%%%%%%%%%%%%%%%%%%%%%%%%%%%%%%%
\section{Contribuições}
\label{sec:chap01_contributions}
O trabalho desenvolvido pretende providenciar uma abordagem de resolução do problema apresentado. Não se aspira fornecer uma solução definitiva, espera-se sim, contribuir com conhecimento de carácter teórico e prático (\idest{um protótipo}), que possibilite a integração futura de uma nova funcionalidade num produto já existente. No decorrer deste trabalho foram abordados temas relativos a \gls{mes} e a \gls{ia}, especificamente \gls{pln} e \gls{iln}. Resumidamente, as contribuições esperadas são a seguir mencionadas:
%
\begin{itemize}
    \item
    {
        Estado da arte no domínio de Processamento de Linguagem Natural e sistemas análogos à solução a desenvolver;
    }
    \item 
    {
        Definição de um modelo que possibilita a \inquotes{conversão} da linguagem natural em informação pertinente para o utilizador do sistema; 
    }
    \item
    {
        Especificação e desenvolvimento do protótipo, considerando uma possível integração com o sistema {\productname}.
    }
    \item 
    {
        Documentação dos requisitos de sistema, incluindo análise e desenho, constando os respetivos artefactos de \gls{uml};
    }
\end{itemize}

De uma forma geral, é realçada a contribuição para o avanço do conhecimento no domínio da \gls{ia}, mais especificamente na área das \glspl{iln}, aplicada ao contexto dos sistemas \glspl{mes}.

%%%%%%%%%%%%%%%%%%%%%%%%%%%%%%%%%
%           SECTION
%%%%%%%%%%%%%%%%%%%%%%%%%%%%%%%%%
\section{Plano e Método de Trabalho}
\label{sec:chap01_workmethodology}
Um plano de trabalho é uma ferramenta importante na gestão de qualquer projeto, na medida em que descreve as fases que o compõem e as tarefas inerentes. Ele está sujeito a alterações ao longo do tempo de vida do projeto, pelo que o plano definido inicialmente pode ser encontrado no Apêndice~\ref{AppendixA}. Relativamente às fases do trabalho, a informação sucinta de cada uma é aqui explanada:

\begin{itemize}
    \item
    {
        \textit{Conceção} -- engloba as tarefas que relacionadas com o problema, o seu enquadramento, o estudo do valor e estado da arte. Ou seja, a visão do projeto;
    }
    \item
    {
        \textit{Análise} -- nesta fase, faz-se um estudo exploratório e de caráter empírico de forma a experimentar a aplicação de diferentes abordagens. Definem-se os casos de uso e arquitetura considerada para o modelo. A presente fase consiste no estudo e preparação para a aplicação do modelo no contexto prático; 
    }
    \item
    {
        \textit{Desenvolvimento} -- desenvolve-se o protótipo com base no modelo conjeturado na fase anterior, envolvendo um período de experimentação;
    }
    \item
    {
        \textit{Validação} -- avalia-se a solução com base nos critérios de sucesso definidos e consequentemente, podem-se registar melhorias e retirar as devidas conclusões;
    }
    \item
    {
        \textit{Documentação} -- engloba a escrita da tese como veículo de transmissão de conhecimento obtido.
    }
\end{itemize}

Quanto ao método de trabalho a seguir na realização deste trabalho, são considerados os seguintes passos:

\begin{enumerate}
    \item 
    {
        \textit{Revisão de literatura disponível sobre o contexto do problema} -- com o objetivo de perceber o estado atual do {\productname} e as implicações que o módulo pode trazer, assim como concluir acerca da relevância do problema e do valor da solução para o produto;
    }
    \item
    {
        \textit{Revisão de literatura existente acerca de \gls{pln} e \gls{iln}, paradigmas arquiteturais relacionados e trabalhos de referência} -- adquirir conhecimentos sobre o estado do \gls{pln} e das \gls{iln}, quais os trabalhos de referência na área, outros também relevantes, técnicas e ferramentas usadas, identificando aspetos relevantes para o trabalho;
    }
    \item
    {
        \textit{Idealização do modelo} -- depois da análise dos conhecimentos adquiridos com as revisões realizadas nos passos descritos anteriormente,
        parte-se para a experimentação de diversas abordagens, definição dos casos de uso e arquitetura prevista;
    }
    \item
    {
        \textit{Implementação do protótipo e validação} -- o foco é pôr em prática a solução concetualizada nas fases anteriores. Após a implementação, valida-se a mesma, de acordo com os critérios de sucesso definidos;
    }
    \item
    {
        \textit{Elaboração da documentação} --  por fim, passa-se à escrita do presente documento e de documentos de suporte, baseando-se nas observações, nas experiências e conclusões obtidas ao longo do projeto.
    }
\end{enumerate}
