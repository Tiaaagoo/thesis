\chapter{Introdução}
\label{chap:Chapter1}
Num mercado crescentemente competitivo e exigente, a necessidade de inovar, de obter vantagem competitiva e simultaneamente, tornar os processos industriais simples e altamente eficazes, recorrendo às tecnologias mais atuais, abrem caminho a uma nova mudança. O fenómeno da Indústria 4.0 surge como a nova (quarta) revolução industrial, baseando-se nas mais recentes tecnologias, que incluem os sistemas ciber-físicos, a \gls{iot} e a \gls{ios}, as quais se baseiam na comunicação através da Internet, permitindo uma interação contínua e partilha de informação entre humanos, entre máquinas e entre o ser humano e máquina~\parencite{complex_view_industry40}. 

A Indústria 4.0 assenta numa variedade de conceitos fundamentais, de diferentes áreas de conhecimento, nomeadamente a noção de \textit{Smart Factory\footnote{Fábrica Inteligente, equipada com sensores, atuadores e sistemas autónomos, permitindo assim um controlo autónomo de processo.}}, a capacidade de auto-organização através da descentralização dos sistemas produtivos, e a interação entre o mundo físico e o digital~\parencite[Fundamental Concepts, p.240]{industry40}. No entanto, é a capacidade de adaptação à necessidade humana, principalmente a \gls{ihc}, que se explora no presente trabalho.

Segundo~\textcite[p.1]{natural_language_translation_intersaction_ai_hci}~\inquotes{as áreas de \gls{ia} e \glsfirst{ihc} estão, cada vez mais, a influenciar-se mutuamente. Alguns sistemas amplamente usados como o Google Translate, Facebook Graph Search e RelateIQ, escondem a complexidade de sistemas de larga escala de \gls{ia} através de interfaces intuitivas.}\footnote{Tradução livre do autor. No original~\inquotes{The fields of artificial intelligence (AI) and human-computer interaction (HCI) are influencing each other like never before. Widely used systems such as Google Translate, Facebook Graph Search, and RelateIQ hide the complexity of large-scale AI systems behind intuitive interfaces.}.}. Apesar de terem propósitos diferentes, ambas as áreas se complementam, na medida em que se focam na relação entre ser humano e máquina. Se a \gls{ia} tem como objetivo emular o intelecto humano, já a \gls{ihc} foca-se em abordagens empíricas de usabilidade e fatores humanos, que influenciam a forma como os utilizadores interagem com o computador~\parencite{natural_language_translation_intersaction_ai_hci}. 

A capacidade dum sistema interpretar a linguagem dos seres humanos e apresentar a informação de uma forma adequada, principalmente no contexto da Indústria 4.0, destaca-se como um fator impulsionador da adaptabilidade do mundo digital à necessidade humana. Nesse sentido, a área de \gls{pln}, a qual se debruça na capacidade dos computadores \inquotes{entenderem} a linguagem humana~\parencite[p.1]{applied_natural_language_processing_with_python}, permite construir ferramentas capazes de definir ações, extrair conhecimento dum sistema e apresentá-lo num formato adequado, a partir de conteúdo textual especificado pelo utilizador, de acordo com a sua própria linguagem. Portanto, este trabalho foca-se no estudo do \gls{pln} e na definição de uma abordagem que permita interação com um sistema produtivo, através de linguagem natural, possibilitando a consulta e pesquisa de informação. 

O capítulo atual faz um pequeno enquadramento do trabalho, esmiúça o problema a resolver, os objetivos e âmbito da tese, a metodologia de avaliação e experimentação, as contribuições do trabalho para a área e o plano de trabalho e respetivo método a ser seguido. Posteriormente, no Capítulo~\ref{chap:Chapter2}, contextualiza-se o problema, fazendo uma descrição da empresa e negócio, e aborda-se o propósito e relevância da resolução do mesmo. O Capítulo~\ref{chap:Chapter3} descreve o estado da arte, introduzindo os conceitos importantes de explorar, os trabalhos de referência e outros trabalhos relevantes na área, e as ferramentas destacadas para a resolução do problema, perspetivando estratégias e possíveis abordagens de solução. No Capítulo~\ref{chap:Chapter4} expõe-se o processo de experimentação prática e o modelo proposto como solução, revelando a visão, os requisitos identificados e arquitetura prevista. Seguidamente, os Capítulos~\ref{chap:Chapter5} e \ref{chap:Chapter6} abordam, respetivamente, o processo de desenvolvimento do protótipo com base no modelo proposto e a validação deste, tendo em conta os critérios de avaliação estipulados. Por fim, o Capítulo~\ref{chap:Chapter7} anuncia as conclusões do trabalho realizado, inferindo sobre os objetivos e critérios da tese, evidenciando-se limitações da abordagem e problemas por explorar.

%%%%%%%%%%%%%%%%%%%%%%%%%%%%%%%%%
%           SECTION
%%%%%%%%%%%%%%%%%%%%%%%%%%%%%%%%%
\section{Problema}
\label{sec:chap01_problem}
O conceito de \gls{mes}, um sistema que, além de gerir as operações dum determinado processo fabril, mantém dados relativos às diversas etapas inerentes ao processo em questão, está intrinsecamente relacionado com a Indústria 4.0, uma iniciativa que se destina a criar fábricas inteligentes, usando tecnologias como os \glspl{cps}, a \gls{iot} e \textit{Cloud Computing}~\parencite{intelligent_manufacturing_context_industry40_review}. O {\productname} é um destes sistemas. Contudo, a capacidade de adaptação às características dos utilizador é um requisito complexo, que nem sempre é passível de ser cumprido. Isso pode dificultar o uso do produto, numa perspetiva de acesso a informação relevante para o processo e de apoio à decisão~\parencite{intelligent_manufacturing_context_industry40_review}. Por outras palavras, se o utilizador pretende efetuar uma determinada pesquisa, necessita de conhecer os detalhes da ferramenta a usar, ao invés de simplesmente \inquotes{pedir} (através de texto ou voz) ao sistema que lhe devolva o resultado.

\textbf{A conceção de um módulo de linguagem natural para interface com o {\productname}, permitindo a consulta e pesquisa de estados do processo de fabrico}, torna-se importante para o sistema, uma vez que possibilita o utilizador interagir com o sistema de uma forma simples, intuitiva, eficiente e natural, através de escrita.

%%%%%%%%%%%%%%%%%%%%%%%%%%%%%%%%%
%           SECTION
%%%%%%%%%%%%%%%%%%%%%%%%%%%%%%%%%
\section{Objetivos}
\label{sec:chap01_objectives}
De uma maneira geral, com este trabalho pretende-se o desenvolvimento de um protótipo apoiado no modelo idealizado para a solução final. Com o intuito de solucionar o problema enunciado, definem-se os seguintes objetivos:

\begin{enumerate}
    \item
    {
        \textit{Contextualizar o problema numa perspetiva de negócio} -- análise detalhada do problema, as implicações que tem para negócio, para o produto \gls{mes} e para os seus utilizadores, descrevendo a relevância do problema e valor intrínseco à sua resolução (Capítulo~\ref{chap:Chapter1} e \ref{chap:Chapter2});
    }
    \item
    {
        \textit{Estudar soluções disponíveis no mercado e/ou ferramentas de processamento de linguagem natural} -- obtenção de informação da área de conhecimento envolvida, de soluções semelhantes, trabalhos de referência e de ferramentas tipicamente usadas na implementação de soluções análogas (Capítulo~\ref{chap:Chapter3});
    }
    \item
    {
        \textit{Definir a ferramenta e abordagem mais adequadas, considerando as diversas opções apresentadas} -- comparação e avaliação das diversas opções identificadas, concluindo acerca do caminho a seguir (Capítulo~\ref{chap:Chapter3} e \ref{chap:Chapter4});
    }
    \item
    {
        \textit{Especificação da arquitetura prevista para módulo} -- que permita responder aos requisitos definidos e antecipar soluções para possíveis problemas (Capítulo~\ref{chap:Chapter4});
    }
    % \item
    % {
    %     \textit{Descrever a semântica de domínio} -- exploração de abordagens ao tratamento e suporte de diferentes domínios (Capítulo~\ref{chap:Chapter4});
    % }
    \item
    {
        \textit{Desenvolvimento de prova de conceito} -- implementação do protótipo de acordo com a arquitetura conceptualizada (Capítulo~\ref{chap:Chapter5});
    }
    \item
    {
        \textit{Prover o protótipo de um mecanismo de feedback para auto-aprendizagem} -- permitirá a adaptabilidade às necessidades do utilizador, melhorando a qualidade das respostas.
        Numa fase inicial, este mecanismo pode não constar no protótipo, ou pode consistir em simplesmente questionar o utilizador sobre a exatidão da resposta apresentada (Capítulo~\ref{chap:Chapter5});
    }
    \item
    {
        \textit{Avaliar a qualidade da solução desenvolvida} -- com base na hipótese formulada e na estratégia de avaliação decidida, concluir acerca da qualidade da abordagem seguida e do contributo do trabalho para a resolução do problema (Capítulo~\ref{chap:Chapter6} e \ref{chap:Chapter7}).
    }
\end{enumerate}

%%%%%%%%%%%%%%%%%%%%%%%%%%%%%%%%%
%           SECTION
%%%%%%%%%%%%%%%%%%%%%%%%%%%%%%%%%
\section{Âmbito}
\label{sec:chap01_scope}
Embora os objetivos estejam definidos, surge a necessidade de explicitar sucintamente o âmbito deste trabalho, bem como os pressupostos a ter em consideração. Por conseguinte, os seguintes assuntos serão abordados:

\begin{itemize}
    \item
    {
        A contextualização do problema da empresa com a prova de conceito a ser desenvolvida, o seu enquadramento com a Indústria 4.0 e utilidade para o cliente final; 
    }
    \item 
    {
        Os conceitos teóricos e adversidades inerentes ao problema, ainda que explorados de uma forma genérica, evitando abordar pormenores ou especificidades do tema;
    }
    \item
    {
        A apresentação e explicação dos exemplos de resolução de problemas semelhantes por parte de terceiros -- trabalhos de referência --, fazendo um levantamento das características relevantes para este projeto;
    }
    \item
    {
        As ferramentas disponíveis e relevantes para este contexto, passíveis de ser aplicadas na solução final;
    }
    \item
    {
        O método científico e processo de engenharia adotado na busca duma abordagem para resolução do problema em questão.
    }
\end{itemize}

Por outro lado, alguns tópicos são demasiado amplos para serem explorados, ou simplesmente não se enquadram nos objetivos desta tese, pelo que não serão abordados:

\begin{itemize}
    \item
    {
        O enquadramento do problema com outros \glspl{mes}. Apenas é contemplada a realidade do problema no contexto do {\productname};
    }
    \item
    {
        As ferramentas para linguagem natural que não mostrem evidências de relevância para o problema, tendo em conta os critérios de complexidade e adesão da comunidade de desenvolvimento;
    }
    \item 
    {
        A inclusão de diferentes domínios no protótipo desenvolvido.
    }
\end{itemize}

O termo \inquotes{Domínio} é empregue ao longo do texto para denotar um conjunto de características que descrevem uma família de conceitos comuns a um determinado processo. Por exemplo, duas empresas que produzem equipamentos médicos, apesar de poderem ter processos de fabrico diferentes, abordam o mesmo domínio.

Assume-se que a prova de conceito desenvolvida visa provar que o modelo proposto atende ao problema, por isso não se espera que todos os casos de uso ou detalhes esperados para a solução final (\exempligratia{uso de \textit{feedback} de utilizador}) sejam contemplados no protótipo.

%%%%%%%%%%%%%%%%%%%%%%%%%%%%%%%%%
%           SECTION
%%%%%%%%%%%%%%%%%%%%%%%%%%%%%%%%%
\section{Avaliação e Experimentação}
\label{sec:chap01_solutionevaluation}
A avaliação do resultado final é imprescindível para a concluir acerca do sucesso do trabalho, permitindo perceber se a conjetura fundada a respeito da prova de conceito é aceite. Desse modo, formula-se a hipótese colocada para o modelo arquitetado, a respetiva metodologia de avaliação e experimentação e os critérios de sucesso a serem considerados.

\subsection{Formulação das Hipóteses}
\label{sec:chap01_hypothesis}
Para a resolução do problema da empresa, o qual foca a melhoria da interação do {\productname} com o utilizador, surgem as seguintes questões:

\begin{enumerate}
    \item
    {
        A integração de um módulo de linguagem natural pode, de facto, melhorar a usabilidade do produto e consequentemente, simplificar processo de apoio à decisão?
    }
    \item
    {
        De que forma se pode avaliar a adequabilidade das respostas da solução às necessidades básicas dos utilizadores?
    }
\end{enumerate}

Embora as perguntas anteriores sejam relevantes para a formulação de hipóteses para a solução final, e devem ser tidas em consideração na abordagem escolhida, não terão um peso significativo na avaliação do resultado deste trabalho. O foco desta tese é o desenvolvimento de um protótipo, cuja abordagem pode ser seguida na implementação de uma solução definitiva no {\productname}. Assim, surge outra pergunta mais pertinente para esta fase, e respetiva hipótese:

\begin{itemize}
    \item
    {  
        \textit{Questão} -- Qual o modelo adequado para a extração de informação de um sistema, usando linguagem natural?
    }
    \item
    {
        \textit{Hipótese} -- O modelo escolhido permite a extração de informação a partir de linguagem natural.
    }
\end{itemize}

A hipótese apresentada auxilia na definição da metodologia de avaliação a adotar nesta tese. A aceitação ou refutação da hipótese formulada permite concluir acerca do trabalho realizado, e da necessidade de reformulação ou adoção de novas hipóteses.

\subsection{Metodologia de Avaliação}
Com o propósito de perceber se o modelo idealizado, aplicado no protótipo desenvolvido, é adequado para o desenvolvimento de uma solução definitiva, e levando em consideração a hipótese formulada anteriormente, definem-se as seguintes estratégias para a metodologia de avaliação deste trabalho:

\begin{enumerate}
    \item 
    {
        \textit{Garantir que o protótipo analisa e responde corretamente a um conjunto de perguntas pré-definidas} -- a solução deverá responder adequadamente a um conjunto limitado de perguntas:
        \begin{itemize}
            \item 
            {
                Quantas operações $O$ foram executadas por semana, durante o mês $M$?
            }
            \item
            {
                Qual o número de operações $O$ por produto e turno, durante o mês $M$?
            }
            \item
            {
                Qual a média de $X$ de operações $O$, no passo $P$ do processo, por turno, no mês $M$? 
            }
            \item
            {
                Qual o número de materiais cujo valor de $X$ é inferior a $Y$, para o passo $P$ do processo, agrupando por $G$?
            }
        \end{itemize}
        
        Nas questões apresentadas, as letras representam as variáveis inerentes ao domínio, que o utilizador conhece e que o sistema deve ser capaz de reconhecer. É importante referir que, na reposta às perguntas referidas anteriormente, é considerado o conjunto de dados de exemplo, entregue pelo supervisor desta tese.
    }
    \item
    {
        \textit{Usar as respostas devolvidas pelo protótipo para concluir acerca da sua exatidão} -- as respostas fornecidas pelo protótipo, face à resposta expectável, permitirão perceber se a abordagem seguida é adequada.
    }
\end{enumerate}

Ambas estratégias possibilitam perceber a adequabilidade do modelo proposto, para a solução a ser integrada no produto e para o utilizador final, quer numa perspetiva de facilidade de utilização, quer na exatidão da resposta dada.

\subsection{Critérios de Sucesso}
De seguida, enumeram-se os critérios de sucesso para o trabalho:

\begin{enumerate}
    \item 
    {
        \textit{A hipótese apresentada anteriormente é aceite} -- a abordagem (modelo) escolhida apresenta resultados satisfatórios face à metodologia de avaliação definida para este trabalho;
    }
    \item
    {
        \textit{O modelo apresentado é extensível e de fácil integração no {\productname}} -- garante-se que a arquitetura especificada considerou a existência de diversos domínios, facilidade e capacidade de integração com o produto;
    }
    \item
    {
        \textit{Prova de conceito dá resposta correta às questões que lhe são colocadas} -- que implica responder corretamente às questões-chave, estabelecidas na metodologia de avaliação, garantindo que a resposta fornecida é semelhante ou igual à resposta que seria esperada;
    }
    \item
    {
        \textit{O modelo é adotado ou refinado de forma a que possa ser usado na solução final} -- o modelo arquitetado revela-se efetivo na resolução do problema, e com o levantamento de possíveis melhorias, pode ser implementado no {\productname}.
    }
    % \item 
    % {
    %     \textit{Tese escrita} -- na qual se abordam o problema, o contexto no qual se insere e o valor que traz ao produto final. Deve conter o estado da arte, apresentando a revisão da literatura existente, focando nas soluções semelhantes e/ou ferramentas relevantes que perspetivam estratégias de solução para o problema. Por fim, descreve-se a solução proposta, contemplando cada uma das fases inerentes ao seu desenvolvimento (visão, análise, desenho e implementação) e faz-se a conclusão acerca do trabalho (todos os objetivos descritos em~\ref{sec:chap01_objectives});
    % }
\end{enumerate}

%%%%%%%%%%%%%%%%%%%%%%%%%%%%%%%%%
%           SECTION
%%%%%%%%%%%%%%%%%%%%%%%%%%%%%%%%%
\section{Contribuições}
\label{sec:chap01_contributions}
O trabalho a ser desenvolvido pretende providenciar uma abordagem de resolução do problema apresentado. Não se aspira fornecer uma solução definitiva, espera-se sim, contribuir com conhecimento de carácter teórico e prático (\idest{um protótipo}), que possibilite a integração futura de uma nova funcionalidade num produto já existente, trazendo-lhe mais-valia funcional, destacando-o dos seus concorrentes. No decorrer deste trabalho serão abordados temas relativos a \gls{mes}, a \gls{ia}, especificamente \gls{pln}, e \textit{Business Intelligence}.

Resumidamente, as contribuições esperadas são a seguir enunciadas:

\begin{itemize}
    \item
    {
        Estado da arte no domínio de Processamento de Linguagem Natural e sistemas análogos à solução a desenvolver;
    }
    \item 
    {
        Documentação dos requisitos de sistema, incluindo análise e desenho, constando os respetivos artefactos de \gls{uml};
    }
    \item 
    {
        Identificação de abordagens para lidar com semântica de diferentes domínios;
    }
    \item 
    {
        Definição de um modelo que possibilita a \inquotes{conversão} da linguagem natural em informação pertinente para o utilizador do sistema; 
    }
    \item
    {
        Especificação e desenvolvimento do protótipo, considerando a futura integração com o sistema {\productname}.
    }
\end{itemize}

De uma forma geral, é realçada a contribuição para o avanço do conhecimento no domínio da \gls{ia}, mais especificamente na área da \gls{pln}, aplicada ao contexto dos sistemas \gls{mes} e que servirá como base para a integração duma solução deste tipo no {\productname}.

%%%%%%%%%%%%%%%%%%%%%%%%%%%%%%%%%
%           SECTION
%%%%%%%%%%%%%%%%%%%%%%%%%%%%%%%%%
\section{Plano e Método de Trabalho}
\label{sec:chap01_workmethodology}
Um plano de trabalho é uma ferramenta importante na gestão de qualquer projeto, na medida em que descreve as fases que o compõem e as tarefas inerentes. Ele está sujeito a alterações ao longo do tempo de vida do projeto, pelo que o plano definido inicialmente pode ser encontrado no Apêndice~\ref{AppendixA}. Relativamente às fases do trabalho, a informação sucinta de cada uma é aqui explanada:

\begin{itemize}
    \item
    {
        \textit{Conceção} -- engloba as tarefas que relacionadas com o problema, o seu enquadramento, o estudo do valor e estado da arte. Ou seja, a visão do projeto;
    }
    \item
    {
        \textit{Análise} -- nesta fase, faz-se um estudo exploratório e de caráter empírico de forma a experimentar a aplicação de diferentes abordagens. Definem-se os casos de uso e arquitetura considerada para o modelo. A presente fase consiste no estudo e preparação para a aplicação do modelo no contexto prático; 
    }
    \item
    {
        \textit{Desenvolvimento} -- desenvolve-se o protótipo com base no modelo conjeturado na fase anterior, envolvendo um período de experimentação;
    }
    \item
    {
        \textit{Validação} -- avalia-se a solução com base nos critérios de sucesso definidos e consequentemente, podem-se registar melhorias e retirar as devidas conclusões;
    }
    \item
    {
        \textit{Documentação} -- engloba a escrita da tese como veículo de transmissão de conhecimento obtido e de outros documentos de suporte, a serem usados no contexto empresarial (\exempligratia{documento de especificação de requisitos de \textit{software}}).
    }
\end{itemize}

Quanto ao método de trabalho a seguir na realização deste trabalho, são considerados os seguintes passos:

\begin{enumerate}
    \item 
    {
        \textit{Revisão de literatura disponível sobre o contexto do problema} -- com o objetivo de perceber o estado atual do {\productname} e as implicações que o módulo trará, assim como concluir acerca da relevância do problema e do valor da solução para o produto;
    }
    \item
    {
        \textit{Revisão de literatura existente acerca de \gls{pln}, paradigmas arquiteturais relacionados e trabalhos de referência} -- adquirir conhecimentos sobre o estado \gls{pln}, quais os trabalhos de referência na área, outros também relevantes, técnicas e ferramentas usadas, identificando aspetos relevantes para o trabalho;
    }
    \item
    {
        \textit{Idealização do modelo} -- depois da análise dos conhecimentos adquiridos com as revisões realizadas nos passos descritos anteriormente,
        parte-se para a experimentação de diversas abordagens, definição dos casos de uso e arquitetura prevista;
    }
    \item
    {
        \textit{Implementação do protótipo e validação} -- o foco é pôr em prática a solução conceptualizada nas fases anteriores. Após a implementação, valida-se a mesma, com vista a perceber se os resultados obtidos são os esperados e proceder ao registo das melhorias que forem evidentes;
    }
    \item
    {
        \textit{Elaboração da documentação} --  por fim, passa-se à escrita do presente documento e de documentos de suporte, baseando-se nas observações, nas experiências e conclusões obtidas ao longo do projeto.
    }
\end{enumerate}