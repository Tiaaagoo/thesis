%%%%%%%%%%%%%%%%%%%%%%%%%%%%%%%%%%%%%%%%%
% TMDEI Dissertation
% LaTeX Template
% Version 0.1 (Dec/2015)
%
% Adapted to TMDEI/ISEP style (Dec/2015) by
%  Nuno Pereira (nap@isep.ipp.pt) and
%  Paulo Baltarejo (pbs@isep.ipp.pt)
%
% Based on MastersDoctoralThesis Version 1.2 by Vel (vel@latextemplates.com) and
% Johannes Böttcher, downloaded from (21/11/15):
% http://www.LaTeXTemplates.com
%
% This template is originally based on a template by:
% Steve Gunn (http://users.ecs.soton.ac.uk/srg/softwaretools/document/templates/)
% Sunil Patel (http://www.sunilpatel.co.uk/thesis-template/)
%
% Template license:
% CC BY-NC-SA 3.0 (http://creativecommons.org/licenses/by-nc-sa/3.0/)
%
%%%%%%%%%%%%%%%%%%%%%%%%%%%%%%%%%%%%%%%%%

%----------------------------------------------------------------------------------------
%	PACKAGES AND OTHER DOCUMENT CONFIGURATIONS
%----------------------------------------------------------------------------------------

\documentclass[
10pt, % The default document font size, options: 10pt, 11pt, 12pt
%twoside,
oneside, % Two side (alternating margins) for binding by default, uncomment to switch to one side (for drafting/reading purposes)
%english, % english for English;
portuguese,% for Portuguese; delete temporary files if you change language (e.g. 'make clean; make')
singlespacing, % Single line spacing, alternatives: onehalfspacing or doublespacing (for drafting/reading purposes)
%draft, % Uncomment to enable draft mode (no pictures, no links, overfull hboxes indicated)
%nolistspacing, % If the document is onehalfspacing or doublespacing, uncomment this to set spacing in lists to single
liststotoc, % Uncomment to add the list of figures/tables/etc to the table of contents (not recommended)
%toctotoc, % Uncomment to add the main table of contents to the table of contents (not recommended)
parskip, % Add space between paragraphs (recommended)
%nohyperref, % Uncomment to not load the hyperref package (not recommended)
nohyperreflinkcolor, % hyperref links are not colored (comment to color links, for example to produce an electronic-only version)
headsepline, % Uncomment to get a line under the header
]{tmdei-style} % The class file specifying the document structure

\usepackage{subcaption}
\usepackage{rotating}
\usepackage{multicol}
\usepackage{soul}
\usepackage{tikz} % Required for creating graphics programmatically (can be removed if not used)
%\usetikzlibrary{arrows} % Required for fancy arrows in TiKZ graphics (can be removed if not used)
\usetikzlibrary{mindmap,trees}
\usepackage{pgfplots} % Required for drawing high--quality function plots (can be removed if not used)
\pgfplotsset{compat=newest}

%
% Next you have examples of admissable citation styles; we recomend using the authoryear-comp citation style (which resembles Harvard); don't forget to only uncomment one
%

% authoryear-comp: recommended citation style (e.g. (Buendía, 1860), (Buendía 1910, Arcadio 1940))
\usepackage[style=authoryear-comp,backend=biber]{biblatex} % Bibtex backend with the authoryear-comp citation style (authoryear citations, bibliography ordered alphabetically)
%\setlength\bibitemsep{.5\baselineskip}
% numeric citation style (e.g. [1], [1-3])
%\usepackage[style=numeric-comp,sorting=none,backend=biber]{biblatex} % Bibtex backend with the numeric-comp citation style (numeric citations, bibliography ordered by appearance)

% alphabetic citation style (e.g. [Buendía10], [Buendía10, Arcadio40])
%\usepackage[style=alphabetic,sorting=none,backend=biber]{biblatex} % Bibtex backend with the alphabetic citation style (alphabetic citations, bibliography ordered by appearance)
\addbibresource{mainbibliography.bib} % The filename of the bibliography

\makeglossaries % build the glossary

%----------------------------------------------------------------------------------------
%	THESIS INFORMATION
%----------------------------------------------------------------------------------------

\thesistitle{Natural Language Querying} % Your thesis title, this is used in the title, print it elsewhere with \ttitle

%\thesissubtitle{Uma abordagem ao problema das Interfaces de Linguagem Natural} % Your thesis title, this is used in the title, print it elsewhere with \tsubtitle

\author{Tiago Gabriel da Silva} % Your name, this is used in the title page, print it elsewhere with \authorname

\subjectarea{Sistemas Computacionais} % Specialization area (Computer Systems, Information and Knowledge Systems, Graphics, Systems and Multimedia, Software Engineering), used in the title page, print it elsewhere with \areaname

\supervisor{Prof. Doutor Paulo Gandra de Sousa} % Your supervisor's name, this is used in the title page, print it elsewhere with \supname

\cosupervisor{{Eng.\textordmasculine} Ricardo Magalhães} % Your co-supervisor's name, this is used in the title page, print it elsewhere with \cosupname (comment, if no co-supervisor)

\committeepresident{{Prof.\textordfeminine} Doutora Isabel Azevedo, DEI/ISEP} % Name of the president of the evaluation committee, print it elsewhere with \presidentname

\committeemembers{Prof. Doutor Nuno Escudeiro, DEI/ISEP} % Name of the evaluation committee members (up to four), print it elsewhere with \committee

\keywords{\textit{Manufacturing Execution System}, Indústria 4.0, Inteligência Artificial, Interfaces de Linguagem Natural, \textit{Machine Learning}} % Please define up to 6 keywords that better describe your work, print it elsewhere with \keywordnames

\university{\href{http://www.isep.ipp.pt}{Instituto Superior de Engenharia do Porto}} % Your university's name and URL, this is used in the title page and abstract, print it elsewhere with \univname

\department{\href{http://www.dei.isep.ipp.pt}{Departamento de Engenharia Informática}} % Your department's name and URL, this is used in the title page and abstract, print it elsewhere with \deptname

\thesisdate{Porto, \today} % thesis date,  print it elsewhere with \tdate
% \thesisdate{Porto, {\ifcase \month\or Janeiro\or Fevereiro\or Março\or Abril\or Maio\or Junho\or Julho\or Agosto\or Setembro\or Outubro\or Novembro\or Dezembro\fi \space \number\year}
% }

\hypersetup{pdftitle=\ttitle} % Set the PDF's title to your title
\hypersetup{pdfauthor=\authorname} % Set the PDF's author to your name
\hypersetup{pdfkeywords=\keywordnames} % Set the PDF's keywords to your keywords

% Variables
\def\companyname{Critical Manufacturing}
\def\productname{Critical Manufacturing MES}
\def\tbd{\hl{\textit{A definir}}}

\begin{document}

%----------------------------------------------------------------------------------------
%	FRONT MATTER
%----------------------------------------------------------------------------------------

% Include the frontmatter of your thesis here
% we include the glossary here (frontmatter is included with \input, so this command is as if it was in main.tex)
% Acrónimos
\newacronym{iot}{IoT}{\textit{Internet of Things}}
\newacronym{ios}{IoS}{\textit{Internet of Services}}
\newacronym{ihc}{IHC}{Interação Humano-Computador}
\newacronym{ia}{IA}{Inteligência Artificial}
\newacronym{pln}{PLN}{Processamento de Linguagem Natural}
\newacronym{mes}{MES}{\textit{Manufacturing Execution System}}
\newacronym{sql}{SQL}{\textit{Structured Query Language}}
\newacronym{uml}{UML}{\textit{Unified Modeling Language}}
\newacronym{ceo}{CEO}{\textit{Chief Executive Officer}}
\newacronym{cto}{CTO}{\textit{Chief Technical Officer}}
\newacronym{erp}{ERP}{\textit{Enterprise Resource Planning}}
\newacronym{cps}{CPS}{\textit{Cyber-Physical System}}
\newacronym{ffe}{FFE}{\textit{Fuzzy Front End}}
\newacronym{npd}{NPD}{\textit{New Product Development}}
\newacronym{ncd}{NCD}{\textit{New Concept Development}}
\newacronym{slp}{SLP}{\textit{Single Layer Perceptron}}
\newacronym{ilnbd}{ILNBD}{Interfaces de Linguagem Natural para Bases de Dados}
\newacronym{atn}{ATN}{\textit{Augmented Transition Network}}
% for defining plural form
% \newacronym[shortplural=aa,longplural=letters a]{a}{A}{the a}

\frontmatter % Use roman page numbering style (i, ii, iii, iv...) for the pre-content pages

\pagestyle{plain} % Default to the plain heading style until the thesis style is called for the body content

%----------------------------------------------------------------------------------------
%	TITLE PAGE
%----------------------------------------------------------------------------------------

\maketitlepage

%----------------------------------------------------------------------------------------
%	DEDICATION  (optional)
%----------------------------------------------------------------------------------------
%\dedicatory{For/Dedicated to/To my\ldots}
\begin{dedicatory}
\tbd
\end{dedicatory}

%----------------------------------------------------------------------------------------
%	ABSTRACT PAGE
%----------------------------------------------------------------------------------------
\begin{abstract}

O paradigma de interação entre Homem e Máquina tem vindo a mudar nos últimos anos. Se, ao longo das últimas décadas, o ser humano tem vindo a interagir com o computador através da escrita (linha de comandos) ou das interfaces gráficas, como se desenvolverá esta interação quando a máquina for capaz de \inquotes{entender} a linguagem natural humana?

\end{abstract}

\begin{abstractotherlanguage}
The interaction paradigm between man and machine has been changing in the last years. Over the last decades, humans have been interacting with the computer through writing (command line) or graphical interfaces. Recently, emerges the interaction through natural language. How to enhance the communication between man and the system used on daily basis, by using natural language? The usage of Natural Language Processing, a field of study of Artificial Intelligence, which may involve Machine Learning or Deep Learning techniques, allows the transformation of human language into a representation adapted to computation systems.

This thesis focus on the design of an approach that allows to consult and present information stored in data warehouses, through usage of natural language. As result, a prototype has been developed by putting into practise the conceptualized approach. Thus, the main goal is to adapt and use the suggested approach in the development of a natural language module to interact with the {\productname}, thereby improving the system's usability.

\end{abstractotherlanguage}

%----------------------------------------------------------------------------------------
%	ACKNOWLEDGEMENTS (optional)
%----------------------------------------------------------------------------------------
\begin{acknowledgements}

\tbd

\end{acknowledgements}

%----------------------------------------------------------------------------------------
%	LIST OF CONTENTS/FIGURES/TABLES PAGES
%----------------------------------------------------------------------------------------

\tableofcontents % Prints the main table of contents

\listoffigures % Prints the list of figures

\listoftables % Prints the list of tables

\iflanguage{portuguese}{
\renewcommand{\listalgorithmname}{Lista de Algor\'itmos}
}
\listofalgorithms % Prints the list of algorithms
\addchaptertocentry{\listalgorithmname}


\renewcommand{\lstlistlistingname}{List of Source Code}
\iflanguage{portuguese}{
\renewcommand{\lstlistlistingname}{Lista de C\'odigo}
}
\lstlistoflistings % Prints the list of listings (programming language source code)
\addchaptertocentry{\lstlistlistingname}


%----------------------------------------------------------------------------------------
%	ABBREVIATIONS
%----------------------------------------------------------------------------------------
%\begin{abbreviations}{ll} % Include a list of abbreviations (a table of two columns)
%%\textbf{LAH} & \textbf{L}ist \textbf{A}bbreviations \textbf{H}ere\\
%%\textbf{WSF} & \textbf{W}hat (it) \textbf{S}tands \textbf{F}or\\
%\end{abbreviations}

%----------------------------------------------------------------------------------------
%	SYMBOLS
%----------------------------------------------------------------------------------------

\begin{symbols}{lll} % Include a list of Symbols (a three column table)

$a$ & distance & \si{\meter} \\
$P$ & power & \si{\watt} (\si{\joule\per\second}) \\
%Symbol & Name & Unit \\

\addlinespace % Gap to separate the Roman symbols from the Greek

$\omega$ & angular frequency & \si{\radian} \\

\end{symbols}



%----------------------------------------------------------------------------------------
%	ACRONYMS
%----------------------------------------------------------------------------------------

\newcommand{\listacronymname}{List of Acronyms}
\iflanguage{portuguese}{
\renewcommand{\listacronymname}{Lista de Acr\'onimos}
}

%Use GLS
\glsresetall
\printglossary[title=\listacronymname,type=\acronymtype,style=long]

%----------------------------------------------------------------------------------------
%	DONE
%----------------------------------------------------------------------------------------

\mainmatter % Begin numeric (1,2,3...) page numbering
\pagestyle{thesis} % Return the page headers back to the "thesis" style


%----------------------------------------------------------------------------------------
%	MAIN BODY
%----------------------------------------------------------------------------------------

% Include the chapters of the thesis as separate folder for each chapter
% Uncomment the lines as you write the chapters

\chapter{Introdução}
\label{chap:Chapter1}
Num mercado crescentemente competitivo e exigente, a necessidade de inovar, de obter vantagem competitiva e simultaneamente, tornar os processos industriais simples e altamente eficazes, recorrendo às tecnologias mais atuais, abrem caminho a uma nova mudança. O fenómeno da Indústria 4.0 surge como a nova (quarta) revolução industrial, baseando-se nas mais recentes tecnologias, que incluem os sistemas ciber-físicos, a \gls{iot} e a \gls{ios}, as quais se baseiam na comunicação através da Internet, permitindo uma interação contínua e partilha de informação entre humanos, entre máquinas e entre o ser humano e máquina~\parencite{complex_view_industry40}. 

A Indústria 4.0 assenta numa variedade de conceitos fundamentais, de diferentes áreas de conhecimento, nomeadamente a noção de \textit{Smart Factory\footnote{Fábrica Inteligente, equipada com sensores, atuadores e sistemas autónomos, permitindo assim um controlo autónomo de processo.}}, a capacidade de auto-organização através da descentralização dos sistemas produtivos, e a interação entre o mundo físico e o digital~\parencite[Fundamental Concepts, p.240]{industry40}. No entanto, é a capacidade de adaptação à necessidade humana, principalmente a \gls{ihc}, que se explora no presente trabalho.

Segundo~\textcite[p.1]{natural_language_translation_intersaction_ai_hci}~\inquotes{as áreas de \gls{ia} e \glsfirst{ihc} estão, cada vez mais, a influenciar-se mutuamente. Alguns sistemas amplamente usados como o Google Translate, Facebook Graph Search e RelateIQ, escondem a complexidade de sistemas de larga escala de \gls{ia} através de interfaces intuitivas.}\footnote{Tradução livre do autor. No original~\inquotes{The fields of artificial intelligence (AI) and human-computer interaction (HCI) are influencing each other like never before. Widely used systems such as Google Translate, Facebook Graph Search, and RelateIQ hide the complexity of large-scale AI systems behind intuitive interfaces.}.}. Apesar de terem propósitos diferentes, ambas as áreas se complementam, na medida em que se focam na relação entre ser humano e máquina. Se a \gls{ia} tem como objetivo emular o intelecto humano, já a \gls{ihc} foca-se em abordagens empíricas de usabilidade e fatores humanos, que influenciam a forma como os utilizadores interagem com o computador~\parencite{natural_language_translation_intersaction_ai_hci}. 

A capacidade dum sistema interpretar a linguagem dos seres humanos e apresentar a informação de uma forma adequada, principalmente no contexto da Indústria 4.0, destaca-se como um fator impulsionador da adaptabilidade do mundo digital à necessidade humana. Nesse sentido, a área de \gls{pln}, a qual se debruça na capacidade dos computadores \inquotes{entenderem} a linguagem humana~\parencite[p.1]{applied_natural_language_processing_with_python}, permite construir ferramentas capazes de definir ações, extrair conhecimento dum sistema e apresentá-lo num formato adequado, a partir de conteúdo textual especificado pelo utilizador, de acordo com a sua própria linguagem. Portanto, este trabalho foca-se no estudo do \gls{pln} e na definição de uma abordagem que permita interação com um sistema produtivo, através de linguagem natural, possibilitando a consulta e pesquisa de informação. 

O capítulo atual faz um pequeno enquadramento do trabalho, esmiúça o problema a resolver, os objetivos e âmbito da tese, a metodologia de avaliação e experimentação, as contribuições do trabalho para a área e o plano de trabalho e respetivo método a ser seguido. Posteriormente, no Capítulo~\ref{chap:Chapter2}, contextualiza-se o problema, fazendo uma descrição da empresa e negócio, e aborda-se o propósito e relevância da resolução do mesmo. O Capítulo~\ref{chap:Chapter3} descreve o estado da arte, introduzindo os conceitos importantes de explorar, os trabalhos de referência e outros trabalhos relevantes na área, e as ferramentas destacadas para a resolução do problema, perspetivando estratégias e possíveis abordagens de solução. No Capítulo~\ref{chap:Chapter4} expõe-se o processo de experimentação prática e o modelo proposto como solução, revelando a visão, os requisitos identificados e arquitetura prevista. Seguidamente, os Capítulos~\ref{chap:Chapter5} e \ref{chap:Chapter6} abordam, respetivamente, o processo de desenvolvimento do protótipo com base no modelo proposto e a validação deste, tendo em conta os critérios de avaliação estipulados. Por fim, o Capítulo~\ref{chap:Chapter7} anuncia as conclusões do trabalho realizado, inferindo sobre os objetivos e critérios da tese, evidenciando-se limitações da abordagem e problemas por explorar.

%%%%%%%%%%%%%%%%%%%%%%%%%%%%%%%%%
%           SECTION
%%%%%%%%%%%%%%%%%%%%%%%%%%%%%%%%%
\section{Problema}
\label{sec:chap01_problem}
O conceito de \gls{mes}, um sistema que, além de gerir as operações dum determinado processo fabril, mantém dados relativos às diversas etapas inerentes ao processo em questão, está intrinsecamente relacionado com a Indústria 4.0, uma iniciativa que se destina a criar fábricas inteligentes, usando tecnologias como os \glspl{cps}, a \gls{iot} e \textit{Cloud Computing}~\parencite{intelligent_manufacturing_context_industry40_review}. O {\productname} é um destes sistemas. Contudo, a capacidade de adaptação às características dos utilizador é um requisito complexo, que nem sempre é passível de ser cumprido. Isso pode dificultar o uso do produto, numa perspetiva de acesso a informação relevante para o processo e de apoio à decisão~\parencite{intelligent_manufacturing_context_industry40_review}. Por outras palavras, se o utilizador pretende efetuar uma determinada pesquisa, necessita de conhecer os detalhes da ferramenta a usar, ao invés de simplesmente \inquotes{pedir} (através de texto ou voz) ao sistema que lhe devolva o resultado.

\textbf{A conceção de um módulo de linguagem natural para interface com o {\productname}, permitindo a consulta e pesquisa de estados do processo de fabrico}, torna-se importante para o sistema, uma vez que possibilita o utilizador interagir com o sistema de uma forma simples, intuitiva, eficiente e natural, através de escrita.

%%%%%%%%%%%%%%%%%%%%%%%%%%%%%%%%%
%           SECTION
%%%%%%%%%%%%%%%%%%%%%%%%%%%%%%%%%
\section{Objetivos}
\label{sec:chap01_objectives}
De uma maneira geral, com este trabalho pretende-se o desenvolvimento de um protótipo apoiado no modelo idealizado para a solução final. Com o intuito de solucionar o problema enunciado, definem-se os seguintes objetivos:

\begin{enumerate}
    \item
    {
        \textit{Contextualizar o problema numa perspetiva de negócio} -- análise detalhada do problema, as implicações que tem para negócio, para o produto \gls{mes} e para os seus utilizadores, descrevendo a relevância do problema e valor intrínseco à sua resolução (Capítulo~\ref{chap:Chapter1} e \ref{chap:Chapter2});
    }
    \item
    {
        \textit{Estudar soluções disponíveis no mercado e/ou ferramentas de processamento de linguagem natural} -- obtenção de informação da área de conhecimento envolvida, de soluções semelhantes, trabalhos de referência e de ferramentas tipicamente usadas na implementação de soluções análogas (Capítulo~\ref{chap:Chapter3});
    }
    \item
    {
        \textit{Definir a ferramenta e abordagem mais adequadas, considerando as diversas opções apresentadas} -- comparação e avaliação das diversas opções identificadas, concluindo acerca do caminho a seguir (Capítulo~\ref{chap:Chapter3} e \ref{chap:Chapter4});
    }
    \item
    {
        \textit{Especificação da arquitetura prevista para módulo} -- que permita responder aos requisitos definidos e antecipar soluções para possíveis problemas (Capítulo~\ref{chap:Chapter4});
    }
    % \item
    % {
    %     \textit{Descrever a semântica de domínio} -- exploração de abordagens ao tratamento e suporte de diferentes domínios (Capítulo~\ref{chap:Chapter4});
    % }
    \item
    {
        \textit{Desenvolvimento de prova de conceito} -- implementação do protótipo de acordo com a arquitetura conceptualizada (Capítulo~\ref{chap:Chapter5});
    }
    \item
    {
        \textit{Prover o protótipo de um mecanismo de feedback para auto-aprendizagem} -- permitirá a adaptabilidade às necessidades do utilizador, melhorando a qualidade das respostas.
        Numa fase inicial, este mecanismo pode não constar no protótipo, ou pode consistir em simplesmente questionar o utilizador sobre a exatidão da resposta apresentada (Capítulo~\ref{chap:Chapter5});
    }
    \item
    {
        \textit{Avaliar a qualidade da solução desenvolvida} -- com base na hipótese formulada e na estratégia de avaliação decidida, concluir acerca da qualidade da abordagem seguida e do contributo do trabalho para a resolução do problema (Capítulo~\ref{chap:Chapter6} e \ref{chap:Chapter7}).
    }
\end{enumerate}

%%%%%%%%%%%%%%%%%%%%%%%%%%%%%%%%%
%           SECTION
%%%%%%%%%%%%%%%%%%%%%%%%%%%%%%%%%
\section{Âmbito}
\label{sec:chap01_scope}
Embora os objetivos estejam definidos, surge a necessidade de explicitar sucintamente o âmbito deste trabalho, bem como os pressupostos a ter em consideração. Por conseguinte, os seguintes assuntos serão abordados:

\begin{itemize}
    \item
    {
        A contextualização do problema da empresa com a prova de conceito a ser desenvolvida, o seu enquadramento com a Indústria 4.0 e utilidade para o cliente final; 
    }
    \item 
    {
        Os conceitos teóricos e adversidades inerentes ao problema, ainda que explorados de uma forma genérica, evitando abordar pormenores ou especificidades do tema;
    }
    \item
    {
        A apresentação e explicação dos exemplos de resolução de problemas semelhantes por parte de terceiros -- trabalhos de referência --, fazendo um levantamento das características relevantes para este projeto;
    }
    \item
    {
        As ferramentas disponíveis e relevantes para este contexto, passíveis de ser aplicadas na solução final;
    }
    \item
    {
        O método científico e processo de engenharia adotado na busca duma abordagem para resolução do problema em questão.
    }
\end{itemize}

Por outro lado, alguns tópicos são demasiado amplos para serem explorados, ou simplesmente não se enquadram nos objetivos desta tese, pelo que não serão abordados:

\begin{itemize}
    \item
    {
        O enquadramento do problema com outros \glspl{mes}. Apenas é contemplada a realidade do problema no contexto do {\productname};
    }
    \item
    {
        As ferramentas para linguagem natural que não mostrem evidências de relevância para o problema, tendo em conta os critérios de complexidade e adesão da comunidade de desenvolvimento;
    }
    \item 
    {
        A inclusão de diferentes domínios no protótipo desenvolvido.
    }
\end{itemize}

O termo \inquotes{Domínio} é empregue ao longo do texto para denotar um conjunto de características que descrevem uma família de conceitos comuns a um determinado processo. Por exemplo, duas empresas que produzem equipamentos médicos, apesar de poderem ter processos de fabrico diferentes, abordam o mesmo domínio.

Assume-se que a prova de conceito desenvolvida visa provar que o modelo proposto atende ao problema, por isso não se espera que todos os casos de uso ou detalhes esperados para a solução final (\exempligratia{uso de \textit{feedback} de utilizador}) sejam contemplados no protótipo.

%%%%%%%%%%%%%%%%%%%%%%%%%%%%%%%%%
%           SECTION
%%%%%%%%%%%%%%%%%%%%%%%%%%%%%%%%%
\section{Avaliação e Experimentação}
\label{sec:chap01_solutionevaluation}
A avaliação do resultado final é imprescindível para a concluir acerca do sucesso do trabalho, permitindo perceber se a conjetura fundada a respeito da prova de conceito é aceite. Desse modo, formula-se a hipótese colocada para o modelo arquitetado, a respetiva metodologia de avaliação e experimentação e os critérios de sucesso a serem considerados.

\subsection{Formulação das Hipóteses}
\label{sec:chap01_hypothesis}
Para a resolução do problema da empresa, o qual foca a melhoria da interação do {\productname} com o utilizador, surgem as seguintes questões:

\begin{enumerate}
    \item
    {
        A integração de um módulo de linguagem natural pode, de facto, melhorar a usabilidade do produto e consequentemente, simplificar processo de apoio à decisão?
    }
    \item
    {
        De que forma se pode avaliar a adequabilidade das respostas da solução às necessidades básicas dos utilizadores?
    }
\end{enumerate}

Embora as perguntas anteriores sejam relevantes para a formulação de hipóteses para a solução final, e devem ser tidas em consideração na abordagem escolhida, não terão um peso significativo na avaliação do resultado deste trabalho. O foco desta tese é o desenvolvimento de um protótipo, cuja abordagem pode ser seguida na implementação de uma solução definitiva no {\productname}. Assim, surge outra pergunta mais pertinente para esta fase, e respetiva hipótese:

\begin{itemize}
    \item
    {  
        \textit{Questão} -- Qual o modelo adequado para a extração de informação de um sistema, usando linguagem natural?
    }
    \item
    {
        \textit{Hipótese} -- O modelo escolhido permite a extração de informação a partir de linguagem natural.
    }
\end{itemize}

A hipótese apresentada auxilia na definição da metodologia de avaliação a adotar nesta tese. A aceitação ou refutação da hipótese formulada permite concluir acerca do trabalho realizado, e da necessidade de reformulação ou adoção de novas hipóteses.

\subsection{Metodologia de Avaliação}
Com o propósito de perceber se o modelo idealizado, aplicado no protótipo desenvolvido, é adequado para o desenvolvimento de uma solução definitiva, e levando em consideração a hipótese formulada anteriormente, definem-se as seguintes estratégias para a metodologia de avaliação deste trabalho:

\begin{enumerate}
    \item 
    {
        \textit{Garantir que o protótipo analisa e responde corretamente a um conjunto de perguntas pré-definidas} -- a solução deverá responder adequadamente a um conjunto limitado de perguntas:
        \begin{itemize}
            \item 
            {
                Quantas operações $O$ foram executadas por semana, durante o mês $M$?
            }
            \item
            {
                Qual o número de operações $O$ por produto e turno, durante o mês $M$?
            }
            \item
            {
                Qual a média de $X$ de operações $O$, no passo $P$ do processo, por turno, no mês $M$? 
            }
            \item
            {
                Qual o número de materiais cujo valor de $X$ é inferior a $Y$, para o passo $P$ do processo, agrupando por $G$?
            }
        \end{itemize}
        
        Nas questões apresentadas, as letras representam as variáveis inerentes ao domínio, que o utilizador conhece e que o sistema deve ser capaz de reconhecer. É importante referir que, na reposta às perguntas referidas anteriormente, é considerado o conjunto de dados de exemplo, entregue pelo supervisor desta tese.
    }
    \item
    {
        \textit{Usar as respostas devolvidas pelo protótipo para concluir acerca da sua exatidão} -- as respostas fornecidas pelo protótipo, face à resposta expectável, permitirão perceber se a abordagem seguida é adequada.
    }
\end{enumerate}

Ambas estratégias possibilitam perceber a adequabilidade do modelo proposto, para a solução a ser integrada no produto e para o utilizador final, quer numa perspetiva de facilidade de utilização, quer na exatidão da resposta dada.

\subsection{Critérios de Sucesso}
De seguida, enumeram-se os critérios de sucesso para o trabalho:

\begin{enumerate}
    \item 
    {
        \textit{A hipótese apresentada anteriormente é aceite} -- a abordagem (modelo) escolhida apresenta resultados satisfatórios face à metodologia de avaliação definida para este trabalho;
    }
    \item
    {
        \textit{O modelo apresentado é extensível e de fácil integração no {\productname}} -- garante-se que a arquitetura especificada considerou a existência de diversos domínios, facilidade e capacidade de integração com o produto;
    }
    \item
    {
        \textit{Prova de conceito dá resposta correta às questões que lhe são colocadas} -- que implica responder corretamente às questões-chave, estabelecidas na metodologia de avaliação, garantindo que a resposta fornecida é semelhante ou igual à resposta que seria esperada;
    }
    \item
    {
        \textit{O modelo é adotado ou refinado de forma a que possa ser usado na solução final} -- o modelo arquitetado revela-se efetivo na resolução do problema, e com o levantamento de possíveis melhorias, pode ser implementado no {\productname}.
    }
    % \item 
    % {
    %     \textit{Tese escrita} -- na qual se abordam o problema, o contexto no qual se insere e o valor que traz ao produto final. Deve conter o estado da arte, apresentando a revisão da literatura existente, focando nas soluções semelhantes e/ou ferramentas relevantes que perspetivam estratégias de solução para o problema. Por fim, descreve-se a solução proposta, contemplando cada uma das fases inerentes ao seu desenvolvimento (visão, análise, desenho e implementação) e faz-se a conclusão acerca do trabalho (todos os objetivos descritos em~\ref{sec:chap01_objectives});
    % }
\end{enumerate}

%%%%%%%%%%%%%%%%%%%%%%%%%%%%%%%%%
%           SECTION
%%%%%%%%%%%%%%%%%%%%%%%%%%%%%%%%%
\section{Contribuições}
\label{sec:chap01_contributions}
O trabalho a ser desenvolvido pretende providenciar uma abordagem de resolução do problema apresentado. Não se aspira fornecer uma solução definitiva, espera-se sim, contribuir com conhecimento de carácter teórico e prático (\idest{um protótipo}), que possibilite a integração futura de uma nova funcionalidade num produto já existente, trazendo-lhe mais-valia funcional, destacando-o dos seus concorrentes. No decorrer deste trabalho serão abordados temas relativos a \gls{mes}, a \gls{ia}, especificamente \gls{pln}, e \textit{Business Intelligence}.

Resumidamente, as contribuições esperadas são a seguir enunciadas:

\begin{itemize}
    \item
    {
        Estado da arte no domínio de Processamento de Linguagem Natural e sistemas análogos à solução a desenvolver;
    }
    \item 
    {
        Documentação dos requisitos de sistema, incluindo análise e desenho, constando os respetivos artefactos de \gls{uml};
    }
    \item 
    {
        Identificação de abordagens para lidar com semântica de diferentes domínios;
    }
    \item 
    {
        Definição de um modelo que possibilita a \inquotes{conversão} da linguagem natural em informação pertinente para o utilizador do sistema; 
    }
    \item
    {
        Especificação e desenvolvimento do protótipo, considerando a futura integração com o sistema {\productname}.
    }
\end{itemize}

De uma forma geral, é realçada a contribuição para o avanço do conhecimento no domínio da \gls{ia}, mais especificamente na área da \gls{pln}, aplicada ao contexto dos sistemas \gls{mes} e que servirá como base para a integração duma solução deste tipo no {\productname}.

%%%%%%%%%%%%%%%%%%%%%%%%%%%%%%%%%
%           SECTION
%%%%%%%%%%%%%%%%%%%%%%%%%%%%%%%%%
\section{Plano e Método de Trabalho}
\label{sec:chap01_workmethodology}
Um plano de trabalho é uma ferramenta importante na gestão de qualquer projeto, na medida em que descreve as fases que o compõem e as tarefas inerentes. Ele está sujeito a alterações ao longo do tempo de vida do projeto, pelo que o plano definido inicialmente pode ser encontrado no Apêndice~\ref{AppendixA}. Relativamente às fases do trabalho, a informação sucinta de cada uma é aqui explanada:

\begin{itemize}
    \item
    {
        \textit{Conceção} -- engloba as tarefas que relacionadas com o problema, o seu enquadramento, o estudo do valor e estado da arte. Ou seja, a visão do projeto;
    }
    \item
    {
        \textit{Análise} -- nesta fase, faz-se um estudo exploratório e de caráter empírico de forma a experimentar a aplicação de diferentes abordagens. Definem-se os casos de uso e arquitetura considerada para o modelo. A presente fase consiste no estudo e preparação para a aplicação do modelo no contexto prático; 
    }
    \item
    {
        \textit{Desenvolvimento} -- desenvolve-se o protótipo com base no modelo conjeturado na fase anterior, envolvendo um período de experimentação;
    }
    \item
    {
        \textit{Validação} -- avalia-se a solução com base nos critérios de sucesso definidos e consequentemente, podem-se registar melhorias e retirar as devidas conclusões;
    }
    \item
    {
        \textit{Documentação} -- engloba a escrita da tese como veículo de transmissão de conhecimento obtido e de outros documentos de suporte, a serem usados no contexto empresarial (\exempligratia{documento de especificação de requisitos de \textit{software}}).
    }
\end{itemize}

Quanto ao método de trabalho a seguir na realização deste trabalho, são considerados os seguintes passos:

\begin{enumerate}
    \item 
    {
        \textit{Revisão de literatura disponível sobre o contexto do problema} -- com o objetivo de perceber o estado atual do {\productname} e as implicações que o módulo trará, assim como concluir acerca da relevância do problema e do valor da solução para o produto;
    }
    \item
    {
        \textit{Revisão de literatura existente acerca de \gls{pln}, paradigmas arquiteturais relacionados e trabalhos de referência} -- adquirir conhecimentos sobre o estado \gls{pln}, quais os trabalhos de referência na área, outros também relevantes, técnicas e ferramentas usadas, identificando aspetos relevantes para o trabalho;
    }
    \item
    {
        \textit{Idealização do modelo} -- depois da análise dos conhecimentos adquiridos com as revisões realizadas nos passos descritos anteriormente,
        parte-se para a experimentação de diversas abordagens, definição dos casos de uso e arquitetura prevista;
    }
    \item
    {
        \textit{Implementação do protótipo e validação} -- o foco é pôr em prática a solução conceptualizada nas fases anteriores. Após a implementação, valida-se a mesma, com vista a perceber se os resultados obtidos são os esperados e proceder ao registo das melhorias que forem evidentes;
    }
    \item
    {
        \textit{Elaboração da documentação} --  por fim, passa-se à escrita do presente documento e de documentos de suporte, baseando-se nas observações, nas experiências e conclusões obtidas ao longo do projeto.
    }
\end{enumerate}
\chapter{Contexto}
\label{chap:Chapter2}
A contextualização do problema é necessária na medida em que contribui para a sua compreensão e resposta. Neste capítulo apresenta-se a empresa que visa ter uma solução para o problema exposto, fazendo uma breve descrição do seu negócio e do seu produto. De seguida, explicita-se a sua relevância para Indústria 4.0 e para o {\productname}, procurando perceber as motivações que levam à necessidade de resolução, estabelecendo-se o caminho para o estudo de problemas da mesma natureza.

%%%%%%%%%%%%%%%%%%%%%%%%%%%%%%%%%
%           SECTION
%%%%%%%%%%%%%%%%%%%%%%%%%%%%%%%%%
\section{A Empresa}
\label{sec:chap02_company}
A {\companyname} é uma empresa fundada em 2009, com sede e centro de engenharia na Maia (Porto, Portugal), subsidiárias em Dresden (Alemanha), Suzhou (China), Austin (Estados Unidos da América) e um escritório comercial em Taiwan. O seu objetivo é proporcionar à indústria uma solução de gestão e controlo de produção, procurando reduzir os custos, flexibilizar o processo para satisfazer a procura e capacitar a organização de uma maior agilidade, visibilidade e fiabilidade~\parencite{cmf_overview}. O compromisso da empresa foca-se no desenvolvimento de~\inquotes{soluções de vanguarda, indo de encontro aos desafios mais importantes da indústria e disponibilizar à lista crescente de clientes satisfeitos, soluções de elevado valor acrescentado, no prazo e orçamento requerido}\footnote{Tradução livre do autor. No original~\inquotes{[...] solutions that address the most urgent industry challenges and provide our growing list of satisfied customers with the highest value solution, on-time and on-budget.}.}~\parencite{cmf_overview}.

A estratégia da empresa está sintetizada na sua missão, visão e valores. Se a missão descreve a razão da empresa existir, ou seja, o seu propósito, já a visão retrata o que se aspira alcançar~\parencite[pp.~65-66]{mission_vision_values_what_do_they_say}. Isto posto, a missão e visão são divulgados a seguir~\parencite{cmf_strategy}: 

\begin{itemize}
    \item 
    {
        \textit{Missão} -- \inquotes{Trazer valor através da convergência de inteligência, operações e tecnologias de automação para a Indústria 4.0.}\footnote{Tradução livre do autor. No original \inquotes{We drive business value through the convergence of intelligence, operations, and automation technologies for Industry 4.0.}.}.
    }
    \item 
    {
        \textit{Visão} -- \inquotes{Tornar a Indústria 4.0 uma realidade para todos fabricantes.}\footnote{Tradução livre do autor. No original \inquotes{We will make Industry 4.0 a reality for all manufacturers.}.}.
    }
\end{itemize}

Relativamente aos valores, são estes que suportam a visão, moldam a cultura empresarial e são a essência da sua identidade. Como tal, de seguida apresentam-se os valores da {\companyname}~\parencite{cmf_strategy}:

\begin{itemize}
    \item 
    {
        \textit{Inovação} -- \inquotes{Exceder as expectativas dos clientes através das soluções mais eficientes e de mais alto valor para indústria.}\footnote{Tradução livre do autor. No original \inquotes{We constantly exceed our customers’ expectations through the most efficient and high value-added manufacturing solutions.}.}.
    }
    \item
    {
        \textit{Agilidade} -- \inquotes{Adaptar as pessoas, processos e soluções de forma a responder à evolução do mundo da manufatura de alta tecnologia.}\footnote{Tradução livre do autor. No original \inquotes{We continuously adapt our people, processes and solutions to respond to the evolving world of high-tech manufacturing.}.}.
    }
    \item
    {
        \textit{Compromisso} -- \inquotes{Defender o sucesso contínuo dos clientes e da empresa.}\footnote{Tradução livre do autor. No original \inquotes{We champion the continued success of our customers and our company.}.}.
    }
\end{itemize}

Resumidamente, a estratégia da empresa foca a melhoria contínua dos processos de fabrico dos clientes, o que possibilita melhor qualidade para o produto final, procurando estar na vanguarda da tecnologia e inovação.

%%%%%%%%%%%%%%%%%%%%%%%%%%%%%%%%%
%           SECTION
%%%%%%%%%%%%%%%%%%%%%%%%%%%%%%%%%
\section{O Produto}
\label{sec:chap02_product}
Nos últimos anos, o mercado dos sistemas de informação empresariais tem vindo a crescer, sobretudo pela necessidade das empresas aumentarem a sua produtividade e consequentemente, melhorarem a sua competitividade~\parencite{mes_literature_review}. \textcite{mes_literature_review} afirma que, embora sistemas \gls{erp} sejam cada vez mais usuais nas empresas, no sentido de gerir as suas operações, estes falham quando aplicados num contexto fabril, ou seja, no \inquotes{chão de fábrica}. Os departamentos produtivos beneficiam de \textit{software} personalizado, que responda às necessidades específicas do foro produtivo/industrial~\parencite{mes_literature_review}.

Nestas circunstâncias surge o conceito de \gls{mes}, fruto da necessidade das empresas de manufatura progredirem no mercado, num ponto de vista de melhoria da reatividade, da qualidade, dos custo de produção e dos prazos de entrega~\parencite {mes_explained_high_level_vision}. Desse modo, as funções de um \gls{mes} estão sobretudo ligadas a atividades de manufatura, que representa uma parte substancial do valor acrescentado em empresas deste setor~\parencite{mes_literature_review}. 

Com o objetivo de apresentar o produto, nesta secção faz-se um enquadramento genérico do conceito \gls{mes} e posteriormente, foca-se o caso específico do {\productname}.

\subsection{\textit{Manufacturing Execution Systems}}
A organização MESA\footnote{\textit{Manufacturing Enterprise Solutions Association}. Disponível em \url{http://www.mesa.org}}, uma comunidade mundial, sem fins lucrativos, que junta empresas de manufatura, de prestação de serviços, analistas, académicos e estudantes, com o propósito de melhorar os resultados do negócio e as operações de produção, através da implementação e implantação de tecnologias de informação e das melhores práticas de gestão, deu o primeiro passo na definição formal de \gls{mes}~\parencite{mes_explained_high_level_vision}:

\begin{quote}
    \inquotes{\textit{Os Manufacturing Execution Systems (MES) fornecem informações que possibilitam a otimização de atividades de produção, desde o lançamento do pedido até aos produtos acabados. Usando dados atualizados e precisos, o MES orienta, inicia, responde e relata as atividades da fábrica à medida que elas ocorrem. A resposta rápida, resultante das mudanças nas condições, associada ao foco na redução de atividades sem valor acrescentado, impulsiona a eficácia das operações e processos fabris. O MES melhora o retorno dos ativos operacionais, bem como o prazo de entrega, gestão de stock, margem bruta e desempenho do fluxo de caixa. O MES fornece informações críticas acerca das atividades de produção em toda a empresa e cadeia logística através de comunicações bidirecionais.}}\footnote{Tradução livre do autor. No original \inquotes{Manufacturing Execution Systems (MES) deliver information that enables the optimization of production activities from order launch to finished goods. Using current and accurate data, MES guides, initiates, responds to, and reports on plant activities as they occur. The resulting rapid response to changing conditions, coupled with a focus on reducing non value-added activities, drives effective plant operations and processes. MES improves the return on operational assets as well as on-time delivery, inventory turns, gross margin, and cash flow performance. MES provides mission-critical information about production activities across the enterprise and supply chain via bi-directional communications.}.}.
\end{quote}

Portanto, o \gls{mes} age como um intermediário entre os diversos processos existentes no \inquotes{chão de fábrica} e os sistemas de \inquotes{alto nível}, existindo comunicação bidirecional entre as camadas, como se demonstra na Figura~\ref{fig:mes_layers}. O \gls{mes} tanto pode fornecer informação acerca dos custos de produção, de indicadores de \textit{performance}, do estado das ordens de fabrico ou rendimento produtivo, como pode também obter dados sobre o planeamento das atividades fabris, parâmetros operacionais, receitas ou instruções de fabrico, por forma a inferir de forma inteligente sobre a fábrica e os seus processos~\parencite{mes_explained_high_level_vision}. Esta bidirecionalidade na comunicação e abrangência no processo produtivo faz com que o \gls{mes} tenha um papel crucial na Indústria 4.0, já que pode acomodar a integração, descentralização e novas tecnologias, ainda que nem todos os sistemas deste tipo tenham sido desenhados dessa forma~\parencite{cmf_mes_definition}.
%
\begin{figure}
    \centering
    \includegraphics[width=.8\textwidth]{ch02/assets/mes_layers.jpg}
    \caption{Ambiente \glsfirst{mes} e as suas camadas, baseado em~\textcite[p.~526]{mes_literature_review}}
    \label{fig:mes_layers}
\end{figure}

Com o intuito de dar resposta às necessidades de diversos ambientes produtivos, as funções apresentadas na Figura~\ref{fig:mes_functions} e discriminadas a seguir são essenciais para um \gls{mes}, nomeadamente no suporte, no controlo e na rastreabilidade de cada atividade produtiva~\parencite{mes_literature_review, mes_explained_high_level_vision, introduction_mes}:

\begin{enumerate}
    \item 
    {
        \textit{Operações/Agendamento de detalhes} -- sequenciamento e distribuição temporal das atividades fabris, por forma a otimizar a \textit{performance}, com base nos recursos disponíveis;
    }
    \item
    {
        \textit{Gestão do processo} -- controlo do fluxo de trabalho, baseado nas atividades produtivas reais e planeadas;
    }
    \item
    {
        \textit{Controlo documental} -- gestão e distribuição de informação relativa a produtos, processos, ordens de fabrico, assim como recolher os certificados e condições de trabalho;
    }
    \item
    {
        \textit{Aquisição de dados} -- monitorização, recolha e tratamento de dados sobre os processos, os materiais e operações, por pessoas, máquinas ou controlos;
    }
    \item
    {
        \textit{Gestão laboral} -- supervisão no uso de pessoal de operações num determinado turno, com base nas qualificações, padrões de trabalho e nas necessidades de negócio;
    }
    \item
    {
        \textit{Gestão da qualidade} -- registo e análise das características do produto e do processo face aos requisitos ideais;
    }
    \item
    {
        \textit{Expedição de unidades de produção} -- dar a ordem para envio de materiais ou ordens para certos setores da fábrica, com o intuito de iniciar um processo ou sub-processo;
    }
    \item
    {
        \textit{Gestão de manutenção} -- planeamento e execução de tarefas que visam manter o equipamento e outros ativos capazes de executar a sua tarefa, de forma eficaz;
    }
    \item
    {
        \textit{Genealogia e rastreabilidade do produto} -- monitorização do progresso das unidades, amostras ou lotes de saída, para a criação de histórico completo do produto;
    }
    \item
    {
        \textit{Análise de desempenho} -- comparação dos resultados medidos com os objetivos e métricas definidas pela corporação, pelos clientes ou órgãos reguladores;
    }
    \item
    {
        \textit{Estado e alocação de recursos} -- orientação sobre o que as pessoas, máquinas ou ferramentas devem fazer, acompanhando que já fizeram e o que estão a fazer no momento.
    }
\end{enumerate}
%
\begin{figure}
    \centering
    \includegraphics[width=\textwidth]{ch02/assets/mes_functions.jpg}
    \caption{Funções do \glsfirst{mes} e o seu enquadramento, extraído de~\textcite{mes_literature_review}}
    \label{fig:mes_functions}
\end{figure}

As funções do \gls{mes} enunciadas servem como base para praticamente qualquer fábrica, fornecendo ferramentas a gestores de fábrica, departamentos de qualidade e manutenção, estando interrelacionadas. Por isso, tratam-se de funções críticas para a maioria dos fabricantes, na medida em que a exigência de processos novos e mais rigorosos no negócio é viável, possibilitando o sucesso no mercado~\parencite{,mes_explained_high_level_vision}. Logo, torna-se evidente que o \gls{mes} traz benefícios para as corporações, alguns alcançáveis num período curto de tempo -- aumento de eficiência e redução de custos; redução no tempo de execução de ordens de fabrico; redução dos custos associados ao trabalho; diminuição ou eliminação de papelada; redução da quantidade de material em processamento; utilização de máquinas mais eficaz --, enquanto que outros, possíveis a longo prazo -- melhoria geral dos processos; maior satisfação do cliente; melhoria na conformidade regulamentar; maior agilidade; melhoria nos prazos de entrega; maior visibilidade da cadeia logística~\parencite{cmf_mes_definition}.

\subsection{{\productname}}
O {\productname} afirma-se como o futuro do \gls{mes}. Trata-se de uma plataforma de \textit{software} com um vasto conjunto modular de aplicações e ferramentas, que dotam os utilizadores de indústrias complexas de agilidade, visibilidade e fiabilidade. O produto adapta-se a diversos processos fabris e às suas operações, sendo fácil a sua implantação, independentemente da infraestrutura existente, permitindo o controlo de produção e de custos na empresa e cadeia logística, resultando nas seguintes vantagens~\parencite{cmf_product_overview}:

\begin{enumerate}
    \item 
    {
        Apresenta um \textbf{baixo custo total de posse}\footnote{\textit{Total Cost of Ownership} (TCO). É uma estimativa financeira usada para avaliar os custos diretos e indiretos associados a uma compra.}, visto que a empresa reduz as despesas associadas à implantação, operação e manutenção do sistema;
    }
    \item
    {
        Fornece um largo conjunto de capacidades que dão resposta aos mais variados requisitos, demonstrando a sua \textbf{cobertura funcional};
    }
    \item
    {
        \textbf{Capacita o utilizador na sua função}, sendo este capaz de desenhar e colocar em produtivo o plano da fábrica, rastreando os materiais e os detalhes do processo;
    }
    \item
    {
        É sistema modular, o que o torna \textbf{extensível, flexível e escalável}, dando aos seus utilizadores acesso a inteligência operacional, de forma fácil e rápida; 
    }
    \item
    {
        A arquitetura logicamente descentralizada, ligada à conectividade a diferentes protocolos para equipamentos e dispositivos e ao suporte de produtos preparados para \gls{iot} e \glspl{cps}, tornam o produto \textbf{preparado para a Indústria 4.0}.
    }
\end{enumerate}

Relativamente à arquitetura do produto (ver Figura~\ref{fig:mes_framework}), a {\companyname} baseou-se nas tecnologias mais
recentes para dotar a sua plataforma da capacidade de adaptação aos diversos ambientes produtivos. Posto isto, a infraestrutura consiste em três camadas, que além de fornecerem particionamento, modularidade e escalabilidade das aplicações, foram projetadas para funcionar em conjunto. Além disso, esta foi desenhada de forma a ser customizável e extensível, dado que cada cliente pode ter os seus próprios requisitos~\parencite{cmf_mes_framework}.
%
\begin{figure}
    \centering
    \includegraphics[width=\textwidth]{ch02/assets/mes_framework.png}
    \caption{Arquitetura do {\productname} e tecnologias usadas, retirado de~\textcite{cmf_mes_framework}}
    \label{fig:mes_framework}
\end{figure}
%
Quanto às especificidades de cada camada, são descritas de seguida, numa perspetiva de entender a responsabilidade de cada uma delas e o seu contributo para a plataforma~\parencite{cmf_mes_framework}.

\begin{itemize}
    \item 
    {
        \textit{Camada de Apresentação (Presentation Tier)} -- projetada para trazer ao utilizador uma experiência rica e interativa. Dispõe de várias capacidades (\exempligratia{permitir aos utilizadores criar a sua própria interface gráfica ou desenvolver ecrãs para um propósito em particular, numa fábrica ou setor}) e é desenvolvida com suporte multi-plataforma, executando em qualquer sistema operativo \textit{desktop} ou móvel;
    }
    \item
    {
        \textit{Camada de Negócio (Business Tier)} -- implementa e expõe todas as funcionalidades como serviços, estando disponíveis vários protocolos de comunicação. Contém uma sub-camada de orquestração usada para definir os diversos fluxos de negócio, fornecendo a capacidade de coordenação usando os objetos de negócio. Por fim, a sub-camada de objetos de negócio segue um modelo hierárquico, o qual facilita o desenvolvimento de entidades com um comportamento comum;
    }
    \item
    {
        \textit{Camada de Dados (Data Tier)} -- desenhado para suportar as capacidades de armazenamento de dados, possibilitando a integração com fontes de dados externas, geração e modificação de relatórios e mineração de dados.
    }
\end{itemize}

O {\productname} é usado em diversas indústrias, particularmente a indústria de semicondutores~\parencite{cmf_industries_semiconductor}, de equipamentos médicos~\parencite{cmf_industries_medical_devices}, de montagem eletrónica~\parencite{cmf_industries_electronics}, procurando dar resposta aos desafios inerentes a cada uma delas.

%%%%%%%%%%%%%%%%%%%%%%%%%%%%%%%%%
%           SECTION
%%%%%%%%%%%%%%%%%%%%%%%%%%%%%%%%%
\section{Relevância do Problema}
\label{sec:chap02_relevance}
Quando um problema surge, é preciso perceber a razão que o torna efetivamente um problema. Ou seja, é necessário explicar a importância de encontrar uma resposta. Nesta secção aborda-se esse assunto, na perspetiva da Indústria 4.0 e do produto {\productname}.

\subsection{Na Indústria 4.0}
A Indústria 4.0, ou quarta revolução industrial, introduz uma necessidade de vanguarda da tecnologia, abrindo caminho para a mecanização, automatização e digitalização dos processos fabris~\parencite{industry40}. Do vários conceitos identificados como pilares da revolução, tais como a descentralização ou a auto-organização, destaca-se a capacidade de adaptação ao ser humano~\parencite{intelligent_manufacturing_context_industry40_review, industry40}. Assim, cimenta-se a visão de fábricas inteligentes, em que os operadores (operadores de linha, engenheiros e gestores) interagem com um ambiente complexo, rico em dados, intuitivamente recolhendo a informação, fazendo com que a \gls{ihc} seja uma necessidade~\parencite{social_factory, hmi_industry40}.

Conforme descrito em~\textcite{hmi_industry40}, as interfaces gráficas representam o elemento intermediário entre o utilizador e o sistema subjacente, havendo a necessidade de serem móveis, sensíveis ao contexto em que se enquadram, e que permitam a filtragem da informação, providenciando apenas as possibilidades relevantes para o problema. Por isso, a integração de reconhecimento de voz e texto (Inteligência Artificial) e reconhecimento gestual (Realidade Aumentada) nas interfaces industriais constitui uma estratégia para potenciar a simbiose entre humano e sistema. A visão passa por, não só aproveitar as capacidades das maquinas, mas também por habilitar os trabalhadores de novas capacidades e engenhos, que lhes possibilite tomar melhores decisões, de forma mais rápida~\parencite{towards_operator_40_typology}. O trabalho~\textcite{towards_operator_40_typology} refere ainda a existência de diferentes interações do operador com partes do sistema, que seriam desejáveis, das quais se destacam as que dizem respeito à interação cognitiva:

\begin{itemize}
    \item
    {
        \textit{Operador e Assistente Inteligente Pessoal} -- um assistente inteligente é um agente de \textit{sofware}, dotado de \gls{ia}, que foi desenvolvido com o objetivo de ajudar o operador a comunicar com máquinas, computadores, bases de dados e outros de sistemas de informação, assim como, assistir nas tarefas de gestão de tempo e na execução de tarefas, através duma interação humanizada. Alguns cenários de uso onde esta interação constitui uma enorme vantagem são na procura e aquisição de informação armazenada, na leitura de instruções de uso de um determinado recurso, na gestão de inventário e no planeamento de atividades;
    }
    \item
    {
        \textit{Operador e Análise de Big Data} -- a análise de \textit{Big Data} corresponde ao processo de coletar, organizar e examinar grandes conjuntos de dados, com o objetivo de obter informação relevante e prever eventos cruciais. Ajuda os operadores (\exempligratia{engenheiros de produção}) a atingir melhores previsões, fornecer visibilidade sobre os indicadores de desempenho do processo, diminuindo o tempo de reação aquando a ocorrência de problemas, levando a decisões mais rápidas e incisivas. Está conectado com outras partes do sistema, como o assistente virtual mencionado no ponto anterior.
    }
\end{itemize}

Portanto, a junção destes dois mundos é desejável, pelas vantagens que traz para o operador na sua relação com o sistema, que inclui o seu empoderamento, a melhoria da capacidade de decisão e no desempenho do processo, o que potencia o valor do produto. Este trabalho, cujo problema representa uma ínfima parte duma problemática maior, equaciona a coligação entre ser humano e máquina, possibilitando o avanço da tecnologia usada na Indústria 4.0.

\subsection{No {\productname}}
O {\productname}, como o próprio nome indica, é um \gls{mes}, que tal como foi abordado anteriormente (ver Secção \ref{sec:chap02_product}), tem um papel crucial na Indústria 4.0. Por essa razão, é esperado que também herde as problemáticas levantadas no contexto desta revolução industrial. De acordo com~\textcite{industry40_revolution_future_mes}, no futuro, os \glspl{mes} devem estar preparados para lidar com aspetos fundamentais da Indústria 4.0, como a descentralização, a integração vertical de sistemas, a conectividade, mobilidade, usabilidade, capacidade analítica e integração na \textit{cloud}. 

Este produto é usado em ambiente produtivo, por operadores de linha e engenheiros/gestores de produção, e fornece uma vista holística sobre a fábrica, permitindo desempenhar diversas funções, tais como a gestão de tarefas/instruções de fabrico, gestão do estado das ordens de fabrico, visualização e geração de relatórios dos custos de produção e \textit{performance} do processo. Para esta última, o {\productname} oferece acesso à informação através da sua interface gráfica (ver Figura~\ref{fig:mes_view}), possibilitando a exportação dos dados para ferramentas como o \textit{Excel} ou \textit{Power BI} para uma análise mais detalhada~\parencite{cmf_services_bi}. 
%
\begin{figure}
    \centering
    \includegraphics[width=\textwidth]{ch02/assets/mes-view.png}
    \caption{Interface gráfica do {\productname} para acesso a informação do processo de fabrico}
    \label{fig:mes_view}
\end{figure}
%
Contudo, este tipo de abordagem é inadequada para alguns casos, dado a dinâmica do processo de fabrico, a qual  obriga a pesquisa de informação rápida, para a toma de decisões expedita. Por outro lado, a solução atual exige o compreensão da forma como dados se encontram armazenados, ainda que não detalhadamente. Por exemplo, perguntas como \inquotes{Qual o índice de desempenho, por máquina, do setor $S$?} ou \inquotes{Quantas peças $P$ foram produzidas, durante o mês de Janeiro, na máquina $M1$?} podem constituir um problema de aquisição de informação. Por um lado, o responsável pela aquisição desse conteúdo vai necessitar de conhecer minimamente o modelo de dados do sistema para que consiga extrair esse conhecimento. Por outro, a dificuldade de adquirir essa informação rapidamente, que além de abrandar o processo de apoio à decisão, não promove a interação natural entre homem e produto.

A introdução de \gls{ia} sobre a forma de linguagem natural visa melhorar a usabilidade, aproximando o operador do sistema subjacente, facilitando o acesso à informação relevante e melhorando o processo de apoio à decisão. Para além disso, a inclusão deste tipo de tecnologia apresenta uma vantagem competitiva para o produto, que permite fortalecer a sua notoriedade, potenciar o seu volume de vendas e consolidar-se como parte do topo dos \glspl{mes} que rumam à quarta revolução industrial.

%%%%%%%%%%%%%%%%%%%%%%%%%%%%%%%%%
%           SECTION
%%%%%%%%%%%%%%%%%%%%%%%%%%%%%%%%%
\section{Síntese}
\label{sec:chap02_chaptersummary}
Neste capítulo tentou-se enquadrar o problema na realidade em que se insere. Para tal, apresentou-se a {\companyname}, a sua visão, missão e valores, os quais estão intrinsecamente ligados ao conceito da Indústria 4.0. 

Seguiu-se a introdução ao produto da {\companyname}, descrevendo o conceito de \gls{mes}, um sistema intermediário entre os processos existentes em contexto fabril, as funções que este alberga e quais os seus benefícios. Depois, focou-se no caso específico do produto, procurando expor as suas características e arquitetura.

Finalmente, fundamentou-se a importância do problema, quer para a Indústria 4.0, quer para o {\productname}, numa perspetiva de mostrar as razões que levam à necessidade de uma solução, que incluem a melhoria de usabilidade e do processo de apoio à decisão.
\chapter{Estado da Arte}
\label{chap:Chapter3}
O presente capítulo tem como objetivo apresentar os conceitos pertinentes para a execução e compreensão do trabalho, as ferramentas de \gls{pln} disponíveis no mercado, relevantes para a resolução do problema e os projetos que focam um problema de cariz semelhante. Na Secção~\ref{sec:chap03_pln} é introduzido o conceito de Processamento de Linguagem Natural, enquadrando-o com a solução a desenvolver. A Secção~\ref{sec:chap03_marketstudy} explora-se o mercado numa perspetiva de encontrar casos de estudo para a solução a desenvolver. Por fim, a Secção~\ref{sec:chap03_existingtools} descrimina as ferramentas mais significativas para a área \gls{pln} e em que medida é que cada pode, ou não, contribuir para o trabalho.

% Introdução ao PLN
\section{Processamento de Linguagem Natural}
\label{sec:chap03_pln}
O \glsfirst{pln} é um campo da Ciência da Computação, \glsfirst{ia} e Linguística que explora a forma como os computadores podem ser usados na compreensão, manipulação e geração automática da linguagem natural, em forma de texto ou voz~\parencite{nlp, applied_natural_language_processing_with_python, pln_extracao_conhecimento}. Nos últimos anos, a área tem-se tornado bastante popular com o acesso fácil a informação através da Internet, estando presente em implementações de \textit{chatbots}, verificadores ortográficos em telemóveis e assistentes de \gls{ia} nos \textit{smartphones}, tais como a Cortana\footnote{Disponível em \url{https://www.microsoft.com/en-us/cortana}.} ou Siri\footnote{Disponível em \url{https://www.apple.com/siri/}.}~\parencite{pln_extracao_conhecimento, applied_natural_language_processing_with_python}. 

\subsection{História}
O início do \gls{pln} remonta aos anos 40, com o desenvolvimento da Ciência da Computação, aliada aos avanços na Linguística, que levou ao aparecimento da teoria da linguagens formais. Muito sucintamente, esta teoria consiste na modelação de estruturas complexas e respetivas regras, ou seja, permitem especificar e reconhecer linguagens a partir de modelos matemáticos (\exempligratia{um alfabeto é uma estrutura simples, a qual é constituída por letras que podem formar palavras, em diferentes idiomas}). Por outro lado, os avanços na \gls{ia}, particularmente com o modelo \gls{slp} apresentado na Figura~\ref{fig:slp}, também contribuíram para este campo~\parencite{applied_natural_language_processing_with_python}. O \gls{slp} é a base dos modelos neuronais usados nos dias de hoje. Warren McCulloch e Walter Pitts propuseram este modelo, baseado na analogia entre células nervosas (neurónios) e os processos computacionais, que permite computar a soma dos pesos ($w_{ni}$) associados a cada entrada ($x_{n}$), dando uma resposta binária consoante o valor da soma ($\sum$) varia de acordo com um determinado valor limite, decidindo se uma determinada ação será executada~\parencite{introduction_theory_neural_computation}.

\begin{figure}[!t]
    \centering
    \includegraphics[width=.8\textwidth]{ch03/assets/slp_model.jpg}
    \caption{\glsfirst{slp}, extraído de~\textcite{applied_natural_language_processing_with_python}}
    \label{fig:slp}
\end{figure}

Ao longo dos anos, várias técnicas foram surgindo na tentativa de resolver problemas associados à compreensão da linguagem natural. Mas, nos últimos 20 anos, notou-se o aumento de interesse no \gls{pln}, juntamente com \gls{ml}, sobretudo devido ao aumento do poder computacional e à facilidade de acesso a dados etiquetados através da Internet~\parencite{applied_natural_language_processing_with_python}.

\subsection{Fundamentos}
O cerne de qualquer tarefa de \gls{pln} está relacionado com a compreensão da própria linguagem. O desenvolvimento deste tipo de aplicações incorre em alguns problemas tais como a processo de pensamento, a representação e significado linguístico e/ou conhecimento do domínio, que estão associados à ambiguidade da linguagem natural~\parencite{nlp, pln_extracao_conhecimento}. Por outras palavras, a ambiguidade surge quando não é possível atribuir um significado único a uma dada expressão. Se uma pessoa é capaz de o fazer, baseada na sua experiência, capacidade de interpretação de contexto ou na sua cultura, já um computador não tem essa mesma capacidade~\parencite{pln_extracao_conhecimento}. Por isso, \textcite{nlp} enuncia que, para ser capaz de compreender a linguagem natural, é importante considerar os vários níveis de conhecimento interdependentes que o ser humano usa na extração de significado:

\begin{itemize}
    \item 
    {
        \textit{Nível fonético ou fonológico} -- encarrega-se a pronúncia;
    }
    \item
    {
        \textit{Nível morfológico} -- lida com os \textit{tokens}, ou seja, partes nucleares de um frase (\exempligratia{palavras, sufixos, prefixos, sinais de pontuação, dígitos, entre outros});
    }
    \item
    {
        \textit{Nível léxico} -- trata do significado léxico dos símbolos e análise de partes do discurso;
    }
    \item 
    {
        \textit{Nível sintático} -- lida com a gramática e a estrutura frásica;
    }
    \item
    {
        \textit{Nível semântico} -- encarrega-se de clarificar o significado da frase ou das palavras;
    }
    \item
    {
        \textit{Nível pragmático} -- ocupa-se da relação entre a linguagem e o contexto, ou seja, contempla a relação com o mundo exterior;
    }
    \item
    {
        \textit{Nível de discurso} -- suporta o agrupamento de diferentes frases, identificando a relação entre elas, de forma a compreender o contexto.
    }
\end{itemize}

Um sistema \gls{pln} pode envolver todos ou alguns destes níveis, cujas atividades permitem solucionar pequenas partes de um problema mais complexo. Algumas destas tarefas ou ferramentas incluem técnicas de segmentação de palavras e construção frásica, \textit{parsing} sintático e estatístico, métodos de desenho de modelos de conhecimento estruturados, redes neuronais e modelos de linguagem neuronal~\parencite{nlp, speech_language_processing}.

\subsection{Linguagem Natural para Bases de Dados}
Uma interface de linguagem natural é um componente que aceita expressões de consulta ou comandos em linguagem natural e providencia as respostas apropriadas, ou seja, esta deve ser capaz de traduzir as frases nas respetivas ações para o sistema~\parencite{nlp}. Neste contexto particular, importa explorar as \gls{ilnbd}, as quais permitem os utilizadores executarem pesquisas em bases de dados usando a linguagem natural~\parencite{overview_nlidb_approaches_implementation_airline, novel_approach_building_generic_portable_contextual_nlidb_system}.

As \gls{ilnbd} apresentam um problema clássico na área de \gls{pln} e constitui um campo de estudo em desenvolvimento. Genericamente, a solução inerente a este problema podem ser divida em duas fases: processamento linguístico, em que a frase de pesquisa é mapeada e traduzida para a \textit{query} de \gls{sql} correspondente, usando funções de mapeamento adequadas; processamento na base de dados, na qual é executado a gestão de acesso ao sistema e execução da respetiva consulta~\parencite{overview_nlidb_approaches_implementation_airline}. Este tipo de sistemas é capaz de responder a uma grande variedade de \textit{queries} de linguagem natural mas são pouco usados comercialmente, principalmente pela pouca robustez nas capacidades de processamento de contexto~\parencite{novel_approach_towards_incorporating_context_processing_nlidb}, pela falta de cobertura linguística ou pelo facto de utilizador poder assumir inteligência por parte do sistema~\parencite{survey_nlidb, overview_nlidb_approaches_implementation_airline}. Ainda assim, existem várias vantagens que contribuem para o desenvolvimento deste tipo de aplicações, nomeadamente a facilidade e simplicidade de utilização, o facto de ser mais adequado para questões que envolvem negação ou quantificação ou a sua tolerância a erros gramaticais~\parencite{survey_nlidb, nlidb_brief_review, overview_nlidb_approaches_implementation_airline}.

\begin{figure}[!h]
    \centering
    \includegraphics[width=.6\textwidth]{ch03/assets/non_contextual_nlidb.jpg}
    \caption{Sistema \glsfirst{ilnbd} não-contextual, extraído de~\textcite{novel_approach_towards_incorporating_context_processing_nlidb}}
    \label{fig:noncontextual_nlidb}
\end{figure}

Num ponto de vista processual, \textcite{novel_approach_towards_incorporating_context_processing_nlidb} menciona que os sistemas \gls{ilnbd} podem ser divididos em dois tipos: não-contextuais e contextuais. Num sistema não-contextual (Figura~\ref{fig:noncontextual_nlidb}) existe a necessidade de modelo semânticos de base, descrevendo regras de domínio. Na fase de análise sintática (\textit{Syntactic Analysis}) é extraída a informação linguística da \textit{query} de linguagem natural. Por sua vez, na análise semântica (\textit{Semantic Analysis}) identificam-se as entidades, atributos a partir da resposta da fase anterior e dos modelos semânticos. Finalmente, na fase de processamento de \textit{queries} (\textit{Query Processing}, as entidades identificadas são mapeadas num grafo, computando-se o caminho mais curto. Dessa forma, é gerada a \textit{query} \gls{sql} e executada para obter os resultados~\parencite{novel_approach_towards_incorporating_context_processing_nlidb}. Já um sistema \gls{ilnbd} contextual (Figura~\ref{fig:contextual_nlidb}) recolhe informação acerca do contexto numa \inquotes{conversa} com o utilizador. Neste caso, as capacidades de processamento devem ser contidas na arquitetura através da inserção de uma nova etapa (\textit{Context Processing}), mantendo intactas as responsabilidades de cada fase~\parencite{novel_approach_towards_incorporating_context_processing_nlidb}.

\begin{figure}[!h]
    \centering
    \includegraphics[width=.7\textwidth]{ch03/assets/contextual_nlidb.jpg}
    \caption{Sistema \glsfirst{ilnbd} contextual, extraído de~\textcite{novel_approach_towards_incorporating_context_processing_nlidb}}
    \label{fig:contextual_nlidb}
\end{figure}

Relativamente às abordagens ou estratégias de desenvolvimento deste tipo de \textit{software}, são várias e cada uma delas apresentam particularidades que podem influenciar a forma como o sistema é desenhado~\parencite{nlidb_brief_review, survey_nlidb}:

\begin{itemize}
    \item 
    {
        \textit{Abordagem simbólica (baseada em regras)} -- a linguagem é analisada e é aplicada lógica baseada em regras, de forma a capturar o significado da linguagem. O conhecimento encontra-se mapeado em regras ou noutras formas de representação;
    }
    \item
    {
        \textit{Abordagem empírica (baseada em experiências}) -- aplica análise estatística ou outro tipo de análises orientadas aos dados. Maior parte dos métodos de \gls{pln} aplicam técnicas estatísticas com modelos \textit{n-gram}, \textit{Hidden Markov} ou gramáticas de contexto livre;
    }
    \item
    {
        \textit{Abordagem de conexão (baseado em redes neuronais)} -- baseada em representações distribuídas que correspondem a regularidades estatísticas na linguagem. Uma vez que as capacidades da linguagem humana assentam na rede neuronal no cérebro, as redes neuronais artificiais apresentam um ponto fulcral na modelação do processamento de linguagem.
    }
\end{itemize}

Quanto à arquitetura dos sistemas \gls{ilnbd}, além de serem variadas, possuem também diferentes interpretações ao problema. Cada uma apresenta vantagens e desvantagens, pelo que é necessário explorar, de forma resumida, as características de cada.

\subsubsection{Sistemas \textit{Pattern Matching}}
Os primeiros esforços no desenvolvimento de sistemas deste género começaram em meados do século XX. O conceito de \textit{Pattern Matching} permite mapear diretamente o \textit{input} do utilizador para a obter o resultado desejado. A implementação destes sistemas implica que os detalhes da base de dados estejam presentes no código, ou seja, torna a solução limitada a um contexto específico e ao número e complexidade de padrões existentes~\parencite{nlidb_brief_review}. A principal vantagem desta abordagem prende-se à simplicidade de implementação, pelo que não há necessidade em conceber módulos de interpretação ou \textit{parsing} da linguagem~\parencite{nlidb_brief_review, survey_nlidb}.

\subsubsection{Sistemas Baseados em Sintaxe}
Os sistemas baseados em sintaxe possibilitam que a \inquotes{questão} do utilizador seja analisada sintaticamente, dando origem a uma árvore que é diretamente mapeada para uma expressão \gls{sql}. Para isso, estes sistemas usam uma gramática que descreve as estruturas sintáticas das perguntas dos utilizadores~\parencite{nlidb_brief_review}. Geralmente, é difícil mapear todas regras que constituem a gramática e o processo de escolha de quais as regras devem ser representadas é complexo. Outro problema é o facto de uma frase poder ter múltiplas corretas árvores de análise sintática, que aquando traduzidas, podem levar a diferentes resultados. Também a dificuldade de transformar a árvore de análise sintática diretamente numa linguagem genérica de base de dados é um problema complexo de resolver~\parencite{survey_nlidb}. A principal vantagem desta abordagem é o facto de fornecer informação acerca da estrutura frásica, possibilitando o mapeamento da semântica em regras produtivas (nós da árvore de análise sintática)~\parencite{nlidb_brief_review}.

\subsubsection{Sistemas de Gramática Semântica}
Apesar da sua semelhança com os sistemas baseados em sintaxe, a ideia inerente a um sistema deste tipo é a simplificar a árvore de análise sintática, através da combinação de alguns nós ou remoção dos mesmos. Posto isto, um sistema de gramática semântica é capaz de refletir melhor a representação semântica da frase, sem as estruturas complexas na árvore, com a possibilidade de designar nomes para os nós, reduzindo a ambiguidade. As principais desvantagens desta abordagem prendem-se com a necessidade de conhecimento prévio do domínio, tornando-se difícil a transposição para um outro e a estrutura específica das árvores de análise sintática não poderia ser usado noutra aplicação~\parencite{survey_nlidb, nlidb_brief_review}.

\subsubsection{Sistemas de Representação Intermediária de Linguagem}
Atualmente, os sistemas \gls{ilnbd} transformam a linguagem natural numa representação intermediária, definida internamente. Assim, a \textit{query} lógica representada na linguagem intermédia expressa o significado da questão colocada pelo utilizador em termos dos conceitos do domínio, os quais são independentes da estrutura da base de dados. Posteriormente, a \textit{query} lógica é traduzida na linguagem genérica de base de dados e avaliada. Esta arquitetura surgiu da dificuldade de traduzir diretamente a linguagem natural para a \gls{sql}, ou outra semelhante. O processo de transformação da \textit{query} lógica para a linguagem de base de dados pode conter várias fases, dependendo da necessidade do sistema~\parencite{nlidb_brief_review}.

% Casos de estudo
\section{Casos de Estudo}
\label{sec:chap03_marketstudy}
A investigação neste campo de estudo tem vindo a desenvolver-se desde o século XX~\parencite{survey_nlidb}. Assim sendo, é importante apresentar e examinar os casos mais pertinentes para o protótipo em desenvolvimento neste trabalho, na perspetiva de perceber quais as inovações que cada um deles trouxe para a área das \glspl{ilnbd} e em que medida se enquadram com o problema em resolução.

\subsection{LUNAR}
O LUNAR é um sistema que dá resposta ao domínio de amostras de rochas trazidas da lua e foi o primeiro sistema \gls{ilnbd}~\parencite{nlidb_brief_review, survey_nlidb}. O desenvolvimento deste sistema surgiu da necessidade de possibilitar aos cientistas envolvidos no estudo das rochas lunares poderem obter informação para formular e testar as suas hipóteses, de uma forma simples e intuitiva. O LUNAR permitia ao cientista executar diversas ações como fazer questões, computar médias e taxas, criar listas baseadas em critérios de seleção ou comparar medidas de diferentes investigadores, usando informação de duas bases de dados, uma contendo dados de análises químicas e a outra com dados de referências bibliográficas. Apesar de ter sido desenvolvido como protótipo, este sistema apresentou um desempenho satisfatório, sendo que cerca de 78\% dos pedidos foram respondidos com sucesso~\parencite{lunar_sciences_nlis}.

\subsection{LADDER}
O LADDER é um sistema desenhado para consultar informação sobre navios da Marinha Americana, por forma a auxiliar os gestores da Marinha no processo de tomada de decisão~\parencite{nlidb_brief_review, developing_nli_complex_data}. O sistema, que usa gramática semântica para tratar \textit{queries} a uma base de dados distribuída, apresenta uma arquitetura de três camadas, cada uma correspondente a um componente do sistema: o INLAND -- \textit{Infomal Natural Language Access to Navy Data} --, é responsável por aceitar a \textit{query} de linguagem natural, produzir a respetiva \textit{query} de base de dados a partir da decomposição da mesma em fragmentos, sendo posteriormente combinados para unidades sintáticas a alto nível, para que sejam reconhecidas, dando origem a um comando enviado para o próximo componente; o IDA -- \textit{Intelligent Data Access} --, compõe uma resposta com base no comando recebido e organiza a sequência correta de \textit{queries} a realizar; o FAM -- \textit{File Access Manager} --, o último componente, tem a responsabilidade de gerir o acesso à base de dados distribuída~\parencite{developing_nli_complex_data}.

\subsection{CHAT-80}
Segundo \textcite{nlidb_brief_review}, o CHAT-80 é um dos sistemas \gls{pln} mais referenciados nos anos 80. O CHAT-80 foi desenvolvido pensando na adaptabilidade a diversos domínios, de forma fácil e eficiente. Foi implementado em \textit{Prolog} e incluía uma base de conhecimento com factos geográficos de mais de 150 países (domínio de geografia mundial) e vocabulário inglês suficiente para interação com uma base de dados, que neste caso específico seria implementada totalmente em \textit{Prolog}. Os autores concordaram que a aplicação devia lidar com um conjunto restrito de linguagem natural relevante para o domínio, uma vez que dessa forma se torna uma linguagem de \textit{query} formal mas acessível para o utilizador~\parencite{efficient_easily_adaptable_system_interpreting_nlq}.

\subsection{JANUS}
O JANUS é uma aplicação \gls{pln} com a capacidade de \inquotes{comunicar} com múltiplos sistemas, tais como bases de dados, sistemas periciais, dispositivos gráficos, sendo capaz de avaliar a \textit{query} de linguagem natural e inferir acerca de quais os recursos a utilizar, sem que o utilizador se apercebesse da complexidade do sistema~\parencite{nlidb_brief_review, access_multiple_underlying_system_janus}. O fluxo do JANUS consistia em extrair as expressões da \textit{query} de linguagem natural, usando uma linguagem desenvolvida para o efeito, denominada \textit{World Model Language}; traduzir essas expressões para uma representação simplificada e normalizada; aplicar o algoritmo desenvolvido para encontrar a combinação adequada de serviços a disponibilizar, de modo a satisfazer o pedido do utilizador; por fim, a criação e execução de um plano para extração da informação~\parencite{access_multiple_underlying_system_janus}.

\subsection{PRECISE}
O PRECISE é um sistema desenvolvido na Universidade de Washington, cuja base de dados alvo é relacional, usando \gls{sql}, e que introduz o conceito de frases semanticamente tratáveis, ou seja, \textit{queries} que podem ser traduzidas para uma representação semântica única~\parencite{overview_nlidb_approaches_implementation_airline, nlidb_brief_review}.

\subsection{NALIX}
O NALIX -- \textit{Natural Language Interface for an XML Database} -- é uma \gls{ilnbd} desenvolvida na Universidade de Michigan, com o intuito de obter informação genérica a partir de uma base de dados em \gls{xml}~\parencite{nalix_interactive_nli_querying_xml}. De acordo com~\textcite{nalix_interactive_nli_querying_xml}, o desafio consiste em traduzir uma \textit{query} de linguagem natural para uma \textit{query} corretamente estruturada para uso numa base de dados, permitindo assim ao utilizador usar operações complexas (\exempligratia{agregação, combinação, junção, entre outras}). 

Relativamente à arquitetura do NALIX (Figura~\ref{fig:nalix_architecture}), o sistema consiste em duas partes: a primeira é responsável pela tradução da \textit{query} de linguagem natural para XQuery\footnote{Disponível em \url{https://www.w3schools.com/xml/xquery_intro.asp}.}, envolvendo os componentes \textit{Parse Tree Classifier}, \textit{Parse Tree Validator} e \textit{Parse Tree Translator}; a segunda suporta a formulação da \textit{query} de base de dados correspondente, usando os componentes \textit{Query Repository} e \textit{Message Generator}~\parencite{nalix_interactive_nli_querying_xml}.

\begin{figure}[!ht]
    \centering
    \includegraphics[width=.9\textwidth]{ch03/assets/nalix_architecture.jpg}
    \caption{Arquitetura do sistema NALIX, extraído de~\textcite{nalix_interactive_nli_querying_xml}}
    \label{fig:nalix_architecture}
\end{figure}

De salientar é que a linguagem de \textit{query} usada pelo NALIX (\textit{Schema Free XQuery}) não necessita que seja explicitado qual o \textit{schema} a ser usado, sendo que é capaz de encontrar automaticamente, para uma dada coleção de expressões/palavras-chave, todas as relações existentes entre estes elementos. Assim, é possível abstrair o sistema do domínio existente~\parencite{nalix_interactive_nli_querying_xml, survey_nlidb}.

\subsection{GINLIDB}
\tbd

\subsection{Sumário do Casos de Estudo}
\tbd

% Ferramentas PLN
\section{Ferramentas PLN}
\label{sec:chap03_existingtools}

\textit{Em desenvolvimento}

\subsection{Amazon Lex}
\textit{Em desenvolvimento}

\subsection{FriendlyData FETCH}
\textit{Em desenvolvimento}

\subsection{Google DialogFlow}
\textit{Em desenvolvimento}

\subsection{IBM Watson}
\textit{Em desenvolvimento}

\subsection{Microsoft LUIS}
\textit{Em desenvolvimento}


\chapter{Conceção}
\label{chap:Chapter4}
Neste capítulo apresenta-se o processo de idealização do modelo de \glsfirst{iln}. Inicialmente, mostra-se o processo de experimentação de algumas ferramentas ou abordagens analisadas no capítulo anterior, procurando perceber se a sua aplicação prática é plausível. Depois, apresenta-se o modelo concetualizado, concretamente a visão geral, em que se faz uma descrição do modelo na sua globalidade, os casos de uso identificados e por fim, a arquitetura definida.

%%%%%%%%%%%%%%%%%%%%%%%%%%%%%%%%%
%           SECTION
%%%%%%%%%%%%%%%%%%%%%%%%%%%%%%%%%
\section{Apreciação Prática}
\label{sec:chap04_approaches}
Com o propósito de colocar em prática o estudo das \glspl{iln}, decidiu-se realizar experiências envolvendo algumas abordagens e ferramentas previamente estudadas (ver Capítulo~\ref{chap:Chapter3}). Posto isto, optou-se por desenvolver pequenas provas de conceito, num período bem definido, de forma a fazer uma rápida avaliação das abordagens, técnicas e ferramentas usadas. Para cada uma, é abordado em que consiste, algumas observações pertinentes e referência dos pontos favoráveis e desfavoráveis.

\subsection{Gramática Baseada em Semântica}
Como proposto em~\textcite[p.~361-403]{natural_language_processing_with_python}, o uso do NLTK permite o desenvolvimento de gramáticas que possibilitam a análise de uma frase pela sua semântica, tal como o exemplo demonstrado a seguir.

\begin{lstlisting}[language=Python,caption={Excerto de uma gramática extraída de~\textcite{natural_language_processing_with_python}},numbers=none,label=lst:grammarexample,basicstyle=\scriptsize]
S[SEM=(?np + WHERE + ?vp)] -> NP[SEM=?np] VP[SEM=?vp]
VP[SEM=(?v + ?pp)] -> IV[SEM=?v] PP[SEM=?pp]
VP[SEM=(?v + ?ap)] -> IV[SEM=?v] AP[SEM=?ap]
NP[SEM=(?det + ?n)] -> Det[SEM=?det] N[SEM=?n]
PP[SEM=(?p + ?np)] -> P[SEM=?p] NP[SEM=?np]
AP[SEM=?pp] -> A[SEM=?a] PP[SEM=?pp]
NP[SEM='Country="greece"'] -> 'Greece'
NP[SEM='Country="china"'] -> 'China'
Det[SEM='SELECT'] -> 'Which' | 'What'
N[SEM='City FROM city_table'] -> 'cities'
IV[SEM=''] -> 'are'
A[SEM=''] -> 'located'
P[SEM=''] -> 'in'
\end{lstlisting}

O código apresentado em~\ref{lst:grammarexample}, usando o NLTK, para a pergunta \inquotes{What cities are located in China?} permite gerar a seguinte \textit{query} de \gls{sql}: SELECT City FROM city\_table WHERE Country=\inquotes{china}. Com esta metodologia é possível mapear diretamente linguagem natural em \gls{sql}, sendo que a gramática funciona como a base de conhecimento. Por conseguinte, torna-se viável a codificação de um \textit{parser} que usa as gramáticas definidas, verificando
alguma é capaz de dar resposta à \textit{query} de linguagem natural.

\subsubsection*{Pontos Favoráveis}
\begin{itemize}
    \item 
    {
        Viabilidade de desenvolvimento de uma solução customizada;
    }
    \item
    {
        Utilização de ferramenta \textit{open source};
    }
    \item 
    {
        Tradução imediata de linguagem natural para \gls{sql}. 
    }
\end{itemize}

\subsubsection*{Pontos Desfavoráveis}
\begin{itemize}
    \item 
    {
        Dificuldade em criar ou estender gramáticas;
    }
    \item
    {
        Complexidade em adaptar para múltiplas línguas ou domínios;
    }
    \item
    {
        Necessidade de conhecimento linguístico, por forma a adaptar a gramática às regras de uma determinada língua;
    }
    \item
    {
        Ambiguidade da semântica. Diferentes expressões podem ou devem gerar o mesmo \gls{sql};
    }
    \item
    {
        Manutenção difícil.
    }
\end{itemize}

\subsection{Pesquisa Semântica}
Nesta abordagem foi usada uma \textit{framework} que não consta nas ferramentas estudadas (ver Secção~\ref{sec:chap03_existingtools}), principalmente porque o seu desenvolvimento encontra-se estagnado desde 2013. Ainda assim, é aceitável o seu uso para testar esta mesma abordagem, no ponto de vista prático. O Quepy é desenvolvido em \textit{Python} e tem como objetivo a transformação de linguagem natural em \textit{SPARQL Protocol and RDF Query Language} (SPARQL)\footnote{Disponível em \url{https://www.w3.org/TR/rdf-sparql-query/}.} e \textit{Metaweb Query Language} (MQL)\footnote{Disponível em \url{https://github.com/nchah/freebase-mql}.}, linguagens de \textit{query} usadas com tecnologia \gls{rdf}\footnote{Disponível em \url{https://www.w3.org/RDF/}.}, \inquotes{um modelo padrão para permuta de dados na Web}\footnote{Tradução livre do autor. No original \inquotes{is a standard model for data interchange on the Web.}.}, intrinsecamente ligado ao campo da Web Semântica~\parencite{resource_description_framework}.

Neste caso, o uso do Quepy implica a definição da linguagem específica de domínio através de código \textit{Python}, respeitando as restrições colocadas na documentação da \textit{framework}, de maneira a que esta seja capaz de reconhecer o padrão introduzido.

\begin{lstlisting}[language=python, caption={Excerto da definição semântica da frase que lida com listagem de produtos},numbers=none,label=lst:quepyexample,basicstyle=\scriptsize]
from quepy.dsl import FixedType, FixedRelation
from quepy.parsing import Lemma, QuestionTemplate

class NameOf(FixedRelation):
    "Name of the entity's property"
    relation = "product_name"
    reverse = True

class IsProduct(FixedType):
    "Defines the entity's type"
    fixedtype = "product"

class ListProducts(QuestionTemplate):
    "Questions such as 'List products' or 'Listing products'"
    regex = Lemma("list") + Lemma("product")

    def interpret(self, match):
        product = IsProduct()
        name = NameOf(product)
        return name, "enum"

\end{lstlisting}

No excerto demonstrado em~\ref{lst:quepyexample}, define-se a expressão regular que caracteriza a questão esperada, para além de detalhes relacionados com a linguagem específica de domínio. A \textit{query} obtida pode ser posteriormente usada numa base de dados \gls{rdf}, como o objetivo de recolher os dados. De notar que, neste caso, sendo o propósito a consulta de uma base de dados relacional, seria necessária uma forma de a transpor ou replicar numa camada compatível com tecnologia \gls{rdf}. Para isso, existem ferramentas capazes de executar esse mapeamento, como por exemplo o D2RQ\footnote{Disponível em \url{http://d2rq.org/}.}.

\subsubsection*{Pontos Favoráveis}
\begin{itemize}
    \item
    {
        Viabilidade de desenvolvimento de uma solução customizada;
    }
    \item
    {
        Tecnologia \textit{open source};
    }
    \item 
    {
        Linguagem específica de domínio permite abstração aos detalhes de implementação associados ao reconhecimento de linguagem natural;
    }
    \item
    {
        Reconhecimento da linguagem natural é feita através da análise de expressões regulares, o que torna fácil compreender a estrutura frásica;
    }
    \item
    {
        Transformação da linguagem natural numa representação que obedece à especificação da tecnologia \gls{rdf}, definida pela W3C.
    }
\end{itemize}

\subsubsection*{Pontos Desfavoráveis}
\begin{itemize}
    \item
    {
        As regras de domínio e o modelo de dados devem ser definidos diretamente no código, não possibilitando a extensibilidade por configuração;
    }
    \item
    {
        Rigidez da linguagem usada, já que o uso de expressões regulares implica uma estrutura minimamente estática para que o reconhecimento seja bem sucedido;
    }
    \item
    {
        O uso de tecnologias \gls{rdf} levam a uma nova camada aplicacional, que envolve o uso de ferramentas não relacionadas com o \gls{pln} e que leva a mais esforço na manutenção do sistema;
    }
    \item
    {
        Necessidade de estudo aprofundado ou conhecimento pericial na área de Web Semântica.
    }
\end{itemize}

\subsection{Reconhecimento de Intenções e Entidades}
Neste contexto, foi utilizado o Microsoft LUIS. A razão para esta escolha deve-se principalmente a dois fatores: o facto de ser uma das plataformas mais usadas e documentadas e pela Microsoft disponibilizar este tipo de ferramentas a quem possua subscrição de estudante. Para esta prova de conceito criou-se um \textit{chatbot} capaz de responder a algumas perguntas colocadas pelo utilizador (\textit{utterance}), identificando qual a sua intenção (\textit{intent}) e quais as entidades envolvidas (\textit{entity}), descritas na Tabela~\ref{tab:luis_intents}.
%
\begin{table}
\caption{Descrição de algumas das intenções e entidades dada a expressão de exemplo, baseado em~\textcite[Concepts]{microsoft_luis_official}}
\label{tab:luis_intents}
\centering
\resizebox{\textwidth}{!}{
\renewcommand{\arraystretch}{1.3}
\footnotesize
\begin{tabular}{l*{2}{|l}}
%
\toprule
%
\tabhead{Expressão de Exemplo}&\tabhead{Intenção}&\tabhead{Entidades}\\
%
\midrule
%
{Good Morning}&{Greeting}&{--}\\
%
{What's the weather like in \underline{Porto}?}&{CheckWeather}&{Place}\\
%
{Turn on the Internet in my \underline{bedroom}, please}&{HomeAutomation}&{Room}\\
%
{Book me a flight to \underline{Lisbon} \underline{next week}}&{BookFlight}&{Place, Datetime}\\
%
\bottomrule
%
\end{tabular}
}
\end{table}

O \textit{chatbot} recorre ao Microsoft LUIS para identificar a intenção do utilizador, e com base nisso e nas entidades recolhidas, pode carregar os dados necessários e apresentá-los da forma que for mais adequada. Neste caso, o \textit{chatbot} retorna apenas a intenção que foi capaz de identificar (ver Figura~\ref{fig:chatbotexample}).
%
\begin{figure}
\centering
\includegraphics[width=.8\textwidth]{ch03/assets/chatbot.png}
\caption{Resultado do teste com \textit{chatbot} desenvolvido usando Microsoft LUIS}
\label{fig:chatbotexample}
\end{figure}
%
É possível dotar o \textit{chatbot} de comportamento para manipular os metadados obtidos, e assim, obter dados de uma base de dados relacional ou obter dados estáticos dum serviço de respostas pré-fabricadas, tal como o QnA Maker\footnote{Disponível em \url{https://www.qnamaker.ai/}}. É importante frisar que, embora esta abordagem tenha sido aplicada recorrendo a um \textit{chatbot}, por imposição da própria plataforma, tal não é obrigatório na utilização da mesma numa solução customizada. 

\subsubsection*{Pontos Favoráveis}
\begin{itemize}
    \item
    {
        As plataformas disponíveis na \textit{cloud} constituem um ponto de partida para uma solução personalizada;
    }
    \item
    {
        A solução é fácil de desenvolver, estender e integrar;
    }
    \item
    {
        A base de conhecimento pode ser totalmente configurável;
    }
    \item
    {
        Robustez na identificação de intenções e entidades, através do uso de modelos de \gls{ml};
    }
\end{itemize}

\subsubsection*{Pontos Desfavoráveis}
\begin{itemize}
    \item
    {
        A aplicação desta abordagem numa biblioteca customizada implica o estudo teórico de \gls{ml} aplicado ao \gls{pln} e consequentemente, maior esforço de desenvolvimento;
    }
    \item
    {
        A adição de novas intenções à base de conhecimento leva também à adição de comportamento para lidar com os mesmas.
    }
\end{itemize}

\subsection{Sinopse}
As abordagens descritas anteriormente baseiam-se nas observações e experiências realizadas, no ponto de vista prático. Portanto, a conclusão aqui exposta leva em consideração os pontos favoráveis e desfavoráveis de cada uma.

A abordagem que parece a mais adequada é a que diz respeito ao reconhecimento de intenções e entidades, principalmente pela facilidade de compreensão e aplicação do conceito. Além do mais, os pontos desfavoráveis mencionados são de índole operacional, pelo que podem ser superados ou até descartados aquando a conceção e/ou desenvolvimento da solução final. Relativamente às abordagens descartadas, aponta-se que a primeira -- gramática baseada em semântica -- se usada no contexto de uma solução final, será difícil manter o seu desenvolvimento e capacidade de cobrir uma gama aceitável de questões válidas. Já a segunda -- pesquisa semântica --, ainda que assente sobre uma tecnologia especificada e aprovada pelo W3C, necessita de uma camada adicional (\gls{rdf}), contribuindo assim para o aumento do esforço em configuração e manutenção do sistema. Por isso, a terceira abordagem revela-se a mais adequada e será contemplada no protótipo a desenvolver.

%%%%%%%%%%%%%%%%%%%%%%%%%%%%%%%%%
%           SECTION
%%%%%%%%%%%%%%%%%%%%%%%%%%%%%%%%%
\section{Modelo Proposto}
\label{sec:chap04_proposal}
O estudo das \glspl{iln}, associado à aplicação prática de algumas ferramentas e abordagens estudadas, permitiram retirar conclusões relevantes de considerar na conceção do modelo. Nesta secção apresenta-se o modelo idealizado para uma \gls{iln} que permita consulta e pesquisa e estados do processo de fabrico, usando uma abordagem de reconhecimento de intenções e entidades.

\subsection{Visão Geral}
O intuito é especificar um modelo que seja possa ser seguido na futura implementação de uma \gls{iln} integrada no {\productname}, que por sua vez irá permitir a consulta e pesquisa de informação acerca do processo fabril, por parte de operadores de linha, engenheiros de produção e gestores. Como se apresenta na Figura~\ref{fig:generic-vision}, o sistema \gls{mes} disponibiliza uma interface para que o utilizador interaja com ele, através de texto e posteriormente voz, esperando obter informação relevante do processo, no formato que melhor se enquadrar.
%
\begin{figure}[!h]
    \centering
    \includegraphics[width=.9\textwidth]{ch04/assets/generic-vision.jpg}
    \caption{Visão da solução desejada para o produto}
    \label{fig:generic-vision}
\end{figure}

O modelo deve ter em conta quatro fatores considerados importantes na possível implementação final: a experiência de utilizador, a extensão da base de conhecimento com o propósito de suportar diferentes domínios, a utilização de \textit{feedback} de utilizador para a aprimoramento da qualidade das respostas geradas e a fácil integração com o {\productname}.

Relativamente à estratégia de reconhecimento de intenções e entidades a ser aplicada no modelo, definem-se os conceitos que lhe estão inerentes, apresentados na Figura~\ref{fig:domain_model}:

\begin{itemize}
    \item 
    {
        \textit{Dynamic Knowledge Base} -- Base de conhecimento dinâmica. Fonte de dados que liga as expressões às intenções e entidades correspondentes e é usado no treino dos modelos \gls{ml} para previsão;
    }
    \item 
    {
        \textit{Static Knowledge Base} -- Base de conhecimento estática. Dicionário onde consta um determinada expressão e respetiva resposta. Pode ser usada para implementar conversação casual ou lidar com expressões que não constam na base de dados dinâmica;
    }
    \item 
    {
        \textit{Intention} -- Intenção. Mapeia ação que o utilizador deseja executar. Normalmente estão-lhe associadas diversas expressões;
    }
    \item 
    {
        \textit{Expression} -- Expressão. Corresponde ao exemplo de estrutura de uma pergunta que o utilizador pode fazer;
    }
    \item 
    {
        \textit{Entity} -- Entidade. Tipicamente refere-se a um conceito de domínio ou do mundo real;
    }
    \item 
    {
        \textit{Static Answer} -- Resposta Estática. Definida na base de conhecimento estática, associada a diferentes expressões. Por exemplo, expressões como \inquotes{Olá} ou \inquotes{Viva} podem ter associadas a respostas como \inquotes{Oi} e \inquotes{Olá}.
    }
\end{itemize}
%
\begin{figure}
    \centering
    \includegraphics[width=.9\textwidth]{ch04/assets/domain-model.jpg}
    \caption{Conceitos associados ao reconhecimento de intenções e entidades e os respetivos relacionamentos}
    \label{fig:domain_model}
\end{figure}
%
Portanto, uma intenção é um conceito composto, base do modelo, que se apresenta como uma indireção face à questão colocada pelo utilizador, ou seja, a questão é analisada pela ação que lhe está associada e não propriamente pelo seu significado. Desta forma, possibilita-se a escalabilidade e a generalização de domínio, \idest{um pedido para obter os materiais mais fabricados num determinado setor constitui uma intenção única, apesar de apresentar entidades diferentes, consoante o domínio}.

\subsection{Casos de Uso}
De um ponto de vista de utilização, o modelo aqui proposto deve saber lidar com \textit{Perguntas}, \textit{Respostas} e \textit{Feedback}, que se podem considerar as principais áreas funcionais de sistema. Todas têm implicações no bom funcionamento do modelo, sendo que a última, ainda que não seja essencial, é importante na medida em que contribui para melhoria contínua do sistema. Em relação às \textit{Perguntas} e \textit{Respostas}, uma vez que estão profundamente ligadas, apresentando funcionalidades comuns, são então enquadradas numa área funcional única, denominada \textit{Q\&A}. A Figura~\ref{fig:use_cases} dá uma visão geral das áreas funcionais do sistema.
%
\begin{figure}
    \centering
    \includegraphics[width=.9\textwidth]{ch04/assets/use-cases.jpg}
    \caption{Áreas funcionais de sistema}
    \label{fig:use_cases}
\end{figure}

As áreas funcionais são descritas em seguida:

\begin{itemize}
    \item 
    {
        \textit{Q\&A} -- corresponde ao conjunto de funcionalidades relacionadas com as perguntas colocadas pelos utilizadores e respetiva procura de respostas;
    }
    \item 
    {
        \textit{Feedback} -- conjunto de funções de sistema que lida com a recolha de \textit{feedback} do utilizador, para contribuir para a melhoria contínua da qualidade das respostas;
    }
    \item 
    {
        \textit{Knowledge Base} -- área funcional \textit{Base de Conhecimento}. Apresenta funcionalidades para a gestão de intenções, expressões, entidades, ou seja, tudo o que as restantes áreas funcionais necessitam para o seu funcionamento.
    }
\end{itemize}

A área funcional \textit{Q\&A} é essencial no contexto deste trabalho, já que se trata da base de interação com o utilizador. Por isso, a Figura~\ref{fig:detailed_use_cases}, apresentada em seguida, detalha algo mais essa área funcional.
%
\begin{figure}[!h]
    \centering
    \includegraphics[width=\textwidth]{ch04/assets/questions-use-cases.jpg}
    \caption{Casos de uso identificados para a área funcional \textit{Q\&A}}
    \label{fig:detailed_use_cases}
\end{figure}

Os casos de uso apresentados no diagrama são descritos a seguir:

\begin{itemize}
    \item 
    {
        \textit{Ask Question} -- Fazer Pergunta. É a funcionalidade principal. O utilizador faz a pergunta com o objetivo de obter a resposta que procura. Depende da compreensão da pergunta e da obtenção de resposta;
    }
    \item 
    {
        \textit{Comprehend Question} -- Compreender a Pergunta. O sistema procura compreender a pergunta feita e traduz para uma representação passível de ser usada na fase de procura da resposta nas fontes de dados disponíveis;
    }
    \item 
    {
        \textit{Recognize Intent and Entities} -- Reconhecer a Intenção e as Entidades. O sistema faz o reconhecimento da intenção e das entidades da pergunta feita com base nos modelo de \gls{ml} treinado com os dados que constam na base de conhecimento;
    }
    \item 
    {
        \textit{Build Intermediate Representation} -- Construir Representação Intermédia. O sistema constrói uma representação intermediária que contém os metadados da pergunta colocada -- intenção, entidades e outros dados relevantes;
    }
    \item 
    {
        \textit{Consult Intentions} -- Consultar as Intenções. Permite a consulta de intenções mantidas na base de conhecimento;
    }
    \item 
    {
        \textit{Consult Entities} -- Consultar as Entidades. Permite a consulta as entidades mantidas na base de conhecimento;
    }
    \item 
    {
        \textit{Fetch Answer} -- Obter a Resposta. O sistema usa o resultado (representação intermediária) da fase de compreensão da pergunta para converter essa representação numa compatível com a fonte de dados a interagir, obtendo assim a resposta.
    }
\end{itemize}

Os casos de uso apresentados são a base das funcionalidades do modelo, e por isso, serão explorados em contexto prático no capítulo seguinte.

\subsection{Arquitetura}
Dada a visão geral do modelo e dos seus casos de uso, pretende-se expor a estrutura da solução numa perspetiva lógica, partindo do pressuposto que será o modelo usado na solução final a desenvolver para o {\productname}. Assim, a Figura~\ref{fig:prototype_architecture} demonstra a arquitetura do modelo, primeiro num vista a alto nível e depois mais focado no elemento principal, explicando a responsabilidade inerente a cada componente.
%
\begin{figure}
\centering
    \begin{subfigure}{\textwidth}
         \centering
         \includegraphics[width=\textwidth]{ch04/assets/generic-architecture.jpg}
         \caption{Arquitetura genérica do protótipo}
         \label{fig:generic_architecture}
     \end{subfigure}
     \bigbreak
     \bigbreak
     \begin{subfigure}{\textwidth}
         \centering
         \includegraphics[width=\textwidth]{ch04/assets/nl-engine.jpg}
         \caption{Arquitetura detalhada do \textit{NL Engine}}
         \label{fig:nlengine_architecture}
     \end{subfigure}
\caption{Arquitetura do protótipo, apresentando um vista genérica e uma mais específica do componente \textit{NL Engine}}
\label{fig:prototype_architecture}
\end{figure}
%
\begin{itemize}
    \item 
    {
        \textit{NL Presentation} -- responsável pela interação com o utilizador. Integra a camada de apresentação do {\productname} e, como tal, é desenvolvido de acordo com as especificidades do subsistema em que se insere;
    }
    \item 
    {
        \textit{NL Engine} -- o módulo de linguagem natural, ou seja, o \inquotes{motor} que permite a tradução de linguagem natural em intenções e respetivas entidades. Integra a camada aplicacional de serviços do produto;
    }
    \item 
    {
        \textit{Q\&A Service} -- trata-se de um subcomponente o \textit{NL Engine}, que conhece as partes envolvidas no processo de aquisição de resposta, sendo responsável orquestrar esse processo. Trabalha em conjunto com o \textit{Comprehension Service} com o objetivo de identificar e mapear a intenção e entidades de uma dada \textit{query} de linguagem natural para obter a resposta da fonte de dados produtivos. Numa analogia à anatomia humana, pode ser considerado o \inquotes{cérebro} do processo;
    }
    \item 
    {
        \textit{Comprehension Service} -- outro subcomponente do \textit{NL Engine}, trabalha com a base de conhecimento definida (\textit{NL Knowledge Base}) para executar a tarefa de compreender a pergunta colocada. É neste componente que se insere a ferramenta para processamento de linguagem natural escolhida;
    }
    \item 
    {
        \textit{Feedback Management} -- tem a responsabilidade de gerir o \textit{feedback} providenciado pelo utilizador e disponibiliza serviços para a gestão dessa mesma informação, que incluem o desencadear do processo de aprendizagem, por exemplo. Inicialmente, o \textit{feedback} pode ser consultado manualmente, possibilitando o uso dessa informação para a melhoria do sistema. Contudo, podem ser aplicadas estratégias que façam uso desta informação de forma automática;
    }
    \item 
    {
        \textit{NL Knowledge Base} -- base de conhecimento de domínio, incluída na camada de persistência do {\productname}. É configurada pela equipa de desenvolvimento e deve mapear as entidades de domínio, as intenções em que estão envolvidas e suportar o registo de \textit{feedback}, para que possa ser usado na aprendizagem do módulo;
    }
    \item 
    {
        \textit{ProductionDB} -- armazém de dados de negócio. Contém a informação que o utilizador deseja obter através de linguagem natural.
    }
\end{itemize}

Apesar da elucidação acerca da responsabilidade de cada componente no sistema, é importante detalhar a forma como estes interagem entre si, para atingir o objetivo. A Figura~\ref{fig:prototype_sequence_diagram} mostra como se desenrola a comunicação entre os diversos componentes, que se passa a descrever: o utilizador (\textit{User}) questiona o sistema através da interface gráfica (\textit{NL Presentation}). A questão é encaminhada para o \textit{Q\&A Service} que se encarrega de \inquotes{pedir} ao \textit{Comprehension Service} para que lhe forneça a compreensão sob forma de representação intermediária. Posto isto, o \textit{Comprehension Service} consulta a base de conhecimento (\textit{NL Knowledge Base}) para consultar o conteúdo existente e, aplicando modelos de \gls{ml}, faz o reconhecimento das intenções e entidades presentes na questão. O \textit{Q\&A Service} trata de converter a representação numa linguagem compatível com o \textit{ProductionDB} e executa a \textit{query} gerada. Aquando a aquisição dos dados, o \textit{Q\&A Service} trata de \inquotes{documentá-los}, ou seja, colocar metadados que sejam importantes para processamento posterior. Por fim, a \textit{NL Presentation} processa o resultado para que este seja apresentado num formato adequado ao utilizador. Após o processo descrito, o sistema pede por \textit{feedback} do utilizador (\textit{User}). Aquando recebido, o \textit{feedback} é entregue ao \textit{Feedback Management}, que se encarrega de enviá-lo para a \textit{NL Knowledge Base} para que seja armazenado. Este último componente salva o \textit{feedback} terminando com sucesso o processo inteiro, estando o sistema pronto para inicializar um novo.

\begin{sidewaysfigure}
    \centering
    \includegraphics[width=\textwidth]{ch04/assets/workflow.jpg}
    \caption{Comunicação entre os componentes do módulo de linguagem natural}
    \label{fig:prototype_sequence_diagram}
\end{sidewaysfigure}

%%%%%%%%%%%%%%%%%%%%%%%%%%%%%%%%%
%           SECTION
%%%%%%%%%%%%%%%%%%%%%%%%%%%%%%%%%
\section{Síntese} 
\label{sec:chap04_chaptersummary}
Neste capítulo descreveu-se o processo que levou à especificação do modelo, dando ênfase aos detalhes da sua arquitetura. 

Começou-se por descrever a apreciação feita, em contexto prático, a algumas das ferramentas e abordagens anteriormente estudadas, fazendo referência aos seus pontos favoráveis e desfavoráveis. A conclusão retirada foi a de que a abordagem de reconhecimento de intenções e entidades seria a escolha adequada para o modelo a conceber.

Posteriormente, apresentou-se o modelo proposto para a solução de \gls{iln}, descrevendo a visão contemplada, as áreas funcionais -- \textit{Q\&A}, \textit{Feedback} e \textit{Knowledge Base} -- e os casos de uso identificados, detalhando os mais importantes. Finalmente, a arquitetura num ponto de vista lógica, explicando os componentes e o fluxo de trabalho entre eles, no qual se apresenta um cenário genérico do funcionamento do processo de tradução e \textit{feedback}.

\chapter{Conclusão}
\label{chap:Chapter5}

\textit{A definir}

\section{Avaliação de objetivos} 
\label{sec:chap5_goals_evaluation}

\textit{A definir}

\section{Resposta ao problema} 
\label{sec:chap5_problem_response}

\textit{A definir}

\section{Limitações e trabalho futuro} 
\label{sec:chap5_future_work_limitations}

\textit{A definir}

\chapter{Validação}
\label{chap:Chapter6}
\tbd

%%%%%%%%%%%%%%%%%%%%%%%%%%%%%%%%%
%           SECTION
%%%%%%%%%%%%%%%%%%%%%%%%%%%%%%%%%
\section{Resposta às Questões-Chave}
\label{sec:chap06_answers}
\tbd

%%%%%%%%%%%%%%%%%%%%%%%%%%%%%%%%%
%           SECTION
%%%%%%%%%%%%%%%%%%%%%%%%%%%%%%%%%
\section{Uso do \textit{Feedback}}
\label{sec:chap06_feedback_usage}
\tbd

%%%%%%%%%%%%%%%%%%%%%%%%%%%%%%%%%
%           SECTION
%%%%%%%%%%%%%%%%%%%%%%%%%%%%%%%%%
\section{Síntese do Capítulo}
\label{sec:chap06_chaptersummary}
\tbd

\chapter{Conclusões}
\label{chap:Chapter7}
Com o presente trabalho tentou-se contribuir para o conhecimento na área do Processamento de Linguagem Natural orientado às Bases de Dados. Tal como referido no início desta dissertação, pretende-se definir uma abordagem de resolução do problema, e não propriamente uma solução definitiva.

Para finalização de qualquer trabalho, é conveniente salientar as conclusões alcançadas, relacioná-las com os objetivos propostos, criticar as suas limitações e referir as perspetivas futuras. Por isso, inicia-se por enumerar os objetivos para o trabalho e avaliar em que medida foram ou não cumpridos, de acordo com os critérios de sucesso definidos. Depois, especificam-se as contribuições e de que forma se apresentam como respostas para o problema. Finalmente, são constatadas as limitações do trabalho desenvolvido e perspetivas de trabalho futuro, apontando problemas em aberto ou as alternativas a contemplar.

%%%%%%%%%%%%%%%%%%%%%%%%%%%%%%%%%
%           SECTION
%%%%%%%%%%%%%%%%%%%%%%%%%%%%%%%%%
\section{Avaliação de Objetivos} 
\label{sec:chap07_goals_evaluation}
\tbd

%%%%%%%%%%%%%%%%%%%%%%%%%%%%%%%%%
%           SECTION
%%%%%%%%%%%%%%%%%%%%%%%%%%%%%%%%%
\section{Resposta ao Problema} 
\label{sec:chap07_problem_response}
\tbd

%%%%%%%%%%%%%%%%%%%%%%%%%%%%%%%%%
%           SECTION
%%%%%%%%%%%%%%%%%%%%%%%%%%%%%%%%%
\section{Limitações e Trabalho Futuro} 
\label{sec:chap07_future_work_limitations}
\tbd


%----------------------------------------------------------------------------------------
%	BIBLIOGRAPHY
%----------------------------------------------------------------------------------------

\printbibliography[heading=bibintoc]

%----------------------------------------------------------------------------------------
%	THESIS CONTENT - APPENDICES
%----------------------------------------------------------------------------------------

\appendix % Cue to tell LaTeX that the following "chapters" are Appendices

% Include the appendices of the thesis as separate files from the Appendices folder
% Uncomment the lines as you write the Appendices

% Appendix A

\chapter{Apêndice Teste} % Main appendix title

\label{AppendixA} % For referencing this appendix elsewhere, use \ref{AppendixA}

Write your Appendix content here.
\input{appendices/appendixB}
\chapter{Protótipo}
\label{AppendixC}
Neste apêndice são mostrados alguns artefactos recolhidos ao longo das fases de conceção, desenvolvimento e validação do protótipo.

\section{Configuração}
Nesta secção apresentam-se algumas imagens referentes ao processo de configuração levado no protótipo.
%
\begin{figure}[!ht]
    \centering
    \includegraphics[width=\textwidth]{appendices/assets/kb07.png}
    \caption{Definição das entidades esperadas}
\end{figure}
%
\begin{figure}
\centering
    \begin{subfigure}{.9\textwidth}
        \centering
        \includegraphics[width=\textwidth]{appendices/assets/kb01.png}
        \caption{Intenção \textit{AverageOperationOnStep}}
     \end{subfigure}
     \begin{subfigure}{.9\textwidth}
         \centering
        \includegraphics[width=\textwidth]{appendices/assets/kb02.png}
        \caption{Intenção \textit{CountMaterialsOnCondition}}
     \end{subfigure}
     \begin{subfigure}{.9\textwidth}
        \centering
        \includegraphics[width=\textwidth]{appendices/assets/kb03.png}
        \caption{Intenção \textit{CountMaterialsOnConditionOnStep}}
     \end{subfigure}
\caption{Intenções definidas, contendo as expressões e respetivas entidades}
\end{figure}
%
\begin{figure}
    \centering
         \begin{subfigure}{.9\textwidth}
        \centering
        \includegraphics[width=\textwidth]{appendices/assets/kb04.png}
        \caption{Intenção \textit{CountMaterialsOnConditionOnStepGrouped}}
     \end{subfigure}
     \begin{subfigure}{.9\textwidth}
        \centering
        \includegraphics[width=\textwidth]{appendices/assets/kb05.png}
        \caption{Intenção \textit{CountOperationsByProduct}}
     \end{subfigure}
     \begin{subfigure}{.9\textwidth}
        \centering
        \includegraphics[width=\textwidth]{appendices/assets/kb06.png}
        \caption{Intenção \textit{SumOperationsByWeek}}
     \end{subfigure}
    \caption{Continuação das intenções definidas, contendo as expressões e respetivas entidades}
\end{figure}

\clearpage

\section{Validação}
Nesta secção apresentam-se algumas imagens referentes ao processo de validação do protótipo.

\begin{figure}[!ht]
\centering
    \begin{subfigure}{.48\textwidth}
        \centering
        \includegraphics[width=\textwidth]{appendices/assets/nlcomprehension01.png}
        \caption{Intenção \textit{SumOperationsByWeek}}
     \end{subfigure}
     \begin{subfigure}{.48\textwidth}
         \centering
        \includegraphics[width=\textwidth]{appendices/assets/nlcomprehension02.png}
        \caption{Intenção \textit{CountOperationByProduct}}
     \end{subfigure}
     \bigbreak
     \begin{subfigure}{.48\textwidth}
        \centering
        \includegraphics[width=\textwidth]{appendices/assets/nlcomprehension03.png}
        \caption{Intenção \textit{CountOperationsByProductPerShift}}
     \end{subfigure}
     \begin{subfigure}{.48\textwidth}
        \centering
        \includegraphics[width=\textwidth]{appendices/assets/nlcomprehension04.png}
        \caption{Intenção \textit{AverageOperationsOnStep}}
     \end{subfigure}
     \bigbreak
     \begin{subfigure}{.48\textwidth}
        \centering
        \includegraphics[width=\textwidth]{appendices/assets/nlcomprehension05.png}
        \caption{Intenção \textit{CountMaterialsOnCondition}}
     \end{subfigure}
     \begin{subfigure}{.48\textwidth}
        \centering
        \includegraphics[width=\textwidth]{appendices/assets/nlcomprehension06.png}
        \caption{Intenção \textit{CountOperationsOnConditionOnStep}}
     \end{subfigure}
\caption{Outras imagens relativas à avaliação de intenções e entidades do protótipo}
\label{fig:nlcomprehesion_others}
\end{figure}

\printglossary
\glsresetall
\end{document}
