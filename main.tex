%%%%%%%%%%%%%%%%%%%%%%%%%%%%%%%%%%%%%%%%%
% TMDEI Dissertation
% LaTeX Template
% Version 0.1 (Dec/2015)
%
% Adapted to TMDEI/ISEP style (Dec/2015) by
%  Nuno Pereira (nap@isep.ipp.pt) and
%  Paulo Baltarejo (pbs@isep.ipp.pt)
%
% Based on MastersDoctoralThesis Version 1.2 by Vel (vel@latextemplates.com) and
% Johannes Böttcher, downloaded from (21/11/15):
% http://www.LaTeXTemplates.com
%
% This template is originally based on a template by:
% Steve Gunn (http://users.ecs.soton.ac.uk/srg/softwaretools/document/templates/)
% Sunil Patel (http://www.sunilpatel.co.uk/thesis-template/)
%
% Template license:
% CC BY-NC-SA 3.0 (http://creativecommons.org/licenses/by-nc-sa/3.0/)
%
%%%%%%%%%%%%%%%%%%%%%%%%%%%%%%%%%%%%%%%%%

%----------------------------------------------------------------------------------------
%	PACKAGES AND OTHER DOCUMENT CONFIGURATIONS
%----------------------------------------------------------------------------------------

\documentclass[
11pt, % The default document font size, options: 10pt, 11pt, 12pt
oneside, % Two side (alternating margins) for binding by default, uncomment to switch to one side (for drafting/reading purposes)
%english, % english for English;
portuguese,% for Portuguese; delete temporary files if you change language (e.g. 'make clean; make')
singlespacing, % Single line spacing, alternatives: onehalfspacing or doublespacing (for drafting/reading purposes)
%draft, % Uncomment to enable draft mode (no pictures, no links, overfull hboxes indicated)
%nolistspacing, % If the document is onehalfspacing or doublespacing, uncomment this to set spacing in lists to single
liststotoc, % Uncomment to add the list of figures/tables/etc to the table of contents (not recommended)
%toctotoc, % Uncomment to add the main table of contents to the table of contents (not recommended)
parskip, % Add space between paragraphs (recommended)
%nohyperref, % Uncomment to not load the hyperref package (not recommended)
nohyperreflinkcolor, % hyperref links are not colored (comment to color links, for example to produce an electronic-only version)
headsepline, % Uncomment to get a line under the header
]{tmdei-style} % The class file specifying the document structure

\usepackage{rotating}
\usepackage{soul}
\usepackage{tikz} % Required for creating graphics programmatically (can be removed if not used)
%\usetikzlibrary{arrows} % Required for fancy arrows in TiKZ graphics (can be removed if not used)
\usepackage{pgfplots} % Required for drawing high--quality function plots (can be removed if not used)
\pgfplotsset{compat=newest}

%
% Next you have examples of admissable citation styles; we recomend using the authoryear-comp citation style (which resembles Harvard); don't forget to only uncomment one
%

% authoryear-comp: recommended citation style (e.g. (Buendía, 1860), (Buendía 1910, Arcadio 1940))
\usepackage[style=authoryear-comp,backend=biber]{biblatex} % Bibtex backend with the authoryear-comp citation style (authoryear citations, bibliography ordered alphabetically)
\setlength\bibitemsep{.5\baselineskip}
% numeric citation style (e.g. [1], [1-3])
%\usepackage[style=numeric-comp,sorting=none,backend=biber]{biblatex} % Bibtex backend with the numeric-comp citation style (numeric citations, bibliography ordered by appearance)

% alphabetic citation style (e.g. [Buendía10], [Buendía10, Arcadio40])
%\usepackage[style=alphabetic,sorting=none,backend=biber]{biblatex} % Bibtex backend with the alphabetic citation style (alphabetic citations, bibliography ordered by appearance)
\addbibresource{mainbibliography.bib} % The filename of the bibliography

\makeglossaries % build the glossary

%----------------------------------------------------------------------------------------
%	THESIS INFORMATION
%----------------------------------------------------------------------------------------

\thesistitle{Natural Language Querying} % Your thesis title, this is used in the title, print it elsewhere with \ttitle

%\thesissubtitle{{[}Thesis Subtitle{]}} % Your thesis title, this is used in the title, print it elsewhere with \tsubtitle

\author{Tiago Gabriel da Silva} % Your name, this is used in the title page, print it elsewhere with \authorname

\subjectarea{Sistemas Computacionais} % Specialization area (Computer Systems, Information and Knowledge Systems, Graphics, Systems and Multimedia, Software Engineering), used in the title page, print it elsewhere with \areaname

\supervisor{Dr. Paulo Gandra da Sousa} % Your supervisor's name, this is used in the title page, print it elsewhere with \supname

\cosupervisor{Eng.º Ricardo Magalhães} % Your co-supervisor's name, this is used in the title page, print it elsewhere with \cosupname (comment, if no co-supervisor)

\committeepresident{} % Name of the president of the evaluation committee, print it elsewhere with \presidentname

\committeemembers{} % Name of the evaluation committee members (up to four), print it elsewhere with \committee

\keywords{Inteligência Artificial, Processamento de Linguagem Natural, \textit{Manufacturing Execution System}, \textit{Querying}} % Please define up to 6 keywords that better describe your work, print it elsewhere with \keywordnames

\university{\href{http://www.isep.ipp.pt}{Instituto Superior de Engenharia do Porto}} % Your university's name and URL, this is used in the title page and abstract, print it elsewhere with \univname

\department{\href{http://www.dei.isep.ipp.pt}{Departamento de Engenharia Informática}} % Your department's name and URL, this is used in the title page and abstract, print it elsewhere with \deptname

\thesisdate{Porto, \today} % thesis date,  print it elsewhere with \tdate

\hypersetup{pdftitle=\ttitle} % Set the PDF's title to your title
\hypersetup{pdfauthor=\authorname} % Set the PDF's author to your name
\hypersetup{pdfkeywords=\keywordnames} % Set the PDF's keywords to your keywords

% Variables
\def\companyname{Critical Manufacturing}
\def\productname{Critical Manufacturing MES}

\begin{document}

%----------------------------------------------------------------------------------------
%	FRONT MATTER
%----------------------------------------------------------------------------------------

% Include the frontmatter of your thesis here
% we include the glossary here (frontmatter is included with \input, so this command is as if it was in main.tex)
% Acrónimos
\newacronym{iot}{IoT}{\textit{Internet of Things}}
\newacronym{ios}{IoS}{\textit{Internet of Services}}
\newacronym{ihc}{IHC}{Interação Humano-Computador}
\newacronym{ia}{IA}{Inteligência Artificial}
\newacronym{pln}{PLN}{Processamento de Linguagem Natural}
\newacronym{mes}{MES}{\textit{Manufacturing Execution System}}
\newacronym{sql}{SQL}{\textit{Structured Query Language}}
\newacronym{uml}{UML}{\textit{Unified Modeling Language}}
\newacronym{ceo}{CEO}{\textit{Chief Executive Officer}}
\newacronym{cto}{CTO}{\textit{Chief Technical Officer}}
\newacronym{erp}{ERP}{\textit{Enterprise Resource Planning}}
\newacronym{cps}{CPS}{\textit{Cyber-Physical System}}
\newacronym{ffe}{FFE}{\textit{Fuzzy Front End}}
\newacronym{npd}{NPD}{\textit{New Product Development}}
\newacronym{ncd}{NCD}{\textit{New Concept Development}}
\newacronym{slp}{SLP}{\textit{Single Layer Perceptron}}
\newacronym{ilnbd}{ILNBD}{Interfaces de Linguagem Natural para Bases de Dados}
\newacronym{atn}{ATN}{\textit{Augmented Transition Network}}
% for defining plural form
% \newacronym[shortplural=aa,longplural=letters a]{a}{A}{the a}

\frontmatter % Use roman page numbering style (i, ii, iii, iv...) for the pre-content pages

\pagestyle{plain} % Default to the plain heading style until the thesis style is called for the body content

%----------------------------------------------------------------------------------------
%	TITLE PAGE
%----------------------------------------------------------------------------------------

\maketitlepage

%----------------------------------------------------------------------------------------
%	DEDICATION  (optional)
%----------------------------------------------------------------------------------------
%\dedicatory{For/Dedicated to/To my\ldots}
\begin{dedicatory}
\tbd
\end{dedicatory}

%----------------------------------------------------------------------------------------
%	ABSTRACT PAGE
%----------------------------------------------------------------------------------------
\begin{abstract}

O paradigma de interação entre Homem e Máquina tem vindo a mudar nos últimos anos. Se, ao longo das últimas décadas, o ser humano tem vindo a interagir com o computador através da escrita (linha de comandos) ou das interfaces gráficas, como se desenvolverá esta interação quando a máquina for capaz de \inquotes{entender} a linguagem natural humana?

\end{abstract}

\begin{abstractotherlanguage}
The interaction paradigm between man and machine has been changing in the last years. Over the last decades, humans have been interacting with the computer through writing (command line) or graphical interfaces. Recently, emerges the interaction through natural language. How to enhance the communication between man and the system used on daily basis, by using natural language? The usage of Natural Language Processing, a field of study of Artificial Intelligence, which may involve Machine Learning or Deep Learning techniques, allows the transformation of human language into a representation adapted to computation systems.

This thesis focus on the design of an approach that allows to consult and present information stored in data warehouses, through usage of natural language. As result, a prototype has been developed by putting into practise the conceptualized approach. Thus, the main goal is to adapt and use the suggested approach in the development of a natural language module to interact with the {\productname}, thereby improving the system's usability.

\end{abstractotherlanguage}

%----------------------------------------------------------------------------------------
%	ACKNOWLEDGEMENTS (optional)
%----------------------------------------------------------------------------------------
\begin{acknowledgements}

\tbd

\end{acknowledgements}

%----------------------------------------------------------------------------------------
%	LIST OF CONTENTS/FIGURES/TABLES PAGES
%----------------------------------------------------------------------------------------

\tableofcontents % Prints the main table of contents

\listoffigures % Prints the list of figures

\listoftables % Prints the list of tables

\iflanguage{portuguese}{
\renewcommand{\listalgorithmname}{Lista de Algor\'itmos}
}
\listofalgorithms % Prints the list of algorithms
\addchaptertocentry{\listalgorithmname}


\renewcommand{\lstlistlistingname}{List of Source Code}
\iflanguage{portuguese}{
\renewcommand{\lstlistlistingname}{Lista de C\'odigo}
}
\lstlistoflistings % Prints the list of listings (programming language source code)
\addchaptertocentry{\lstlistlistingname}


%----------------------------------------------------------------------------------------
%	ABBREVIATIONS
%----------------------------------------------------------------------------------------
%\begin{abbreviations}{ll} % Include a list of abbreviations (a table of two columns)
%%\textbf{LAH} & \textbf{L}ist \textbf{A}bbreviations \textbf{H}ere\\
%%\textbf{WSF} & \textbf{W}hat (it) \textbf{S}tands \textbf{F}or\\
%\end{abbreviations}

%----------------------------------------------------------------------------------------
%	SYMBOLS
%----------------------------------------------------------------------------------------

\begin{symbols}{lll} % Include a list of Symbols (a three column table)

$a$ & distance & \si{\meter} \\
$P$ & power & \si{\watt} (\si{\joule\per\second}) \\
%Symbol & Name & Unit \\

\addlinespace % Gap to separate the Roman symbols from the Greek

$\omega$ & angular frequency & \si{\radian} \\

\end{symbols}



%----------------------------------------------------------------------------------------
%	ACRONYMS
%----------------------------------------------------------------------------------------

\newcommand{\listacronymname}{List of Acronyms}
\iflanguage{portuguese}{
\renewcommand{\listacronymname}{Lista de Acr\'onimos}
}

%Use GLS
\glsresetall
\printglossary[title=\listacronymname,type=\acronymtype,style=long]

%----------------------------------------------------------------------------------------
%	DONE
%----------------------------------------------------------------------------------------

\mainmatter % Begin numeric (1,2,3...) page numbering
\pagestyle{thesis} % Return the page headers back to the "thesis" style


%----------------------------------------------------------------------------------------
%	MAIN BODY
%----------------------------------------------------------------------------------------

% Include the chapters of the thesis as separate folder for each chapter
% Uncomment the lines as you write the chapters

\chapter{Problema}
\label{chap:pre1}

Neste capítulo são apresentados o enunciado do problema (secção~\ref{sec:pre1_problem}), os objetivos do trabalho, o âmbito e pressupostos associados (secção~\ref{sec:pre1_objectives}) e, por fim, os critérios de sucesso a considerar para a avaliação final do trabalho (seccção~\ref{sec:pre1_success_criteria}).

\section{Enunciado do problema}
\label{sec:pre1_problem}

O conceito de \gls{MES}, um sistema que, além de gerir as operações dum determinado processo fabril, mantém dados relativos às diversas etapas inerentes ao processo em questão, está intrinsecamente relacionado com a Indústria 4.0. O Critical Manufacturing \gls{MES} é um destes sistemas. Contudo, a sua incapacidade parcial de adaptar-se às características dos utilizadores, torna-o difícil de usar, numa perspetiva de acesso a informação relevante para o processo e de apoio à decisão. Por outras palavras, se o utilizador pretende efetuar uma determinada pesquisa, necessita de conhecer os detalhes da ferramenta a usar, ao invés de simplesmente \inquotes{pedir} (através de texto ou voz) ao sistema que lhe devolva o resultado.

A \textbf{conceção de um módulo de linguagem natural para interface com o Critical Manufacturing \gls{MES}, permitindo a consulta e pesquisa de estados do processo de fabrico}, torna-se importante para o sistema, uma vez que possibilita o utilizador interagir com o sistema de uma forma simples, intuitiva, eficiente e natural, através de escrita.

\section{Objetivos}
\label{sec:pre1_objectives}
De uma maneira geral, com este trabalho pretende-se desenvolver uma solução baseada em linguagem natural que promova a interação do utilizador com o sistema \gls{MES}. Com o intuito de solucionar o problema enunciado na secção~\ref{sec:pre1_problem}, definem-se os seguintes objetivos:

\begin{enumerate}
    \item
    \label{enum:pre1_objectives_1}
    {
        \textit{Contextualizar o problema numa perspetiva de negócio} -- análise detalhada do problema, as implicações que tem para negócio e para o produto \gls{MES}, descrevendo o valor intrínseco à solução (Capítulo~\ref{chap:Chapter2});
    }
    \item
    \label{enum:pre1_objectives_2}
    {
        \textit{Estudar soluções disponíveis no mercado e/ou bibliotecas de processamento de linguagem natural} -- obtenção de informação da área de conhecimento envolvida, de ferramentas semelhantes e de bibliotecas tipicamente usadas na implementação de tais módulos (Capítulo~\ref{chap:Chapter3});
    }
    \item
    \label{enum:pre1_objectives_3}
    {
        \textit{Definir a solução mais adequada, considerando as diversas opções apresentadas} -- comparação e avaliação das diversas opções identificadas, selecionando a(s) mais adequada(s), justificando essa(s) escolha(s) (Capítulo~\ref{chap:Chapter3});
    }
    \item
    \label{enum:pre1_objectives_4}
    {
        \textit{Especificação da arquitetura do módulo} -- que permita responder aos requisitos definidos e antecipar soluções para possíveis problemas (\textit{a definir});
    }
    \item
    \label{enum:pre1_objectives_5}
    {
        \textit{Descrever a semântica de domínio} -- identificação dos domínios a explorar e construção de uma base de conhecimento semântico para o módulo (\textit{a definir});
    }
    \item
    \label{enum:pre1_objectives_6}
    {
        \textit{Desenvolvimento de prova de conceito} -- implementação da solução de acordo com a arquitetura conceptualizada (\textit{a definir});
    }
    \item
    \label{enum:pre1_objectives_7}
    {
        \textit{Avaliar a qualidade da solução desenvolvida} -- com base nas estratégias de avaliação definidas em~\ref{enum:pre1_qualitystrategies}, concluir acerca da qualidade da solução e do contributo do trabalho para a resolução do problema (\textit{a definir});
    }
    \item
    \label{enum:pre1_objectives_8}
    {
        \textit{Elaboração da tese escrita} -- como forma de transmitir o conhecimento alcançado durante a elaboração do trabalho.
    }
\end{enumerate}

\section{Âmbito}

Embora os objetivos estejam definidos, surge a necessidade de explicitar sucintamente o âmbito do trabalho, bem como os pressupostos a ter em consideração. Por conseguinte, os seguintes assuntos não serão abordados:

\begin{itemize}
    \item
    {
        O enquadramento do problema com outros sistemas \gls{MES}. Apenas é contemplada a realidade do problema no contexto do Critical Manufacturing \gls{MES} (objetivo~\ref{enum:pre1_objectives_1});
    }
    \item
    {
        As soluções e bibliotecas de linguagem natural que não mostrem evidências de relevância para o problema, tendo em conta os critérios de preço, adesão da comunidade de desenvolvimento e respetiva complexidade (objetivo~\ref{enum:pre1_objectives_2});
    }
    \item 
    {
        A inclusão de diferentes domínios na solução desenvolvida (objetivos~\ref{enum:pre1_objectives_3}, \ref{enum:pre1_objectives_4} e \ref{enum:pre1_objectives_5});
    }
    \item 
    {
        A integração com o Critical Manufacturing \gls{MES}, apesar de ser levada em consideração aquando o desenho da solução (objetivos~\ref{enum:pre1_objectives_4}, \ref{enum:pre1_objectives_6} e \ref{enum:pre1_objectives_7}).
    }
\end{itemize}

O termo \inquotes{Domínio} é empregue ao longo do texto para denotar um conjunto de características que descrevem uma família de conceitos comuns a um determinado processo. Por exemplo, duas empresas que produzem equipamentos médicos, apesar de poderem ter processos de fabrico diferentes, abordam o mesmo domínio.

Neste trabalho assume-se que a solução a desenvolver, embora pensada para integrar em diversos (diferentes) processos de manufatura, é uma prova de conceito, pelo que deverá considerar um processo fabril (a ser definido) e consequentemente, lidar com a semântica específica desse domínio.

\section{Critérios de Sucesso}
\label{sec:pre1_success_criteria}

A avaliação do resultado final é imprescindível para concluir acerca do trabalho. Dessa forma, especificam-se a metodologia de avaliação para solução desenvolvida e os critérios de sucesso a serem considerados.

Para metodologia de avaliação deste trabalho foram definidas as seguintes estratégias:

\begin{enumerate}
\label{enum:pre1_qualitystrategies}
    \item 
    {
        \textit{Garantir que a solução analisa e responde corretamente a um conjunto de perguntas pré-definidas} -- a solução deverá responder adequadamente a um conjunto limitado de perguntas:
        \begin{itemize}
            \item 
            {
                Quantas operações foram executadas por semana, durante o mês M?
            }
            \item
            {
                Qual o número de operações O por produto e turno, durante o mês M?
            }
            \item
            {
                Qual a média de X de operações O, no passo P do processo, por turno, no mês M? 
            }
            \item
            {
                Qual o número de materiais cujo valor de X é inferior a Y, para o passo P do processo, agrupando por G?
            }
        \end{itemize}
        
        Nas questões apresentadas, as letras representam as variáveis próprias do domínio, que o utilizador conhece e que o sistema deve ser capaz de reconhecer.
    }
    \item
    {
        \textit{Prover a solução de um mecanismo de feedback para auto-aprendizagem} -- o que permitirá ao módulo adaptar-se às necessidades do utilizador, melhorando a qualidade das suas respostas. Numa fase inicial, este mecanismo consiste simplesmente em questionar o utilizador sobre a exatidão da resposta apresentada.
    }
\end{enumerate}

De seguida, enumeram-se os critérios de sucesso para o trabalho, associados aos respetivos objetivos:

\begin{enumerate}
    \item 
    {
        \textit{Tese escrita} -- na qual se abordam o problema, o contexto no qual se insere e o valor que traz ao produto final. Deve conter o estado da arte, apresentando a revisão da literatura existente, focando nas soluções semelhantes e/ou ferramentas relevantes que perspetivam estratégias de solução para o problema. Por fim, descreve-se a solução proposta, contemplando cada uma das fases inerentes ao seu desenvolvimento (visão, análise, desenho e implementação) e faz-se a conclusão acerca do trabalho (todos os objetivos descritos em~\ref{sec:pre1_objectives});
    }
    \item
    {
        \textit{A solução apresentada é extensível a outros domínios e facilmente integrada no Critical Manufacturing \gls{MES}} -- garante-se assim que a arquitetura especificada considerou diversos domínios, facilidade e capacidade de integração com o produto, ainda mesmo sendo um protótipo (objetivos~\ref{enum:pre1_objectives_3}, \ref{enum:pre1_objectives_4} e \ref{enum:pre1_objectives_6});
    }
    \item
    {
        \textit{Prova de conceito dá resposta correta ao domínio definido} -- que implica responder corretamente às questões listadas em \ref{enum:pre1_qualitystrategies}, garantindo que semântica foi bem definida e que o requisito de qualidade está cumprido (objetivos~\ref{enum:pre1_objectives_5}, \ref{enum:pre1_objectives_6} e \ref{enum:pre1_objectives_7}).
    }
\end{enumerate}

% ----------------------------------------
% CONTRIBUIÇÕES
% ----------------------------------------
\chapter{Contribuições Esperadas}
\label{chap:pre2}

% ----------------------------------------
% METODO DE TRABALHO
% ----------------------------------------
%\chapter{Método de trabalho}
%\label{chap:pre3}
%\textit{A definir}

% ----------------------------------------
% CONTEXTO
% ---------------------------------------- 
\chapter{Contexto}
\label{chap:Chapter2}

\textit{A definir}

\section{A Empresa}
\label{sec:chap2_company}

A {\companyname} é uma empresa fundada em 2009, com sede e centro de engenharia na Maia (Porto, Portugal), subsidiárias em Dresden (Alemanha), Suzhou (China), Austin (Estados Unidos da América) e um escritório comercial em Taiwan. O objetivo é proporcionar à indústria uma solução de gestão e controlo de produção, procurando reduzir os custos de produção, flexibilizar para satisfazer a procura e capacitar a organização de uma maior agilidade, visibilidade e fiabilidade~\parencite{cmf_overview}. O compromisso da empresa~\parencite{cmf_overview} foca-se no desenvolvimento de~\inquotes{soluções de vanguarda, indo de encontro aos desafios mais importantes da indústria e disponibilizar à lista crescente de clientes satisfeitos, soluções de elevado valor acrescentado, no prazo e orçamento requerido}\footnote{Tradução livre do autor. No original~\inquotes{[...] solutions that address the most urgent industry challenges and provide our growing list of satisfied customers with the highest value solution, on-time and on-budget.}.}.

A estratégia da empresa está sintetizada na sua missão, visão e valores. Se a missão descreve a razão da empresa existir, ou seja, o seu propósito, já a visão retrata o que se aspira alcançar~\parencite[pp.~65-66]{mission_vision_values_what_do_they_say}. Isto posto, a missão e visão são divulgados a seguir~\parencite{cmf_strategy}: 

\begin{itemize}
    \item 
    {
        \textit{Missão} -- \inquotes{Trazer valor através da convergência de inteligência, operações e tecnologias de automação para a Indústria 4.0.}\footnote{Tradução livre do autor. No original \inquotes{We drive business value through the convergence of intelligence, operations, and automation technologies for Industry 4.0.}.}.
    }
    \item 
    {
        \textit{Visão} -- \inquotes{Tornar a Indústria 4.0 uma realidade para todos fabricantes.}\footnote{Tradução livre do autor. No original \inquotes{We will make Industry 4.0 a reality for all manufacturers.}.}.
    }
\end{itemize}

Relativamente aos valores, são estes que suportam a visão, moldam a cultura empresarial e são a essência da sua identidade. Como tal, de seguida apresentam-se os valores da Critical Manufacturing~\parencite{cmf_strategy}:

\begin{itemize}
    \item 
    {
        \textit{Inovação} -- \inquotes{Exceder as expectativas dos clientes através das soluções mais eficientes e de mais alto valor para indústria.}\footnote{Tradução livre do autor. No original \inquotes{We constantly exceed our customers’ expectations through the most efficient and high value-added manufacturing solutions.}.}.
    }
    \item
    {
        \textit{Agilidade} -- \inquotes{Adaptar as pessoas, processos e soluções de forma a responder à evolução do mundo da manufatura de alta tecnologia.}\footnote{Tradução livre do autor. No original \inquotes{We continuously adapt our people, processes and solutions to respond to the evolving world of high-tech manufacturing.}.}.
    }
    \item
    {
        \textit{Compromisso} -- \inquotes{Defender o sucesso contínuo dos clientes e da empresa.}\footnote{Tradução livre do autor. No original \inquotes{We champion the continued success of our customers and our company.}.}.
    }
\end{itemize}

Por conseguinte, com base na sua estratégia, o presente trabalho pretende demonstrar viabilidade e o valor da utilização da \gls{IA}, nomeadamente na área do \gls{PLN}, para a interação com o {\productname} e obter informação pertinente para o utilizador final. 

\section{O Produto}
\label{sec:chap2_product}

Nos últimos anos, o mercado dos sistemas de informação empresariais tem vindo a crescer, sobretudo pela necessidade das empresas aumentarem a sua produtividade e consequentemente, melhorarem a sua competitividade. Embora sistemas \gls{ERP} sejam cada vez mais usuais nas empresas, no sentido de gerir as suas operações, estes falham quando aplicados num contexto fabril, ou seja, no \inquotes{chão de fábrica}. Os departamentos produtivos beneficiam de \textit{software} personalizado, que responda às necessidades específicas do foro produtivo/industrial~\parencite{mes_literature_review}. 

Nestas circunstâncias surge o conceito de \gls{MES}, fruto da necessidade das empresas de manufatura progredirem no mercado, num ponto de vista de melhoria da reatividade, da qualidade, dos custo de produção e dos prazos de entrega. Desse modo, as funções de um \gls{MES} estão sobretudo ligadas a atividades de manufatura, que representa uma parte substancial do valor acrescentado em empresas deste setor~\parencite{mes_literature_review}. 

Com o objetivo de apresentar o produto, nesta secção faz-se um enquadramento genérico do conceito \gls{MES} e posteriormente, foca-se o caso específico do {\productname}.

\subsection{\textit{Manufacturing Execution Systems}}

A organização MESA\footnote{Manufacturing Enterprise Solutions Association. \url{http://www.mesa.org}}, uma comunidade mundial, sem fins lucrativos, que junta empresas de manufatura, de prestação de serviços, analistas, académicos e estudantes, com o propósito de melhorar os resultados do negócio e as operações de produção, através da implementação e implantação de tecnologias de informação e das melhores práticas de gestão, deu o primeiro passo na definição formal de \gls{MES}~\parencite{mes_explained_high_level_vision}:

\begin{quote}
    \inquotes{\textit{Os Manufacturing Execution Systems (MES) fornecem informações que possibilitam a otimização de atividades de produção, desde o lançamento do pedido até aos produtos acabados. Usando dados atualizados e precisos, o MES orienta, inicia, responde e relata as atividades da fábrica à medida que elas ocorrem. A resposta rápida, resultante das mudanças nas condições, associada ao foco na redução de atividades sem valor acrescentado, impulsiona a eficácia das operações e processos fabris. O MES melhora o retorno dos ativos operacionais, bem como o prazo de entrega, gestão de stock, margem bruta e desempenho do fluxo de caixa. O MES fornece informações críticas acerca das atividades de produção em toda a empresa e cadeia logística através de comunicações bidirecionais.}}\footnote{Tradução livre do autor. No original \inquotes{Manufacturing Execution Systems (MES) deliver information that enables the optimization of production activities from order launch to finished goods. Using current and accurate data, MES guides, initiates, responds to, and reports on plant activities as they occur. The resulting rapid response to changing conditions, coupled with a focus on reducing non value-added activities, drives effective plant operations and processes. MES improves the return on operational assets as well as on-time delivery, inventory turns, gross margin, and cash flow performance. MES provides mission-critical information about production activities across the enterprise and supply chain via bi-directional communications.}.}.
\end{quote}

Portanto, o sistema \gls{MES} age como um intermediário entre os diversos processos existentes no \inquotes{chão de fábrica} e os sistemas de \inquotes{alto nível}, existindo comunicação bidirecional entre as camadas, como se demonstra na Figura~\ref{fig:mes_layers}. O \gls{MES} tanto pode fornecer informação acerca dos custos de produção, de indicadores de \textit{performance}, do estado das ordens de fabrico ou rendimento produtivo, como pode também obter dados sobre o planeamento das atividades fabris, parâmetros operacionais, receitas ou instruções de fabrico, por forma a inferir de forma inteligente sobre a fábrica e os seus processos~\parencite{mes_explained_high_level_vision}.

\begin{figure}[!ht]
    \centering
    \includegraphics[width=.75\textwidth]{ch2/assets/mes_layers.jpg}
    \caption{Ambiente \gls{MES} e as suas camadas, baseado em~\textcite[p.~526]{mes_literature_review}.}
    \label{fig:mes_layers}
\end{figure}

Com o intuito de dar resposta às necessidades de diversos ambientes produtivos, as funções apresentadas de seguida são essenciais para um \gls{MES}~\ref{fig:mes_functions}

\begin{figure}[!ht]
    \centering
    \includegraphics[width=\textwidth]{ch2/assets/mes_functions.jpg}
    \caption{bla}
    \label{fig:mes_functions}
\end{figure}


\subsection{\textit{{\productname}}}

\section{Análise de Valor}
\label{sec:chap2_valueanalysis}

\textit{A definir}

% ----------------------------------------
% ESTADO DA ARTE
% ---------------------------------------- 
% Chapter 3

\chapter{Estado da Arte}
\label{chap:Chapter3}

%----------------------------------------------------------------------------------------
%	
%----------------------------------------------------------------------------------------


% ----------------------------------------
% PROPOSTA DE SOLUÇÃO
% ---------------------------------------- 
% Se necessário....

% ----------------------------------------
% PLANO DE TRABALHO
% ---------------------------------------- 
\chapter{Plano de Trabalho}
\label{chap:pre7}
\textit{A definir}
%\chapter{Introdução}
\label{chap:chapter1}

A indústria desempenha um papel importante na economia mundial. Desde o início da industrialização, fatores como a evolução tecnológica, as forças políticas e económicas, a competitividade de mercado, levam a mudanças no paradigma industrial, as quais se designam de revoluções industriais. A primeira revolução inicia-se com a mecanização dos processos, segue-se a segunda com o uso da energia elétrica e posteriormente a terceira revolução, fruto da digitalização e informatização industrial~\parencite{industry40, modern_industrial_revolution_exit_failure_internal_control_systems}.

Atualmente, num mercado crescentemente competitivo e exigente, a necessidade de inovar, de obter vantagem competitiva e simultaneamente, tornar os processos industriais simples e altamente eficazes, recorrendo às tecnologias mais atuais, abrem caminho a uma nova mudança. O fenómeno da Indústria 4.0 surge como a nova (quarta) revolução industrial, baseando-se nas mais recentes tecnologias, que incluem os sistemas ciber-físicos, a~\gls{IoT} e a~\gls{IoS}, as quais se baseiam na comunicação através da Internet, permitindo uma interação contínua e partilha de informação entre humanos, entre máquinas e entre o ser humano e máquina~\parencite{complex_view_industry40}. 

A Indústria 4.0 assenta numa variedade de conceitos fundamentais, de diferentes áreas de conhecimento, nomeadamente a noção de \textit{Smart Factory\footnote{Fábrica Inteligente, equipada com sensores, atuadores e sistemas autónomos, permitindo assim um controlo autónomo de processo.}}, a capacidade de auto-organização, através da descentralização dos sistemas produtivos, e a interação entre o mundo físico e o digital~\parencite[Fundamental Concepts, p.240]{industry40}. No entanto, é a capacidade de adaptação à necessidade humana, principalmente a \gls{IHC}, que se pretende explorar com presente trabalho.

Segundo~\textcite[p.1]{natural_language_translation_intersaction_ai_hci}~\inquotes{as áreas de \gls{IA} e Interação Humano-Computador (\gls{IHC}) estão, cada vez mais, a influenciar-se mutuamente. Sistemas amplamente usados como o Google Translate, Facebook Graph Search e RelateIQ escondem a complexidade de sistemas de larga escala de \gls{IA} através de interfaces intuitivas.}\footnote{Tradução livre do autor. No original~\inquotes{The fields of artificial intelligence (AI) and human-computer interaction (HCI) are influencing each other like never before. Widely used systems such as Google Translate, Facebook Graph Search, and RelateIQ hide the complexity of large-scale AI systems behind intuitive interfaces.}.}. Apesar de terem propósitos diferentes, ambas as áreas se complementam, na medida em que se focam na relação entre ser humano e máquina. Se a \gls{IA} tem como objetivo emular o intelecto humano, já a \gls{IHC} foca-se em abordagens empíricas de usabilidade e fatores humanos, que influenciam a forma como os utilizadores interagem com o computador~\parencite{natural_language_translation_intersaction_ai_hci}. 

A capacidade dum sistema interpretar a linguagem dos seres humanos e apresentar a informação de uma forma adequada, principalmente no contexto da Indústria 4.0, destaca-se como um fator impulsionador da adaptabilidade do mundo digital à necessidade humana. Nesse sentido, a área de \gls{PLN}, a qual se debruça na capacidade dos computadores \inquotes{entenderem} a linguagem humana~\parencite[p.1]{applied_natural_language_processing_with_python}, permite construir ferramentas capazes de definir ações, extrair conhecimento dum sistema e apresentá-lo num formato adequado, a partir de conteúdo textual especificado pelo utilizador, de acordo com a sua própria linguagem. 

\section{Problema} 
\label{sec:chap1_problem}

Num contexto da Industria 4.0, o conceito de \gls{MES}

\textbf{Conceção de módulo de linguagem natural para interface com o Critical Manufacturing MES, permitindo a consulta e pesquisa de estados do processo de fabrico.}

\section{Objetivos}
\label{sec:chap1_objectives}
\hl{...}

\section{Âmbito}
\label{sec:chap1_scope}
\hl{...}

\section{Critérios de Sucesso}
\label{sec:chap1_success_criteria}
\hl{...}

\section{Contribuições}
\label{sec:chap1_contribuitions}
\hl{...}

\section{Metodologia de trabalho}
\label{sec:chap1_methodology}
\hl{...}

\section{Estrutura da tese}
\label{sec:chap1_structure}
\hl{...}

%\chapter{Contexto}
\label{chap:Chapter2}

\textit{A definir}

\section{A Empresa}
\label{sec:chap2_company}

A {\companyname} é uma empresa fundada em 2009, com sede e centro de engenharia na Maia (Porto, Portugal), subsidiárias em Dresden (Alemanha), Suzhou (China), Austin (Estados Unidos da América) e um escritório comercial em Taiwan. O objetivo é proporcionar à indústria uma solução de gestão e controlo de produção, procurando reduzir os custos de produção, flexibilizar para satisfazer a procura e capacitar a organização de uma maior agilidade, visibilidade e fiabilidade~\parencite{cmf_overview}. O compromisso da empresa~\parencite{cmf_overview} foca-se no desenvolvimento de~\inquotes{soluções de vanguarda, indo de encontro aos desafios mais importantes da indústria e disponibilizar à lista crescente de clientes satisfeitos, soluções de elevado valor acrescentado, no prazo e orçamento requerido}\footnote{Tradução livre do autor. No original~\inquotes{[...] solutions that address the most urgent industry challenges and provide our growing list of satisfied customers with the highest value solution, on-time and on-budget.}.}.

A estratégia da empresa está sintetizada na sua missão, visão e valores. Se a missão descreve a razão da empresa existir, ou seja, o seu propósito, já a visão retrata o que se aspira alcançar~\parencite[pp.~65-66]{mission_vision_values_what_do_they_say}. Isto posto, a missão e visão são divulgados a seguir~\parencite{cmf_strategy}: 

\begin{itemize}
    \item 
    {
        \textit{Missão} -- \inquotes{Trazer valor através da convergência de inteligência, operações e tecnologias de automação para a Indústria 4.0.}\footnote{Tradução livre do autor. No original \inquotes{We drive business value through the convergence of intelligence, operations, and automation technologies for Industry 4.0.}.}.
    }
    \item 
    {
        \textit{Visão} -- \inquotes{Tornar a Indústria 4.0 uma realidade para todos fabricantes.}\footnote{Tradução livre do autor. No original \inquotes{We will make Industry 4.0 a reality for all manufacturers.}.}.
    }
\end{itemize}

Relativamente aos valores, são estes que suportam a visão, moldam a cultura empresarial e são a essência da sua identidade. Como tal, de seguida apresentam-se os valores da Critical Manufacturing~\parencite{cmf_strategy}:

\begin{itemize}
    \item 
    {
        \textit{Inovação} -- \inquotes{Exceder as expectativas dos clientes através das soluções mais eficientes e de mais alto valor para indústria.}\footnote{Tradução livre do autor. No original \inquotes{We constantly exceed our customers’ expectations through the most efficient and high value-added manufacturing solutions.}.}.
    }
    \item
    {
        \textit{Agilidade} -- \inquotes{Adaptar as pessoas, processos e soluções de forma a responder à evolução do mundo da manufatura de alta tecnologia.}\footnote{Tradução livre do autor. No original \inquotes{We continuously adapt our people, processes and solutions to respond to the evolving world of high-tech manufacturing.}.}.
    }
    \item
    {
        \textit{Compromisso} -- \inquotes{Defender o sucesso contínuo dos clientes e da empresa.}\footnote{Tradução livre do autor. No original \inquotes{We champion the continued success of our customers and our company.}.}.
    }
\end{itemize}

Por conseguinte, com base na sua estratégia, o presente trabalho pretende demonstrar viabilidade e o valor da utilização da \gls{IA}, nomeadamente na área do \gls{PLN}, para a interação com o {\productname} e obter informação pertinente para o utilizador final. 

\section{O Produto}
\label{sec:chap2_product}

Nos últimos anos, o mercado dos sistemas de informação empresariais tem vindo a crescer, sobretudo pela necessidade das empresas aumentarem a sua produtividade e consequentemente, melhorarem a sua competitividade. Embora sistemas \gls{ERP} sejam cada vez mais usuais nas empresas, no sentido de gerir as suas operações, estes falham quando aplicados num contexto fabril, ou seja, no \inquotes{chão de fábrica}. Os departamentos produtivos beneficiam de \textit{software} personalizado, que responda às necessidades específicas do foro produtivo/industrial~\parencite{mes_literature_review}. 

Nestas circunstâncias surge o conceito de \gls{MES}, fruto da necessidade das empresas de manufatura progredirem no mercado, num ponto de vista de melhoria da reatividade, da qualidade, dos custo de produção e dos prazos de entrega. Desse modo, as funções de um \gls{MES} estão sobretudo ligadas a atividades de manufatura, que representa uma parte substancial do valor acrescentado em empresas deste setor~\parencite{mes_literature_review}. 

Com o objetivo de apresentar o produto, nesta secção faz-se um enquadramento genérico do conceito \gls{MES} e posteriormente, foca-se o caso específico do {\productname}.

\subsection{\textit{Manufacturing Execution Systems}}

A organização MESA\footnote{Manufacturing Enterprise Solutions Association. \url{http://www.mesa.org}}, uma comunidade mundial, sem fins lucrativos, que junta empresas de manufatura, de prestação de serviços, analistas, académicos e estudantes, com o propósito de melhorar os resultados do negócio e as operações de produção, através da implementação e implantação de tecnologias de informação e das melhores práticas de gestão, deu o primeiro passo na definição formal de \gls{MES}~\parencite{mes_explained_high_level_vision}:

\begin{quote}
    \inquotes{\textit{Os Manufacturing Execution Systems (MES) fornecem informações que possibilitam a otimização de atividades de produção, desde o lançamento do pedido até aos produtos acabados. Usando dados atualizados e precisos, o MES orienta, inicia, responde e relata as atividades da fábrica à medida que elas ocorrem. A resposta rápida, resultante das mudanças nas condições, associada ao foco na redução de atividades sem valor acrescentado, impulsiona a eficácia das operações e processos fabris. O MES melhora o retorno dos ativos operacionais, bem como o prazo de entrega, gestão de stock, margem bruta e desempenho do fluxo de caixa. O MES fornece informações críticas acerca das atividades de produção em toda a empresa e cadeia logística através de comunicações bidirecionais.}}\footnote{Tradução livre do autor. No original \inquotes{Manufacturing Execution Systems (MES) deliver information that enables the optimization of production activities from order launch to finished goods. Using current and accurate data, MES guides, initiates, responds to, and reports on plant activities as they occur. The resulting rapid response to changing conditions, coupled with a focus on reducing non value-added activities, drives effective plant operations and processes. MES improves the return on operational assets as well as on-time delivery, inventory turns, gross margin, and cash flow performance. MES provides mission-critical information about production activities across the enterprise and supply chain via bi-directional communications.}.}.
\end{quote}

Portanto, o sistema \gls{MES} age como um intermediário entre os diversos processos existentes no \inquotes{chão de fábrica} e os sistemas de \inquotes{alto nível}, existindo comunicação bidirecional entre as camadas, como se demonstra na Figura~\ref{fig:mes_layers}. O \gls{MES} tanto pode fornecer informação acerca dos custos de produção, de indicadores de \textit{performance}, do estado das ordens de fabrico ou rendimento produtivo, como pode também obter dados sobre o planeamento das atividades fabris, parâmetros operacionais, receitas ou instruções de fabrico, por forma a inferir de forma inteligente sobre a fábrica e os seus processos~\parencite{mes_explained_high_level_vision}.

\begin{figure}[!ht]
    \centering
    \includegraphics[width=.75\textwidth]{ch2/assets/mes_layers.jpg}
    \caption{Ambiente \gls{MES} e as suas camadas, baseado em~\textcite[p.~526]{mes_literature_review}.}
    \label{fig:mes_layers}
\end{figure}

Com o intuito de dar resposta às necessidades de diversos ambientes produtivos, as funções apresentadas de seguida são essenciais para um \gls{MES}~\ref{fig:mes_functions}

\begin{figure}[!ht]
    \centering
    \includegraphics[width=\textwidth]{ch2/assets/mes_functions.jpg}
    \caption{bla}
    \label{fig:mes_functions}
\end{figure}


\subsection{\textit{{\productname}}}

\section{Análise de Valor}
\label{sec:chap2_valueanalysis}

\textit{A definir}
%% Chapter 3

\chapter{Estado da Arte}
\label{chap:Chapter3}

%----------------------------------------------------------------------------------------
%	
%----------------------------------------------------------------------------------------

%% Chapter 4

\chapter{Solução Proposta}
\label{chap:Chapter4}

%----------------------------------------------------------------------------------------
%	VISÃO GERAL
%----------------------------------------------------------------------------------------

\section{Visão Geral} 
\label{sec:chap4_general_vision}
\hl{...}

%----------------------------------------------------------------------------------------
%	CONCEÇÃO
%----------------------------------------------------------------------------------------

\section{Conceção} 
\label{sec:chap4_conception}
\hl{...}

%----------------------------------------------------------------------------------------
%	DESENVOLVIMENTO
%----------------------------------------------------------------------------------------

\section{Desenvolvimento} 
\label{sec:chap4_development}
\hl{...}

%----------------------------------------------------------------------------------------
%	DESENVOLVIMENTO
%----------------------------------------------------------------------------------------

\section{Operação} 
\label{sec:chap4_operation}
\hl{...}

%----------------------------------------------------------------------------------------
%	DESENVOLVIMENTO
%----------------------------------------------------------------------------------------

\section{Validação} 
\label{sec:chap4_validation}
\hl{...}
%% Chapter 5
\chapter{Conclusões}
\label{chap:Chapter5}

%----------------------------------------------------------------------------------------
%	VISÃO GERAL
%----------------------------------------------------------------------------------------

\section{Avaliação de objetivos} 
\label{sec:chap5_goals_evaluation}
\hl{...}

%----------------------------------------------------------------------------------------
%	CONCEÇÃO
%----------------------------------------------------------------------------------------

\section{Resposta ao problema} 
\label{sec:chap5_problem_response}
\hl{...}

%----------------------------------------------------------------------------------------
%	LIMITAÇõES E TRABALHO FUTURO
%----------------------------------------------------------------------------------------

\section{Limitações e trabalho futuro} 
\label{sec:chap5_future_work_limitations}
\hl{...}



%----------------------------------------------------------------------------------------
%	BIBLIOGRAPHY
%----------------------------------------------------------------------------------------

\printbibliography[heading=bibintoc]

%----------------------------------------------------------------------------------------
%	THESIS CONTENT - APPENDICES
%----------------------------------------------------------------------------------------

\appendix % Cue to tell LaTeX that the following "chapters" are Appendices

% Include the appendices of the thesis as separate files from the Appendices folder
% Uncomment the lines as you write the Appendices

%% Appendix A

\chapter{Apêndice Teste} % Main appendix title

\label{AppendixA} % For referencing this appendix elsewhere, use \ref{AppendixA}

Write your Appendix content here.
%\input{appendices/appendixB}
%\chapter{Protótipo}
\label{AppendixC}
Neste apêndice são mostrados alguns artefactos recolhidos ao longo das fases de conceção, desenvolvimento e validação do protótipo.

\section{Configuração}
Nesta secção apresentam-se algumas imagens referentes ao processo de configuração levado no protótipo.
%
\begin{figure}[!ht]
    \centering
    \includegraphics[width=\textwidth]{appendices/assets/kb07.png}
    \caption{Definição das entidades esperadas}
\end{figure}
%
\begin{figure}
\centering
    \begin{subfigure}{.9\textwidth}
        \centering
        \includegraphics[width=\textwidth]{appendices/assets/kb01.png}
        \caption{Intenção \textit{AverageOperationOnStep}}
     \end{subfigure}
     \begin{subfigure}{.9\textwidth}
         \centering
        \includegraphics[width=\textwidth]{appendices/assets/kb02.png}
        \caption{Intenção \textit{CountMaterialsOnCondition}}
     \end{subfigure}
     \begin{subfigure}{.9\textwidth}
        \centering
        \includegraphics[width=\textwidth]{appendices/assets/kb03.png}
        \caption{Intenção \textit{CountMaterialsOnConditionOnStep}}
     \end{subfigure}
\caption{Intenções definidas, contendo as expressões e respetivas entidades}
\end{figure}
%
\begin{figure}
    \centering
         \begin{subfigure}{.9\textwidth}
        \centering
        \includegraphics[width=\textwidth]{appendices/assets/kb04.png}
        \caption{Intenção \textit{CountMaterialsOnConditionOnStepGrouped}}
     \end{subfigure}
     \begin{subfigure}{.9\textwidth}
        \centering
        \includegraphics[width=\textwidth]{appendices/assets/kb05.png}
        \caption{Intenção \textit{CountOperationsByProduct}}
     \end{subfigure}
     \begin{subfigure}{.9\textwidth}
        \centering
        \includegraphics[width=\textwidth]{appendices/assets/kb06.png}
        \caption{Intenção \textit{SumOperationsByWeek}}
     \end{subfigure}
    \caption{Continuação das intenções definidas, contendo as expressões e respetivas entidades}
\end{figure}

\clearpage

\section{Validação}
Nesta secção apresentam-se algumas imagens referentes ao processo de validação do protótipo.

\begin{figure}[!ht]
\centering
    \begin{subfigure}{.48\textwidth}
        \centering
        \includegraphics[width=\textwidth]{appendices/assets/nlcomprehension01.png}
        \caption{Intenção \textit{SumOperationsByWeek}}
     \end{subfigure}
     \begin{subfigure}{.48\textwidth}
         \centering
        \includegraphics[width=\textwidth]{appendices/assets/nlcomprehension02.png}
        \caption{Intenção \textit{CountOperationByProduct}}
     \end{subfigure}
     \bigbreak
     \begin{subfigure}{.48\textwidth}
        \centering
        \includegraphics[width=\textwidth]{appendices/assets/nlcomprehension03.png}
        \caption{Intenção \textit{CountOperationsByProductPerShift}}
     \end{subfigure}
     \begin{subfigure}{.48\textwidth}
        \centering
        \includegraphics[width=\textwidth]{appendices/assets/nlcomprehension04.png}
        \caption{Intenção \textit{AverageOperationsOnStep}}
     \end{subfigure}
     \bigbreak
     \begin{subfigure}{.48\textwidth}
        \centering
        \includegraphics[width=\textwidth]{appendices/assets/nlcomprehension05.png}
        \caption{Intenção \textit{CountMaterialsOnCondition}}
     \end{subfigure}
     \begin{subfigure}{.48\textwidth}
        \centering
        \includegraphics[width=\textwidth]{appendices/assets/nlcomprehension06.png}
        \caption{Intenção \textit{CountOperationsOnConditionOnStep}}
     \end{subfigure}
\caption{Outras imagens relativas à avaliação de intenções e entidades do protótipo}
\label{fig:nlcomprehesion_others}
\end{figure}

%----------------------------------------------------------------------------------------

\end{document}
