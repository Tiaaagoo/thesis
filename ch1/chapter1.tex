\chapter{Introdução}
\label{chap:chapter1}

A indústria desempenha um papel importante na economia mundial. Desde o início da industrialização, fatores como a evolução tecnológica, as forças políticas e económicas, levam a mudanças no paradigma industrial, as quais se designam de revoluções industriais. A primeira revolução inicia-se com a mecanização dos processos, segue-se a segunda com o uso da energia elétrica e posteriormente a terceira revolução, fruto da digitalização e informatização industrial~\parencite{industry40, the_modern_industrial_revolution_exit_failure_internal_control_systems}.

Atualmente, num mercado crescentemente competitivo e exigente, a necessidade de inovar, de obter vantagem competitiva e simultaneamente, tornar os processos industriais simples e altamente eficazes, recorrendo às tecnologias mais atuais, abrem caminho a uma nova mudança. O fenómeno da Industria 4.0 surge como a nova (quarta) revolução industrial, baseando-se nas mais recentes tecnologias, que incluem os sistemas ciber-físicos, a~\gls{IoT} e a~\gls{IoS}, as quais se baseiam na comunicação através da Internet, permitindo uma interação contínua e partilha de informação entre humanos, entre máquinas e entre o ser humano e máquina~\parencite{complex_view_industry40}. 


\section{Problema} 
\label{sec:ch1_problem}
a
\hl{...}

\section{Objetivos}
\label{sec:ch1_objectives}

\hl{...}


\section{Âmbito}
\label{sec:ch1_scope}

\hl{...}



\section{Critérios de Sucesso}
\label{sec:ch1_success_criteria}

\hl{...}


\section{Contribuições}
\label{sec:ch1_contribuitions}

\hl{...}


\section{Metodologia de trabalho}
\label{sec:ch1_methodology}

\hl{...}


\section{Estrutura da tese}
\label{sec:ch1_structure}

\hl{...}
