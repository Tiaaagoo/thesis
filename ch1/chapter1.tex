\chapter{Introdução}
\label{chap:chapter1}

A indústria desempenha um papel importante na economia mundial. Desde o início da industrialização, fatores como a evolução tecnológica, as forças políticas e económicas, a competitividade de mercado, levam a mudanças no paradigma industrial, as quais se designam de revoluções industriais. A primeira revolução inicia-se com a mecanização dos processos, segue-se a segunda com o uso da energia elétrica e posteriormente a terceira revolução, fruto da digitalização e informatização industrial~\parencite{industry40, modern_industrial_revolution_exit_failure_internal_control_systems}.

Atualmente, num mercado crescentemente competitivo e exigente, a necessidade de inovar, de obter vantagem competitiva e simultaneamente, tornar os processos industriais simples e altamente eficazes, recorrendo às tecnologias mais atuais, abrem caminho a uma nova mudança. O fenómeno da Indústria 4.0 surge como a nova (quarta) revolução industrial, baseando-se nas mais recentes tecnologias, que incluem os sistemas ciber-físicos, a \gls{IoT} e a \gls{IoS}, as quais se baseiam na comunicação através da Internet, permitindo uma interação contínua e partilha de informação entre humanos, entre máquinas e entre o ser humano e máquina~\parencite{complex_view_industry40}. 

A Indústria 4.0 assenta numa variedade de conceitos fundamentais, de diferentes áreas de conhecimento, nomeadamente a noção de \textit{Smart Factory\footnote{Fábrica Inteligente, equipada com sensores, atuadores e sistemas autónomos, permitindo assim um controlo autónomo de processo.}}, a capacidade de auto-organização, através da descentralização dos sistemas produtivos, e a interação entre o mundo físico e o digital~\parencite[Fundamental Concepts, p.240]{industry40}. No entanto, é a capacidade de adaptação à necessidade humana, principalmente a \gls{IHC}, que se pretende explorar com presente trabalho.

Segundo~\textcite[p.1]{natural_language_translation_intersaction_ai_hci}~\inquotes{as áreas de \gls{IA} e Interação Humano-Computador (\gls{IHC}) estão, cada vez mais, a influenciar-se mutuamente. Sistemas amplamente usados como o Google Translate, Facebook Graph Search e RelateIQ escondem a complexidade de sistemas de larga escala de \gls{IA} através de interfaces intuitivas.}\footnote{Tradução livre do autor. No original~\inquotes{The fields of artificial intelligence (AI) and human-computer interaction (HCI) are influencing each other like never before. Widely used systems such as Google Translate, Facebook Graph Search, and RelateIQ hide the complexity of large-scale AI systems behind intuitive interfaces.}.}. Apesar de terem propósitos diferentes, ambas as áreas se complementam, na medida em que se focam na relação entre ser humano e máquina. Se a \gls{IA} tem como objetivo emular o intelecto humano, já a \gls{IHC} foca-se em abordagens empíricas de usabilidade e fatores humanos, que influenciam a forma como os utilizadores interagem com o computador~\parencite{natural_language_translation_intersaction_ai_hci}. 

A capacidade dum sistema interpretar a linguagem dos seres humanos e apresentar a informação de uma forma adequada, principalmente no contexto da Indústria 4.0, destaca-se como um fator impulsionador da adaptabilidade do mundo digital à necessidade humana. Nesse sentido, a área de \gls{PLN}, a qual se debruça na capacidade dos computadores \inquotes{entenderem} a linguagem humana~\parencite[p.1]{applied_natural_language_processing_with_python}, permite construir ferramentas capazes de definir ações, extrair conhecimento dum sistema e apresentá-lo num formato adequado, a partir de conteúdo textual especificado pelo utilizador, de acordo com a sua própria linguagem. 

\section{Problema}
\label{sec:chap1_problem}

O conceito de \gls{MES}, um sistema que, além de gerir as operações dum determinado processo fabril, mantém dados relativos às diversas etapas inerentes ao processo em questão, está intrinsecamente relacionado com a Indústria 4.0. O Critical Manufacturing \gls{MES} é um destes sistemas. Contudo, a sua incapacidade parcial de adaptar-se às características dos utilizadores, torna-o difícil de usar, numa perspetiva de acesso a informação relevante para o processo e de apoio à decisão. Por outras palavras, se o utilizador pretende efetuar uma determinada pesquisa, necessita de conhecer os detalhes da ferramenta a usar, ao invés de simplesmente \inquotes{pedir} (através de texto ou voz) ao sistema que lhe devolva o resultado.

A \textbf{conceção de um módulo de linguagem natural para interface com o Critical Manufacturing \gls{MES}, permitindo a consulta e pesquisa de estados do processo de fabrico}, torna-se importante para o sistema, uma vez que possibilita o utilizador interagir com o sistema de uma forma simples, intuitiva, eficiente e natural, através de escrita.

\section{Objetivos, Âmbito e Pressupostos}
\label{sec:chap1_objectives}
De uma maneira geral, com este trabalho pretende-se desenvolver uma solução baseada em linguagem natural que promova a interação do utilizador com o sistema \gls{MES}. Com o intuito de solucionar o problema enunciado na secção~\ref{sec:chap1_problem}, definem-se os seguintes objetivos:

\begin{enumerate}
    \item
    \label{enum:ch1_objectives_1}
    {
        \textit{Contextualizar o problema numa perspetiva de negócio} -- análise detalhada do problema, as implicações que tem para negócio e para o produto \gls{MES}, descrevendo o valor intrínseco à solução (Capítulo~\ref{chap:Chapter2});
    }
    \item
    \label{enum:ch1_objectives_2}
    {
        \textit{Estudar soluções disponíveis no mercado e/ou bibliotecas de processamento de linguagem natural} -- obtenção de informação da área de conhecimento envolvida, de ferramentas semelhantes e de bibliotecas tipicamente usadas na implementação de tais módulos (Capítulo~\ref{chap:Chapter3});
    }
    \item
    \label{enum:ch1_objectives_3}
    {
        \textit{Definir a solução mais adequada, considerando as diversas opções apresentadas} -- comparação e avaliação das diversas opções identificadas, selecionando a(s) mais adequada(s), justificando essa(s) escolha(s) (Capítulo~\ref{chap:Chapter3});
    }
    \item
    \label{enum:ch1_objectives_4}
    {
        \textit{Especificação da arquitetura do módulo} -- que permita responder aos requisitos definidos e antecipar soluções para possíveis problemas (\hl{a definir...});
    }
    \item
    \label{enum:ch1_objectives_5}
    {
        \textit{Descrever a semântica de domínio} -- identificação dos domínios a explorar e construção de uma base de conhecimento semântico para o módulo (\hl{a definir...});
    }
    \item
    \label{enum:ch1_objectives_6}
    {
        \textit{Desenvolvimento de prova de conceito} -- implementação da solução de acordo com a arquitetura conceptualizada (\hl{a definir...});
    }
    \item
    \label{enum:ch1_objectives_7}
    {
        \textit{Avaliar a qualidade da solução desenvolvida} -- com base nas metodologias de avaliação definidas, concluir acerca da qualidade da solução e do contributo do trabalho para a resolução do problema (\hl{a definir...});
    }
    \item
    \label{enum:ch1_objectives_8}
    {
        \textit{Elaboração da tese escrita} -- como forma de transmitir o conhecimento alcançado durante a elaboração do trabalho.
    }
\end{enumerate}

Embora os objetivos estejam definidos, surge a necessidade de explicitar sucintamente os assuntos que serão ou não abordados, ou seja, qual o âmbito do trabalho e os pressupostos a ter em consideração.

\section{Critérios de Sucesso}
\label{sec:chap1_success_criteria}
\hl{...}

\section{Contribuições Esperadas}
\label{sec:chap1_contribuitions}
\hl{...}

\section{Metodologia de Trabalho}
\label{sec:chap1_methodology}
\hl{...}

\section{Estrutura da Tese}
\label{sec:chap1_structure}
\hl{...}
