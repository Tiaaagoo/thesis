\chapter{Introdução}
\label{chap:chapter1}

A indústria desempenha um papel importante na economia mundial. Desde o início da industrialização, fatores como a evolução tecnológica, as forças políticas e económicas, a competitividade de mercado, levam a mudanças no paradigma industrial, as quais se designam de revoluções industriais. A primeira revolução inicia-se com a mecanização dos processos, segue-se a segunda com o uso da energia elétrica e posteriormente a terceira revolução, fruto da digitalização e informatização industrial~\parencite{industry40, modern_industrial_revolution_exit_failure_internal_control_systems}.

Atualmente, num mercado crescentemente competitivo e exigente, a necessidade de inovar, de obter vantagem competitiva e simultaneamente, tornar os processos industriais simples e altamente eficazes, recorrendo às tecnologias mais atuais, abrem caminho a uma nova mudança. O fenómeno da Indústria 4.0 surge como a nova (quarta) revolução industrial, baseando-se nas mais recentes tecnologias, que incluem os sistemas ciber-físicos, a~\gls{IoT} e a~\gls{IoS}, as quais se baseiam na comunicação através da Internet, permitindo uma interação contínua e partilha de informação entre humanos, entre máquinas e entre o ser humano e máquina~\parencite{complex_view_industry40}. 

A Indústria 4.0 assenta numa variedade de conceitos fundamentais, de diferentes áreas de conhecimento, nomeadamente a noção de \textit{Smart Factory\footnote{Fábrica Inteligente, equipada com sensores, atuadores e sistemas autónomos, permitindo assim um controlo autónomo de processo.}}, a capacidade de auto-organização, através da descentralização dos sistemas produtivos, e a interação entre o mundo físico e o digital~\parencite[Fundamental Concepts, p.240]{industry40}. No entanto, é a capacidade de adaptação à necessidade humana, principalmente a \gls{IHC}, que se pretende explorar com presente trabalho.

Segundo~\textcite[p.1]{natural_language_translation_intersaction_ai_hci}~\inquotes{as áreas de \gls{IA} e Interação Humano-Computador (\gls{IHC}) estão, cada vez mais, a influenciar-se mutuamente. Sistemas amplamente usados como o Google Translate, Facebook Graph Search e RelateIQ escondem a complexidade de sistemas de larga escala de \gls{IA} através de interfaces intuitivas.}\footnote{Tradução livre do autor. No original~\inquotes{The fields of artificial intelligence (AI) and human-computer interaction (HCI) are influencing each other like never before. Widely used systems such as Google Translate, Facebook Graph Search, and RelateIQ hide the complexity of large-scale AI systems behind intuitive interfaces.}.}. Apesar de terem propósitos diferentes, ambas as áreas se complementam, na medida em que se focam na relação entre ser humano e máquina. Se a \gls{IA} tem como objetivo emular o intelecto humano, já a \gls{IHC} foca-se em abordagens empíricas de usabilidade e fatores humanos, que influenciam a forma como os utilizadores interagem com o computador~\parencite{natural_language_translation_intersaction_ai_hci}. 

A capacidade dum sistema interpretar a linguagem dos seres humanos e apresentar a informação de uma forma adequada, principalmente no contexto da Indústria 4.0, destaca-se como um fator impulsionador da adaptabilidade do mundo digital à necessidade humana. Nesse sentido, a área de \gls{PLN}, a qual se debruça na capacidade dos computadores \inquotes{entenderem} a linguagem humana~\parencite[p.1]{applied_natural_language_processing_with_python}, permite construir ferramentas capazes de definir ações, extrair conhecimento dum sistema e apresentá-lo num formato adequado, a partir de conteúdo textual especificado pelo utilizador, de acordo com a sua própria linguagem. 

\section{Problema} 
\label{sec:chap1_problem}

Num contexto da Industria 4.0, o conceito de \gls{MES}

\textbf{Conceção de módulo de linguagem natural para interface com o Critical Manufacturing MES, permitindo a consulta e pesquisa de estados do processo de fabrico.}

\section{Objetivos}
\label{sec:chap1_objectives}
\hl{...}

\section{Âmbito}
\label{sec:chap1_scope}
\hl{...}

\section{Critérios de Sucesso}
\label{sec:chap1_success_criteria}
\hl{...}

\section{Contribuições}
\label{sec:chap1_contribuitions}
\hl{...}

\section{Metodologia de trabalho}
\label{sec:chap1_methodology}
\hl{...}

\section{Estrutura da tese}
\label{sec:chap1_structure}
\hl{...}
